\documentclass[11pt]{article}
\title{Assignment 1 Solutions}
\author{Aarush Dilawri}
\date{}
\setlength{\headheight}{13.59999pt}

\usepackage{enumitem}
\usepackage{fancyhdr}
\usepackage{parskip}
\usepackage[margin=2.3cm]{geometry} 
\usepackage{titlesec}
\usepackage{setspace}
\usepackage{graphicx}
\usepackage{amsfonts}



\usepackage{amsmath}


\setstretch{1.2}

% Page header/footer setup
\pagestyle{fancy}
\fancyhf{}
\lhead{MAIN PAPER - MA}
\rhead{2008}
\cfoot{\thepage}

% Section title format
\titleformat{\section}{\large\bfseries}{\thesection.}{0.5em}{}

% Controlled spacing between list items
\setlist[itemize]{itemsep=0.4em}
\setlist[enumerate]{itemsep=0.4em}

\begin{document}

\begin{center}
    \Large\textbf{MA : MATHEMATICS} \\
    \normalsize Duration: Three Hours \hfill Maximum Marks: 150
\end{center}

\vspace{0.5cm}

\section*{Read the following instructions carefully}

\begin{enumerate}[label=\arabic*.]
    \item This question paper contains 29 printed pages including pages for rough work. Please check all pages and report discrepancy, if any.

    \item Write your registration number, your name and name of the examination centre at the specified locations on the right half of the ORS.

    \item Using HB pencil, darken the appropriate bubble under each digit of your registration number and the letters corresponding to your paper code.

    \item All the questions in this question paper are of objective type.

    \item Questions must be answered on the Objective Response Sheet (ORS) by darkening the appropriate bubble (marked A, B, C, D) using HB pencil against the question number on the left-hand side of the ORS. \textbf{(Each question has only one correct answer)}. In case you wish to change an answer, erase the old answer completely. More than one answer bubbled against a question will be treated as a wrong answer.

    \item Questions 1 through 20 are 1-mark questions and questions 21 through 85 are 2-mark questions.

    \item Questions 73 through 75 is one set of common data questions, questions 74 and 75 has another pair of common data questions. The question pairs (76, 77), (78, 79), (80, 81), (82, 83) and (84, 85) are questions linked to each other. The answer to the second question of the above pairs will depend on the answer to the first question of the pair. If the first question in the linked pair is wrongly answered or not answered, then the answer to the second question in the pair will not be evaluated.

    \item Unattempted questions will result in zero mark and wrong answers will result in NEGATIVE marks. For all 1-mark questions, \textbf{\(-\frac{1}{3}\)} mark will be deducted for each wrong answer. For all 2-mark questions, \textbf{\(-\frac{2}{3}\)} mark will be deducted for each wrong answer.

    \item Calculator is allowed. Charts, graph sheets or tables are NOT allowed in the examination hall.

    \item Rough work may be done on the pages provided at the end of the question paper.
\end{enumerate}



\newpage 

\begin{center}
    \textbf{\underline{Notations and Definitions used in the paper}}
\end{center}

\vspace{0.3cm}

\begin{tabbing}
    $\phi$ \hspace{2cm} \= : The empty set \\
    $X \setminus Y$ \> : $\{x \in X : x \notin Y\}$ \\
    $\mathbb{N}$ \> : The set of all positive integers \\
    $\mathbb{Z}$ \> : The set of all integers \\
    $\mathbb{Z}_n$ \> : The set of all integers modulo $n$ \\
    $\mathbb{R}$ \> : The set of all real numbers \\
    $\mathbb{R}^n$ \> : $\{(x_1, \dots, x_n) : x_i \in \mathbb{R} \text{ for } 1 \leq i \leq n \}$ \\
    $\mathbb{C}$ \> : The set of all complex numbers \\
    $\mathbb{Z}[x_1, \dots, x_r]$ \> : The polynomial ring in the variables $x_1, \dots, x_r$ with coefficients in $\mathbb{Z}$ \\
    $f'$ \> : The derivative of the function $f$ \\
    $\mu$ \> : The Lebesgue measure on $\mathbb{R}$ \\
    $\| \cdot \|_p$ \> : The $p$-norm for $1 \leq p \leq \infty$ \\
    $\mathcal{C}([0,1])$ \> : The set of all real-valued continuous functions on $[0,1]$ \\
    $\mathcal{C}^1([0,1])$ \> : The set of all real-valued continuously differentiable functions on $[0,1]$ \\
    $\mathcal{C}^\infty([0,1])$ \> : The set of all real-valued infinitely differentiable functions on $[0,1]$ \\
    $x^T$ \> : The transpose of the vector $x$ \\
    $M^T$ \> : The transpose of the matrix $M$ \\
    $N(m, \sigma^2)$ \> : The normal distribution with mean $m$ and variance $\sigma^2$ \\
\end{tabbing}


\title{\textbf{MAIN PAPER - MA 2008}}
\date{}



\maketitle

\section*{Q.1 – Q.20 carry one mark each.}

\textbf{Q.1.} Consider the subspace $W = \{[a_{ij}] : a_{ij} = 0 \text{ if } i \text{ is even} \}$ of all $10 \times 10$ real matrices. Then the dimension of $W$ is  
\\[1ex] \hfill \textit{[GATE EE 2025]}
\newline
(A) 25 \hspace{1cm} (B) 50 \hspace{1cm} (C) 75 \hspace{1cm} (D) 100

\vspace{0.5cm}

\textbf{Q.2.} Let $S$ be the open unit disk and $f : S \to \mathbb{C}$ be a real-valued analytic function with $f(0) = 1$. Then the set $\{z \in S : f(z) = 1\}$ is  
\\[1ex] \hfill \textit{[GATE EE 2025]}
\newline
(A) empty \hspace{1cm} (B) nonempty finite \hspace{1cm} (C) countably infinite \hspace{1cm} (D) uncountable

\vspace{0.5cm}

\textbf{Q.3.} Let $E = \{(x, y) \in \mathbb{R}^2 : 0 \le x \le 1, 0 \le y \le x \}$. Then $\int\int_E (x+y) \, dx \, dy$ is equal to 
\\[1ex] \hfill \textit{[GATE EE 2025]}
\newline
(A) –1 \hspace{1cm} (B) 0 \hspace{1cm} (C) 12 \hspace{1cm} (D) 1

\vspace{0.5cm}

\textbf{Q.4.} For $(x, y) \in \mathbb{R}^2$, let  
\[
f(x, y) = 
\begin{cases}
\frac{2xy}{x^2 + y^2} & \text{if } (x, y) \ne (0, 0), \\
0 & \text{if } (x, y) = (0, 0)
\end{cases}
\]  \\[1ex] \hfill \textit{[GATE EE 2025]}

Then  \newline
(A) $f_x$ and $f_y$ exist at $(0,0)$, and $f$ is continuous at $(0,0)$  \newline
(B) $f_x$ and $f_y$ exist at $(0,0)$, and $f$ is discontinuous at $(0,0)$  \newline
(C) $f_x$ and $f_y$ do not exist at $(0,0)$, and $f$ is continuous at $(0,0)$  \newline
(D) $f_x$ and $f_y$ do not exist at $(0,0)$, and $f$ is discontinuous at $(0,0)$\newline

\vspace{0.5cm}

\textbf{Q.5.} Let $y$ be a solution of $y' = e^{-x^2}$ on $[0,1]$ which satisfies $y(0) = 0$.
\\[1ex] \textit{[GATE EE 2025]}

Then  \newline
(A) $y(x) > 0$ for $x > 0$  \newline
(B) $y(x) < 0$ for $x > 0$  \newline
(C) $y$ changes sign in $[0,1]$  \newline
(D) $y = 0$ for $x > 0$\newline

\vspace{0.5cm}

\textbf{Q.6.} For the equation $x(x - 1)y'' + \sin(x)y' + 2x(x - 1)y = 0$, consider the following statements:\newline

P: $x = 0$ is a regular singular point.  \newline
Q: $x = 1$ is a regular singular point.\newline
\\[1ex] \textit{[GATE EE 2025]}

Then  \newline
(A) both P and Q are true  \newline
(B) P is false but Q is true  \newline
(C) P is true but Q is false  \newline
(D) both P and Q are false\newline

\vspace{0.5cm}

\textbf{Q.7.} Let $G = \mathbb{R} \setminus \{0\}$ and $H = \{-1, 1\}$ be groups under multiplication. Then the map  \newline
\[
\varphi : G \to H \text{ defined by } \varphi(x) = \frac{x}{|x|}
\]  
is  
\\[1ex] \textit{[GATE EE 2025]}
\newline
(A) not a homomorphism  \newline
(B) a one-one homomorphism, which is not onto  \newline
(C) an onto homomorphism, which is not one-one  \newline
(D) an isomorphism\newline

\begin{flushleft}

\textbf{Q.8} The number of maximal ideals in $\mathbb{Z}_{12}$ is
\\[1ex] \textit{[GATE EE 2025]}
\newline
(A) 0 \hspace{1cm} (B) 1 \hspace{1cm} (C) 2 \hspace{1cm} (D) 3
\newline



\textbf{Q.9} For $1 \leq p \leq \infty$, let $\| \cdot \|_p$ denote the $p$-norm on $\mathbb{R}^2$. If $\| \cdot \|_p$ satisfies the parallelogram law, then $p$ is equal to
\\[1ex] \textit{[GATE EE 2025]}
\newline
(A) 1 \hspace{1cm} (B) 2 \hspace{1cm} (C) 3 \hspace{1cm} (D) $\infty$
\newline
\newline

\textbf{Q.10} Consider the initial value problem $\frac{dy}{dx} = f(x, y), \, y(x_0) = y_0$. The aim is to compute the value of $y_1 = y(x_1)$, where $x_1 = x_0 + h \, (h > 0)$. At $x = x_1$, if the value of $y_1$ is equated to the corresponding value of the straight line passing through $(x_0, y_0)$ and having the slope equal to the slope of the curve $y(x)$ at $x = x_0$, then the method is called
\\[1ex] \textit{[GATE EE 2025]}
\newline
(A) Euler's method \hspace{0.5cm} (B) Improved Euler's method \hspace{0.5cm} (C) Backward Euler's method \hspace{0.5cm} (D) Taylor series method of order 2



\textbf{Q.11} The solution of $xu_t + yu_x = 0$ is of the form
\\[1ex] \textit{[GATE EE 2025]}
\newline
(A) $f\left(\frac{y}{x}\right)$ \hspace{1cm} (B) $f(x+y)$ \hspace{1cm} (C) $f(x - y)$ \hspace{1cm} (D) $f(xy)$


\textbf{Q.12} If the partial differential equation $(x - 1)^2 u_{xx} - (y - 2)^2 u_{yy} + 2xu_x + 2yu_y + 2xyu = 0$ is parabolic in $S \subset \mathbb{R}^2$ but not in $\mathbb{R}^2 \setminus S$, then $S$ is
\\[1ex] \textit{[GATE EE 2025]}
\newline
(A) $\{(x, y) \in \mathbb{R}^2 : x = 1 \text{ or } y = 2 \}$ \hspace{0.5cm} 
(B) $\{(x, y) \in \mathbb{R}^2 : x = 1 \text{ and } y = 2 \}$ \newline
(C) $\{(x, y) \in \mathbb{R}^2 : x = 1 \}$ \hspace{1.5cm} 
(D) $\{(x, y) \in \mathbb{R}^2 : y = 2 \}$
\newline
\newline

\textbf{Q.13} Let $E$ be a connected subset of $\mathbb{R}$ with at least two elements. Then the number of elements in $E$ is
\\[1ex] \textit{[GATE EE 2025]}
\newline
(A) exactly two \hspace{1cm} (B) more than two but finite \hspace{1cm} (C) countably infinite \hspace{1cm} (D) uncountable
\newline
\newline

\textbf{Q.14} Let $X$ be a non-empty set. Let $\mathcal{T}_1$ and $\mathcal{T}_2$ be two topologies on $X$ such that $\mathcal{T}_1$ is strictly contained in $\mathcal{T}_2$. If $f : (X, \mathcal{T}_3) \rightarrow (X, \mathcal{T}_3)$ is the identity map, then
\\[1ex] \textit{[GATE EE 2025]}
\newline
(A) both $f$ and $f^{-1}$ are continuous \hspace{1cm}
(B) both $f$ and $f^{-1}$ are not continuous \newline
(C) $f$ is continuous but $f^{-1}$ is not continuous \hspace{0.5cm}
(D) $f$ is not continuous but $f^{-1}$ is continuous
\newline
\newline

\textbf{Q.15} Let $X_1, X_2, \dots, X_{10}$ be a random sample from $N(80, 3^2)$ distribution. Define
\[
S = \sum_{i=1}^{10} U_i, \quad T = \sum_{i=1}^{10} \left(U_i - \frac{S}{10} \right)^2
\]
where $U_i = \frac{X_i - 80}{3}, \, i = 1, 2, \dots, 10$. Then the value of $E(ST)$ is equal to
\\[1ex] \textit{[GATE EE 2025]}
\newline
(A) 0 \hspace{1cm} (B) 1 \hspace{1cm} (C) 10 \hspace{1cm} (D) $\frac{80}{3}$
\newline
\end{flushleft}

\textbf{Q.16} Two (indistinguishable) fair coins are tossed simultaneously. Given that ONE of them lands up head, the probability of the OTHER to land up tail is equal to
\\[1ex] \textit{[GATE EE 2025]}
\newline
(A) $\dfrac{1}{3}$ \hspace{10mm}
(B) $\dfrac{1}{2}$ \hspace{10mm}
(C) $\dfrac{2}{3}$ \hspace{10mm}
(D) $\dfrac{3}{4}$ \newline

\textbf{Q.17} Let $c_{ij} \geq 2$ be the cost of the $(i,j)^{\text{th}}$ cell of an assignment problem. If a new cost matrix is generated by the elements $c'_{ij} = \dfrac{1}{2} c_{ij} + 1$, then
\\[1ex] \textit{[GATE EE 2025]}
\newline
(A) optimal assignment plan remains unchanged and cost of assignment decreases \newline
(B) optimal assignment plan changes and cost of assignment decreases \newline
(C) optimal assignment plan remains unchanged and cost of assignment increases \newline
(D) optimal assignment plan changes and cost of assignment increases
\newline
\newline
\textbf{Q.18} Let a primal linear programming problem admit an optimal solution. Then the corresponding dual problem 
\\[1ex] \textit{[GATE EE 2025]}
\newline
(A) does not have a feasible solution \newline
(B) has a feasible solution but does not have any optimal solution \newline
(C) does not have a convex feasible region \newline
(D) has an optimal solution
\newline
\newline
\textbf{Q.19} In any system of particles, suppose we do not assume that the internal forces come in pairs. Then the fact that the sum of internal forces is zero follows from
\\[1ex] \textit{[GATE EE 2025]}
\newline
(A) Newton's second law \newline
(B) conservation of angular momentum \newline
(C) conservation of energy \newline
(D) principle of virtual displacement
\newline
\newline
\textbf{Q.20} Let $q_1, q_2, \dots, q_n$ be the generalized coordinates and $\dot{q}_1, \dot{q}_2, \dots, \dot{q}_n$ be the generalized velocities in a conservative force field. If under a transformation $\varphi$, the new coordinate system has the generalized coordinates $Q_1, Q_2, \dots, Q_n$ and velocities $\dot{Q}_1, \dot{Q}_2, \dots, \dot{Q}_n$, then the equation \newline
\[
\frac{\partial L}{\partial q_i} - \frac{d}{dt} \left( \frac{\partial L}{\partial \dot{q}_i} \right)
\]
takes the form
\\[1ex] \textit{[GATE EE 2025]}
\newline
(A) $\dfrac{\partial L}{\partial Q_i} - \dfrac{d}{dt} \left( \dfrac{\partial L}{\partial \dot{Q}_i} \right)$ \hspace{10mm}
(B) $\dfrac{\partial L}{\partial Q_i} - \dfrac{\partial}{\partial t} \left( \dfrac{\partial L}{\partial \dot{Q}_i} \right)$ \newline
(C) $\dfrac{\partial L}{\partial \dot{Q}_i} - \dfrac{d}{dt} \left( \dfrac{\partial L}{\partial Q_i} \right)$ \hspace{10mm}
(D) $\dfrac{\partial L}{\partial \dot{Q}_i} - \dfrac{\partial}{\partial t} \left( \dfrac{\partial L}{\partial Q_i} \right)$
\newline
\newline
\textbf{Q.21 to Q.75 carry two marks each.}

\textbf{Q.21} Let $T: \mathbb{R}^4 \rightarrow \mathbb{R}^4$ be the linear map satisfying \newline
\[
T(e_1) = e_2, \quad T(e_2) = e_3, \quad T(e_3) = 0, \quad T(e_4) = e_1,
\]
where $\{ e_1, e_2, e_3, e_4 \}$ is the standard basis of $\mathbb{R}^4$. Then 
\\[1ex] \textit{[GATE EE 2025]}
\newline
(A) $T$ is idempotent \hspace{10mm}
(B) $T$ is invertible \hspace{10mm}
(C) Rank $T = 3$ \hspace{10mm}
(D) $T$ is nilpotent
\newline
\newline

\noindent\textbf{Q.22}  
Let  
\[
M = \begin{pmatrix}
1 & 1 & 2 \\
0 & 1 & 1 \\
0 & 1 & 1
\end{pmatrix}, \quad
V = \{ Mx : x \in \mathbb{R}^3\}.
\]
Then an orthonormal basis for \(V\) is  
\\[1ex] \textit{[GATE EE 2025]}

\begin{enumerate}[label=(\Alph*)]

\item \(\left\{
\begin{pmatrix}1\\0\\0\end{pmatrix},
\begin{pmatrix}0\\\frac{2}{\sqrt{5}}\\\frac{1}{\sqrt{5}}\end{pmatrix},
\begin{pmatrix}\frac{2}{\sqrt{6}}\\\frac{1}{\sqrt{6}}\\\frac{1}{\sqrt{6}}\end{pmatrix}
\right\}\)
\item \(\left\{
\begin{pmatrix}1\\0\\0\end{pmatrix},
\begin{pmatrix}0\\\frac{1}{\sqrt{2}}\\\frac{1}{\sqrt{2}}\end{pmatrix}
\right\}\)
\item \(\left\{
\begin{pmatrix}1\\0\\0\end{pmatrix},
\begin{pmatrix}\frac{1}{\sqrt{3}}\\\frac{1}{\sqrt{3}}\\\frac{1}{\sqrt{3}}\end{pmatrix},
\begin{pmatrix}\frac{2}{\sqrt{6}}\\\frac{1}{\sqrt{6}}\\\frac{1}{\sqrt{6}}\end{pmatrix}
\right\}\)
\item \(\left\{
\begin{pmatrix}1\\0\\0\end{pmatrix},
\begin{pmatrix}0\\0\\1\end{pmatrix}
\right\}\)
\end{enumerate}

\vspace{0.5cm}

\noindent\textbf{Q.23}  
For any \(n \in \mathbb{N}\), let \(P_n\) be the vector space of all real polynomials of degree $< n$. Define

\[
T\bigl(p\bigr)(x) = p(x) - \int_{0}^{x} p(t)\,dt.
\]
Then \(\dim(\ker T)\) is  
\\[1ex] \textit{[GATE EE 2025]}

\begin{enumerate}[label=(\Alph*)]

\item 0 \item 1 \item \(n\) \item \(n+1\)
\end{enumerate}

\vspace{0.5cm}

\noindent\textbf{Q.24}  
Let  
\[
M = \begin{pmatrix}
1 & 0 & 0 \\
0 & \cos\theta & -\sin\theta \\
0 & \sin\theta & \cos\theta
\end{pmatrix}, \quad 0 < \theta < \frac{\pi}{2}, \quad
V = \{ u \in \mathbb{R}^3 : Mu = u \}.
\]
Then \(\dim(V)\) is  
\\[1ex] \textit{[GATE EE 2025]}

\begin{enumerate}[label=(\Alph*)]

\item 0 \item 1 \item 2 \item 3
\end{enumerate}

\vspace{0.5cm}

\noindent\textbf{Q.25}  
The number of linearly independent eigenvectors of the matrix  
\[
\begin{pmatrix}
2 & 2 & 0 & 0 \\
2 & 1 & 0 & 0 \\
0 & 0 & 3 & 0 \\
0 & 0 & 1 & 4
\end{pmatrix}
\]
is  
\\[1ex] \textit{[GATE EE 2025]}

\begin{enumerate}[label=(\Alph*)]

\item 1 \item 2 \item 3 \item 4
\end{enumerate}

\vspace{0.5cm}

\noindent\textbf{Q.26}  
Let \(f\) be a bilinear transformation mapping \(-1\mapsto1\), \(i\mapsto 0\), and \(-i\mapsto 0\). Then \(f(i)\) is equal to  
\\[1ex] \textit{[GATE EE 2025]}

\begin{enumerate}[label=(\Alph*)]

\item \(-2\) \item \(-1\) \item \(i\) \item \(-i\)
\end{enumerate}

\begin{flushleft}
\textbf{Q.27} Which one of the following does NOT hold for all continuous functions $f:[-\pi,\pi] \rightarrow \mathbb{C}$?
\\[1ex] \textit{[GATE EE 2025]}
\newline
(A) If $f(-t) = f(t)$ for each $t \in [-\pi, \pi]$, then $\int_{-\pi}^{\pi} f(t)\,dt = 2\int_{0}^{\pi} f(t)\,dt$ \newline
(B) If $f(-t) = -f(t)$ for each $t \in [-\pi, \pi]$, then $\int_{-\pi}^{\pi} f(t)\,dt = 0$ \newline
(C) $\int_{-\pi}^{\pi} f(-t)\,dt = -\int_{-\pi}^{\pi} f(t)\,dt$ \newline
(D) There is an $\alpha$ with $-\pi < \alpha < \pi$ such that $\int_{-\pi}^{\pi} f(t)\,dt = 2\pi f(\alpha)$ \newline

\vspace{5pt}
\textbf{Q.28} Let $S$ be the positively oriented circle given by $|z - 3| = 2$. Then the value of $\int_{S} \frac{dz}{z^2 + 4}$ is
\\[1ex] \textit{[GATE EE 2025]}
\newline
(A) $-\frac{\pi}{2}$ \hspace{12pt}
(B) $\frac{\pi}{2}$ \hspace{12pt}
(C) $-\frac{i\pi}{2}$ \hspace{12pt}
(D) $\frac{i\pi}{2}$ \newline

\vspace{5pt}
\textbf{Q.29} Let $T$ be the closed unit disk and $\partial T$ be the unit circle. Then which one of the following holds for every analytic function $f : T \rightarrow \mathbb{C}$?
\\[1ex] \textit{[GATE EE 2025]}
\newline
(A) $|f|$ attains its minimum and its maximum on $\partial T$ \newline
(B) $|f|$ attains its minimum on $\partial T$ but need not attain its maximum on $\partial T$ \newline
(C) $|f|$ attains its maximum on $\partial T$ but need not attain its minimum on $\partial T$ \newline
(D) $|f|$ need not attain its maximum on $\partial T$ and also it need not attain its minimum on $\partial T$ \newline

\vspace{5pt}
\textbf{Q.30} Let $S$ be the disk $|z| < 3$ in the complex plane and let $f : S \rightarrow \mathbb{C}$ be an analytic function such that
\[
f\left(\frac{\sqrt{2}}{n}\right) = \frac{2}{n^2 \left(1 + \frac{\sqrt{2}}{n}\right)} \quad \text{for each natural number } n.
\]

Then $f(\sqrt{2})$ is equal to
\\[1ex] \textit{[GATE EE 2025]}
\newline
(A) $3 - 2\sqrt{2}$ \hspace{12pt}
(B) $3 + 2\sqrt{2}$ \hspace{12pt}
(C) $2 - 3\sqrt{2}$ \hspace{12pt}
(D) $2 + 3\sqrt{2}$ \newline

\vspace{5pt}
\textbf{Q.31} Which one of the following statements holds? 
\\[1ex] \textit{[GATE EE 2025]}
\newline
(A) The series $\sum_{n=0}^{\infty} x^n$ converges for each $x \in [-1, 1]$ \newline
(B) The series $\sum_{n=0}^{\infty} x^n$ converges uniformly in $(-1, 1)$ \newline
(C) The series $\sum_{n=1}^{\infty} \frac{x^n}{n}$ converges for each $x \in [-1, 1]$ \newline
(D) The series $\sum_{n=1}^{\infty} \frac{x^n}{n^2}$ converges uniformly in $(-1, 1)$ \newline
\end{flushleft}

\begin{flushleft}
\textbf{Q.32} For $x \in [-\pi, \pi]$, let \newline
$f(x) = (x + \pi)(\pi - x) \quad \text{and} \quad g(x) = \begin{cases}
\cos\left(\frac{1}{x}\right) & \text{if } x \neq 0, \\
0 & \text{if } x = 0
\end{cases}$ \newline

Consider the statements \newline
P: The Fourier series of $f$ converges uniformly to $f$ on $[-\pi, \pi]$. \newline
Q: The Fourier series of $g$ converges uniformly to $g$ on $[-\pi, \pi]$. \newline
Then
\\[1ex] \textit{[GATE EE 2025]}
\newline
(A) P and Q are true \hspace{2em} (B) P is true but Q is false \newline
(C) P is false but Q is true \hspace{2em} (D) Both P and Q are false \newline
\end{flushleft}

\begin{flushleft}
\textbf{Q.33} Let $W = \left\{(x, y, z) \in \mathbb{R}^3 : 1 \leq x^2 + y^2 + z^2 \leq 4 \right\}$ and $F : W \rightarrow \mathbb{R}^3$ be defined by \newline
\hspace*{2em} $F(x, y, z) = \frac{(x, y, z)}{x^2 + y^2 + z^2} \quad \text{for } (x, y, z) \in W$. \newline
If $\partial W$ denotes the boundary of $W$ oriented by the outward normal $\vec{n}$ to $W$, then $\iint_{\partial W} \vec{F} \cdot \vec{n} \, dS$ is equal to
\\[1ex] \textit{[GATE EE 2025]}
\newline
(A) 0 \hspace{3em} (B) $4\pi$ \hspace{3em} (C) $8\pi$ \hspace{3em} (D) $12\pi$ \newline
\end{flushleft}

\begin{flushleft}
\textbf{Q.34} For each $n \in \mathbb{N}$, let $f_n : [0,1] \rightarrow \mathbb{R}$ be a measurable function such that $|f_n(t)| \leq \frac{1}{\sqrt{t}}$ for all $t \in (0,1]$. Let $f : [0,1] \rightarrow \mathbb{R}$ be defined by $f(t) = 1$ if $t$ is irrational and $f(t) = -1$ if $t$ is rational. Assume that $f_n(t) \rightarrow f(t)$ as $n \rightarrow \infty$ for all $t \in [0,1]$. Then
\\[1ex] \textit{[GATE EE 2025]}
\newline
(A) $f$ is not measurable \newline
(B) $f$ is measurable and $\int_{[0,1]} f_n \, d\mu \to 1$ as $n \to \infty$ \newline
(C) $f$ is measurable and $\int_{[0,1]} f_n \, d\mu \to 0$ as $n \to \infty$ \newline
(D) $f$ is measurable and $\int_{[0,1]} f_n \, d\mu \to -1$ as $n \to \infty$ \newline
\end{flushleft}

\begin{flushleft}
\textbf{Q.35} Let $y_1$ and $y_2$ be two linearly independent solutions of \newline
\hspace*{2em} $y'' + (\sin x) y = 0$, \quad $0 \leq x \leq 1$. \newline
Let $g(x) = W(y_1, y_2)(x)$ be the Wronskian of $y_1$ and $y_2$. Then 
\\[1ex] \textit{[GATE EE 2025]}
\newline
(A) $g' > 0$ on $[0,1]$ \hspace{2em} (B) $g' < 0$ on $[0,1]$ \newline
(C) $g'$ vanishes at only one point of $[0,1]$ \hspace{2em} (D) $g'$ vanishes at all points of $[0,1]$ \newline
\end{flushleft}

\begin{flushleft}
\textbf{Q.36} One particular solution of $y''' - y'' + y' + y = e^x$ is a constant multiple of
\\[1ex] \textit{[GATE EE 2025]}
\newline
(A) $x e^x$ \hspace{3em} (B) $x e^{-x}$ \hspace{3em} (C) $x^2 e^x$ \hspace{3em} (D) $x^2 e^{-x}$ \newline
\end{flushleft}

\begin{flushleft}
\textbf{Q.37} Let $a, b \in \mathbb{R}$. Let $y = (y_1, y_2)^T$ be a solution of the system of equations \newline
\hspace*{2em} $y_1' = y_2$, \quad $y_2' = a y_1 + b y_2$. \newline
Every solution $y(x) \to 0$ as $x \to \infty$ if
\\[1ex] \textit{[GATE EE 2025]}
\newline
(A) $a < 0, b < 0$ \hspace{2em} (B) $a < 0, b > 0$ \newline
(C) $a > 0, b < 0$ \hspace{2em} (D) $a > 0, b > 0$ \newline
\end{flushleft}

\begin{flushleft}
\textbf{Q.38} Let $G$ be a group of order $45$. Let $H$ be a 3-Sylow subgroup of $G$ and $K$ be a 5-Sylow subgroup of $G$. Then 
\\[1ex] \textit{[GATE EE 2025]}
\newline
(A) both $H$ and $K$ are normal in $G$ \newline
(B) $H$ is normal in $G$ but $K$ is not normal in $G$ \newline
(C) $H$ is not normal in $G$ but $K$ is normal in $G$ \newline
(D) both $H$ and $K$ are not normal in $G$ \newline
\end{flushleft}

\begin{flushleft}
\textbf{Q.39} The ring $\mathbb{Z}[\sqrt{-11}]$ is
\\[1ex] \textit{[GATE EE 2025]}
\newline
(A) a Euclidean Domain \newline
(B) a Principal Ideal Domain, but not a Euclidean Domain \newline
(C) a Unique Factorization Domain, but not a Principal Ideal Domain \newline
(D) not a Unique Factorization Domain \newline
\end{flushleft}

\begin{flushleft}
\textbf{Q.40} Let $R$ be a Principal Ideal Domain and $a, b$ any two non-unit elements of $R$. Then the ideal generated by $a$ and $b$ is also generated by 
\\[1ex] \textit{[GATE EE 2025]}
\newline
(A) $a + b$ \hspace{3em} (B) $ab$ \hspace{3em} (C) $\gcd(a, b)$ \hspace{3em} (D) $\mathrm{lcm}(a, b)$ \newline
\end{flushleft}

\begin{flushleft}
\textbf{Q.41} Consider the action of $S_4$, the symmetric group of order 4, on $\mathbb{Z}[x_{1}, x_{2}, x_{3}, x_{4}]$ given by \newline
\hspace*{2em} $\sigma \cdot p(x_1, x_2, x_3, x_4) = p(x_{\sigma(1)}, x_{\sigma(2)}, x_{\sigma(3)}, x_{\sigma(4)})$ for $\sigma \in S_4$. \newline
Let $H \leq S_4$ denote the cyclic subgroup generated by $(1\ 4\ 2\ 3)$. Then the cardinality of the orbit \newline
\hspace*{2em} $\mathcal{O}_H(x_1 x_3 + x_2 x_4)$ of $H$ on the polynomial $x_1 x_3 + x_2 x_4$ is 
\\[1ex] \textit{[GATE EE 2025]}
\newline
(A) 1 \hspace{3em} (B) 2 \hspace{3em} (C) 3 \hspace{3em} (D) 4 \newline
\end{flushleft}

\begin{flushleft}
\textbf{Q.42} Let $f : \ell^2 \to \mathbb{R}$ be defined by $f(x_1, x_2, \dots) = \sum \limits_{k=1}^{\infty} \frac{x_k}{2^k}$ for $(x_1, x_2, \dots) \in \ell^2$. Then $\|f\|$ is equal to
\\[1ex] \textit{[GATE EE 2025]}
\newline
(A) $\frac{1}{2}$ \hspace{3em} (B) $1$ \hspace{3em} (C) $2$ \hspace{3em} (D) $\frac{1}{\sqrt{2} - 1}$ \newline
\end{flushleft}

\begin{flushleft}
\textbf{Q.43} Consider $\mathbb{R}^3$ with norm $\|\cdot\|$, and the linear transformation $T : \mathbb{R}^3 \to \mathbb{R}^3$ defined by the $3 \times 3$ matrix \newline
\[
\begin{bmatrix}
1 & 1 & 3 \\
2 & 2 & 2 \\
1 & 3 & -2
\end{bmatrix}
\]
Then the operator norm $\|T\|$ of $T$ is equal to
\\[1ex] \textit{[GATE EE 2025]}
\newline
(A) 6 \hspace{3em} (B) 7 \hspace{3em} (C) 8 \hspace{3em} (D) $\sqrt{42}$ \newline
\end{flushleft}

\begin{flushleft}
\textbf{Q.44} Consider $\mathbb{R}^2$ with norm $\|\cdot\|_{\infty}$, and let $Y = \{(y_1, y_2) \in \mathbb{R}^2 : y_1 + y_2 = 0\}$. If $g: Y \to \mathbb{R}$ is defined by $g(y_1, y_2) = y_1$, for $(y_1, y_2) \in Y$, then 
\\[1ex] \textit{[GATE EE 2025]}
\newline
(A) $g$ has no Hahn-Banach extension to $\mathbb{R}^2$ \newline
(B) $g$ has a unique Hahn-Banach extension to $\mathbb{R}^2$ \newline
(C) Every linear functional $f : \mathbb{R}^2 \to \mathbb{R}$ satisfying $f(-1, 1) = 1$ is a Hahn-Banach extension of $g$ to $\mathbb{R}^2$ \newline
(D) The functionals $f_1, f_2 : \mathbb{R}^2 \to \mathbb{R}$ given by $f_1(x_1, x_2) = x_2$ and $f_2(x_1, x_2) = -x_1$ are both Hahn-Banach extensions of $g$ to $\mathbb{R}^2$ \newline
\end{flushleft}

\begin{flushleft}
\textbf{Q.45} Let $X$ be a Banach space and $Y$ be a normed linear space. Consider a sequence $(F_n)$ of bounded linear maps from $X$ to $Y$ such that for each fixed $x \in X$, the sequence $(F_n(x))$ is bounded in $Y$. Then 
\\[1ex] \textit{[GATE EE 2025]}
\newline
(A) For each fixed $x \in X$, the sequence $(F_n(x))$ is convergent in $Y$ \newline
(B) For each fixed $n \in \mathbb{N}$, the set $\{F_n(x): x \in X\}$ is bounded in $Y$ \newline
(C) The sequence $(\|F_n\|)$ is bounded in $\mathbb{R}$ \newline
(D) The sequence $(F_n)$ is uniformly bounded on $X$ \newline
\end{flushleft}

\begin{flushleft}
\textbf{Q.46} Let $H = L^2([0, \pi])$ with the usual inner product. For $n \in \mathbb{N}$, let \newline
\[
u_n(t) = \frac{\sqrt{2}}{\sqrt{\pi}} \sin(n t),\ t \in [0, \pi], \quad \text{and} \quad E = \{u_n : n \in \mathbb{N}\}
\]
Then 
\\[1ex] \textit{[GATE EE 2025]}
\newline
(A) $E$ is not a linearly independent subset of $H$ \newline
(B) $E$ is a linearly independent subset of $H$, but is not an orthonormal subset of $H$ \newline
(C) $E$ is an orthonormal subset of $H$, but is not an orthonormal basis for $H$ \newline
(D) $E$ is an orthonormal basis for $H$ \newline
\end{flushleft}

\begin{flushleft}
\textbf{Q.47} Let $X = \mathbb{R}$ and let $\mathfrak{T} = \{U \subseteq X : X - U \text{ is finite}\} \cup \{\emptyset, X\}$. The sequence \newline
\[
1, \frac{1}{2}, \frac{1}{3}, \dots, \frac{1}{n}, \dots
\]
in $(X, \mathfrak{T})$
\\[1ex] \textit{[GATE EE 2025]}
\newline
(A) converges to 0 and not to any other point of $X$ \newline
(B) does not converge to 0 \newline
(C) converges to each point of $X$ \newline
(D) is not convergent in $X$ \newline
\end{flushleft}

\begin{flushleft}
\textbf{Q.48} Let $E = \{(x, y) \in \mathbb{R}^2 : |x| \leq 1, |y| \leq 1\}$. Define $f : E \to \mathbb{R}$ by $f(x, y) = \frac{x + y}{1 + x^2 + y^2}$. \newline
Then the range of $f$ is a
\\[1ex] \textit{[GATE EE 2025]}
\newline
(A) connected open set \hspace{3em} (B) connected closed set \newline
(C) bounded open set \hspace{3em} (D) closed and unbounded set \newline
\end{flushleft}

\begin{flushleft}
\textbf{Q.49} Let $X = \{1, 2, 3\}$ and $\mathfrak{T} = \{\emptyset, \{1\}, \{2\}, \{1, 2\}, \{2, 3\}, \{1, 2, 3\}\}$. The topological space $(X, \mathfrak{T})$ is said to have the property $P$ if for any two proper disjoint closed subsets $Y$ and $Z$ of $X$, there exist disjoint open sets $U, V$ such that $Y \subseteq U$ and $Z \subseteq V$. Then the topological space $(X, \mathfrak{T})$ 
\\[1ex] \textit{[GATE EE 2025]}
\newline
(A) is $T_1$ and satisfies $P$ \newline
(B) is $T_1$ and does not satisfy $P$ \newline
(C) is not $T_1$ and satisfies $P$ \newline
(D) is not $T_1$ and does not satisfy $P$ \newline
\end{flushleft}

\begin{flushleft}
\textbf{Q.50} Which one of the following subsets of $\mathbb{R}$ (with the usual metric) is NOT complete? 
\\[1ex] \textit{[GATE EE 2025]}
\newline
(A) $[1,2] \cup [3,4]$ \hspace{2em}
(B) $[0, \infty)$ \hspace{2em}
(C) $[0,1]$ \hspace{2em}
(D) $\{0\} \cup \left\{\frac{1}{n} : n \in \mathbb{N}\right\}$ \newline
\end{flushleft}

\begin{flushleft}
\textbf{Q.51} Consider the function
\[
f(x) =
\begin{cases}
k(x - \lfloor x \rfloor), & 0 \le x < 2 \\
0, & \text{otherwise}
\end{cases}
\]
where $\lfloor x \rfloor$ is the integral part of $x$. The value of $k$ for which the above function is a probability density function of some random variable is 
\\[1ex] \textit{[GATE EE 2025]}
\newline
(A) $\dfrac{1}{4}$ \hspace{2em}
(B) $\dfrac{1}{2}$ \hspace{2em}
(C) $1$ \hspace{2em}
(D) $2$ \newline
\end{flushleft}

\begin{flushleft}
\textbf{Q.52} For two random variables $X$ and $Y$, the regression lines are given by $Y = 5X - 15$ and $Y = 10X - 35$. Then the regression coefficient of $X$ on $Y$ is 
\\[1ex] \textit{[GATE EE 2025]}
\newline
(A) $0.1$ \hspace{2em}
(B) $0.2$ \hspace{2em}
(C) $5$ \hspace{2em}
(D) $10$ \newline
\end{flushleft}

\begin{flushleft}
\textbf{Q.53} In an examination there are 80 questions each having four choices. Exactly one of these four choices is correct and the other three are wrong. A student is awarded 1 mark for each correct answer, and $-0.25$ for each wrong answer. If a student ticks the answer of each question randomly, then the expected value of his/her total marks in the examination is 
\\[1ex] \textit{[GATE EE 2025]}
\newline
(A) $-15$ \hspace{2em}
(B) $0$ \hspace{2em}
(C) $5$ \hspace{2em}
(D) $20$ \newline
\end{flushleft}

\begin{flushleft}
\textbf{Q.54} Let $X_1, X_2, \dots, X_n$ be a random sample from uniform distribution on $[0, \theta]$. Then the maximum likelihood estimator (MLE) of $\theta$ based on the above random sample is 
\\[1ex] \textit{[GATE EE 2025]}
\newline
(A) $\dfrac{2}{n} \sum_{i=1}^{n} X_i$ \hspace{2em}
(B) $\dfrac{1}{n} \sum_{i=1}^{n} X_i$ \newline
(C) $\min\{X_1, X_2, \dots, X_n\}$ \hspace{2em}
(D) $\max\{X_1, X_2, \dots, X_n\}$ \newline
\end{flushleft}

\begin{flushleft}
\textbf{Q.55} The cost matrix of a transportation problem is given by
\[
\begin{bmatrix}
1 & 2 & 3 & 4 \\
4 & 3 & 2 & 1 \\
0 & 2 & 2 & 1
\end{bmatrix}
\]
The following are the values of variables in a feasible solution:
\[
x_{12} = 6, \quad x_{23} = 2, \quad x_{24} = 6, \quad x_{31} = 4, \quad x_{33} = 6
\]
Then which of the following is correct? \\
 \textit{[GATE EE 2025]}

(A) The solution is degenerate and basic \\
(B) The solution is non-degenerate and basic \\
(C) The solution is degenerate and non-basic \\
(D) The solution is non-degenerate and non-basic
\end{flushleft}

\begin{flushleft}
\textbf{Q.56} The maximum value of $z = 3x_1 - x_2$ subject to $2x_1 - x_2 \leq 1, \, x_1 \leq 3$ and $x_1, x_2 \geq 0$ is 
\\[1ex] \textit{[GATE EE 2025]}


(A) 0 \hspace{2em} (B) 4 \hspace{2em} (C) 6 \hspace{2em} (D) 9
\end{flushleft}

\begin{flushleft}
\textbf{Q.57} Consider the problem of maximizing $z = 2x_1 + 3x_2 - 4x_3 + x_4$ subject to
\[
\begin{aligned}
x_1 + x_2 + x_3 &= 2, \\
x_1 - x_2 + x_5 &= 2, \\
2x_1 + 3x_2 + 2x_5 - x_4 &= 0, \\
x_1, x_2, x_3, x_4 &\geq 0.
\end{aligned}
\]
Then 
\\[1ex] \textit{[GATE EE 2025]}


(A) (1,0,1,4) is a basic feasible solution but (2,0,0,4) is not \\
(B) (1,0,1,4) is not a basic feasible solution but (2,0,0,4) is \\
(C) neither (1,0,1,4) nor (2,0,0,4) is a basic feasible solution \\
(D) both of (1,0,1,4) and (2,0,0,4) are basic feasible solutions
\end{flushleft}

\begin{flushleft}
\textbf{Q.58} In the closed system of a simple harmonic motion of a pendulum, let $H$ denote the Hamiltonian and $E$ be the total energy. Then
\\[1ex] \textit{[GATE EE 2025]}

(A) $H$ is a constant and $H = E$ \\
(B) $H$ is a constant but $H \neq E$ \\
(C) $H$ is not constant but $H = E$ \\
(D) $H$ is not constant and $H \neq E$
\end{flushleft}

\begin{flushleft}
\textbf{Q.59} The possible values of $\alpha$ for which the variational problem
\[
J[y(x)] = \int_0^1 \left( 3y^2 + 2x y^2 \right) dx, \quad y(\alpha) = 1
\]
has extremals are 
\\[1ex] \textit{[GATE EE 2025]}

(A) $-1, 0$ \hspace{2em} (B) $0, 1$ \hspace{2em} (C) $-1, 1$ \hspace{2em} (D) $-1, 0, 1$
\end{flushleft}

\begin{flushleft}
\textbf{Q.60} The functional $\displaystyle \int_0^1 (y'^2 + x y^4) dx$, given $y(1) = 1$, achieves its 
\\[1ex] \textit{[GATE EE 2025]}

(A) weak maximum on all its extremals \\
(B) weak minimum on all its extremals \\
(C) weak maximum on some, but not on all of its extremals \\
(D) weak minimum on some, but not on all of its extremals
\end{flushleft}

\begin{flushleft}
\textbf{Q.61} The integral equation
\[
x(t) = \sin t + \lambda \int_0^t \left( s^2 + e^{s+t} \right) x(s) \, ds, \quad 0 \leq t \leq 1, \, \lambda \in \mathbb{R}, \, \lambda \neq 0
\]
has a solution for
\\[1ex] \textit{[GATE EE 2025]}

(A) all non-zero values of $\lambda$ \\
(B) no value of $\lambda$ \\
(C) only countably many positive values of $\lambda$ \\
(D) only countably many negative values of $\lambda$
\end{flushleft}

\begin{flushleft}
\textbf{Q.62} The integral equation 
\[
x(t) - \int_0^t \cos t \sec s \, x(s) \, ds = \sinh t, \quad 0 \leq t \leq 1
\]
has
\\[1ex] \textit{[GATE EE 2025]}

(A) no solution \\
(B) a unique solution \\
(C) more than one but finitely many solutions \\
(D) infinitely many solutions
\end{flushleft}

\begin{flushleft}
\textbf{Q.63} If $y_{i+1} = y_i + h \, \varphi(f, x_i, y_i, h)$, $i = 1, 2, \ldots$, where
\[
\varphi(f, x, y, h) = a f(x, y) + b f(x + h, y + h f(x, y))
\]
is a second order accurate scheme to solve the initial value problem $\dfrac{dy}{dx} = f(x, y), y(x_0) = y_0$, then $a$ and $b$, respectively, are 
\\[1ex] \textit{[GATE EE 2025]}

(A) $\dfrac{h}{2}, \dfrac{h}{2}$ \hspace{2em}
(B) 1, $-1$ \hspace{2em}
(C) $\dfrac{1}{2}, \dfrac{1}{2}$ \hspace{2em}
(D) $h$, $-h$
\end{flushleft}

\begin{flushleft}
\textbf{Q.64} If a quadrature formula
\[
\frac{3}{2} \left[ \frac{1}{3} f(0) + K f\left( \frac{1}{3} \right) + \frac{1}{3} f(1) \right]
\]
that approximates $\int_0^1 f(x) \, dx$, is found to be exact for quadratic polynomials, then the value of $K$ is 
\\[1ex] \textit{[GATE EE 2025]}

(A) 2 \hspace{2em} (B) 1 \hspace{2em} (C) 0 \hspace{2em} (D) $-1$
\end{flushleft}

\begin{flushleft}
\textbf{Q.65} If 
\[
\begin{bmatrix}
1 & 4 & 3 \\
2 & 7 & 9 \\
5 & 4 & 9
\end{bmatrix}
\begin{bmatrix}
l_{11} & 0 & 0 \\
l_{21} & l_{22} & 0 \\
l_{31} & l_{32} & l_{33}
\end{bmatrix}
= 
\begin{bmatrix}
h_{11} & 0 & 0 \\
0 & h_{22} & 0 \\
0 & 0 & h_{33}
\end{bmatrix}
\begin{bmatrix}
u_{11} & u_{12} & u_{13} \\
0 & u_{22} & u_{23} \\
0 & 0 & u_{33}
\end{bmatrix}
\]
then the value of $a$ is
\\[1ex] \textit{[GATE EE 2025]}

(A) $-2$ \hspace{2em} (B) $-1$ \hspace{2em} (C) 1 \hspace{2em} (D) 2
\end{flushleft}

\begin{flushleft}
\textbf{Q.66} Using the least squares method, if a curve $y = ax^2 + bx + c$ is fitted to the collinear data points $(-1, -3)$, $(1, 3)$, $(3, 5)$ and $(7, 13)$, then the triplet $(a, b, c)$ is equal to
\\[1ex] \textit{[GATE EE 2025]}

(A) $(-1, 2, 0)$ \hspace{2em} (B) $(0, 2, -1)$ \hspace{2em} (C) $(2, -1, 0)$ \hspace{2em} (D) $(0, -1, 1)$
\end{flushleft}

\begin{flushleft}
\textbf{Q.67} A quadratic polynomial $p(x)$ is constructed by interpolating the data points $(0,1)$, $(1,e)$ and $(2,e^2)$. If $\sqrt{e}$ is approximated by using $p(x)$, then its approximate value is 
\\[1ex] \textit{[GATE EE 2025]}


(A) $\dfrac{1}{8} \left( 3 + 6e - e^2 \right)$ \\
(B) $\dfrac{1}{8} \left( 3 - 6e + 2e^2 \right)$ \\
(C) $\dfrac{1}{8} \left( 3 - 6e - e^2 \right)$ \\
(D) $\dfrac{1}{8} \left( 3 + 6e - 2e^2 \right)$
\end{flushleft}

\begin{flushleft}
\textbf{Q.68} The characteristic curve of $2y u_t + (2x + y^2) u_x = 0$ passing through $(0, 0)$ is 
\\[1ex] \textit{[GATE EE 2025]}

(A) $y^2 = 2(e^t + x - 1)$ \hspace{2em}
(B) $y^2 = 2(e^{-t} - x + 1)$ \\
(C) $y^2 = 2(e^t - x - 1)$ \hspace{2em}
(D) $y^2 = 2(e^{-t} + x + 1)$
\end{flushleft}

\begin{flushleft}
\textbf{Q.69} The initial value problem $u_t + u_x = 1$, $u(s,s) = \sin s$, $0 \leq s \leq 1$, has 
\\[1ex] \textit{[GATE EE 2025]}


(A) two solutions \hspace{2em}
(B) a unique solution \hspace{2em}
(C) no solution \hspace{2em}
(D) infinitely many solutions
\end{flushleft}

\begin{flushleft}
\textbf{Q.70} Let $u(x,t)$ be the solution of $u_{tt} - u_{xx} = 1$, $x \in \mathbb{R}, t > 0$, with \\
$u(x,0) = 0, \ u_t(x,0) = 0, \ x \in \mathbb{R}$. Then $u(1/2, 1/2)$ is equal to 
\\[1ex] \textit{[GATE EE 2025]}

(A) $\dfrac{1}{8}$ \hspace{2em} 
(B) $-\dfrac{1}{8}$ \hspace{2em} 
(C) $\dfrac{1}{4}$ \hspace{2em} 
(D) $-\dfrac{1}{4}$
\end{flushleft}

\begin{flushleft}
\textbf{Common Data Questions}
\newline
\textbf{Common Data for Questions 71, 72, 73:} \\
Let $X = C([0,1])$ with sup norm $\| \cdot \|$.
\end{flushleft}

\begin{flushleft}
\textbf{Q.71} Let $S = \{ x \in X : \|x\| \leq 1 \}$. Then 
\\[1ex] \textit{[GATE EE 2025]}


(A) $S$ is convex and compact \\
(B) $S$ is not convex but compact \\
(C) $S$ is convex but not compact \\
(D) $S$ is neither convex nor compact
\end{flushleft}

\begin{flushleft}
\textbf{Q.72} Which one of the following is true? 
\\[1ex] \textit{[GATE EE 2025]}


(A) $C^\infty([0,1])$ is dense in $X$ \\
(B) $X$ is dense in $L^2([0,1])$ \\
(C) $X$ has a countable basis \\
(D) There is a sequence in $X$ which is uniformly Cauchy on $[0,1]$ but does not converge uniformly on $[0,1]$
\end{flushleft}

\begin{flushleft}
\textbf{Q.73} Let $I = \{ x \in X : x(0) = 0 \}$. Then 
\\[1ex] \textit{[GATE EE 2025]}


(A) $I$ is not an ideal of $X$ \\
(B) $I$ is an ideal, but not a prime ideal of $X$ \\
(C) $I$ is a prime ideal, but not a maximal ideal of $X$ \\
(D) $I$ is a maximal ideal of $X$
\end{flushleft}

\begin{flushleft}
\textbf{Common Data for Questions 74, 75:} \\
Let $X = C([0,1])$ and $Y = C([0,1])$, both with the sup norm. Define $F: X \to Y$ by $F(x) = x + x'$
and $f(x) = x(1) + x'(1)$ for $x \in X$.
\end{flushleft}

\begin{flushleft}
\textbf{Q.74} Then
\\[1ex] \textit{[GATE EE 2025]}

(A) $F$ and $f$ are continuous \hspace{2em}
(B) $F$ is continuous and $f$ is discontinuous \\
(C) $F$ is discontinuous and $f$ is continuous \hspace{2em}
(D) $F$ and $f$ are discontinuous
\end{flushleft}

\begin{flushleft}
\textbf{Q.75} Then
\\[1ex] \textit{[GATE EE 2025]}

(A) $F$ and $f$ are closed maps \\
(B) $F$ is a closed map and $f$ is not a closed map \\
(C) $F$ is not a closed map and $f$ is a closed map \\
(D) Neither $F$ nor $f$ is a closed map
\end{flushleft}

\bigskip

\noindent\textbf{Linked Answer Questions: Q.76 to Q.85 carry two marks each.}

\bigskip

\noindent\textbf{Statement for Linked Answer Questions 76 \& 77:} \\
Let $N = \begin{bmatrix} \frac{3}{5} & -\frac{4}{5} & 0 \\ \frac{4}{5} & \frac{3}{5} & 0 \\ 0 & 0 & 1 \end{bmatrix}$

\begin{flushleft}
\textbf{Q.76} Then $N$ is 
\\[1ex] \textit{[GATE EE 2025]}

(A) non-invertible \hspace{2em}
(B) skew-symmetric \\
(C) symmetric \hspace{2em}
(D) orthogonal
\end{flushleft}

\begin{flushleft}
\textbf{Q.77} If $M$ is any $3 \times 3$ real matrix, then $\text{trace}(NMN^T)$ is equal to 
\\[1ex] \textit{[GATE EE 2025]}

(A) $\left( \text{trace}(N) \right)^2 \text{trace}(M)$ \\
(B) $2\, \text{trace}(N) + \text{trace}(M)$ \\
(C) $\text{trace}(M)$ \\
(D) $\left( \text{trace}(N) \right)^2 + \text{trace}(M)$
\end{flushleft}

\bigskip

\noindent\textbf{Statement for Linked Answer Questions 78 \& 79:} \\
Let $f(z) = \cos z - \dfrac{\sin z}{z}$ for non-zero $z \in \mathbb{C}$ and $f(0) = 0$. Also, let $g(z) = \sinh z$ for $z \in \mathbb{C}$.

\begin{flushleft}
\textbf{Q.78} Then $f(z)$ has a zero at $z = 0$ of order 
\\[1ex] \textit{[GATE EE 2025]}

(A) 0 \hspace{2em} (B) 1 \hspace{2em} (C) 2 \hspace{2em} (D) greater than 2
\end{flushleft}

\begin{flushleft}
\textbf{Q.79} Then $\dfrac{g(z)}{f(z)}$ has a pole at $z = 0$ of order 
\\[1ex] \textit{[GATE EE 2025]}

(A) 1 \hspace{2em} (B) 2 \hspace{2em} (C) 3 \hspace{2em} (D) greater than 3
\end{flushleft}

\bigskip

\noindent\textbf{Statement for Linked Answer Questions 80 \& 81:} \\
Let $n \geq 3$ be an integer. Let $y$ be the polynomial solution of \\
\[
(1 - x^2)y'' - 2xy' + n(n-1)y = 0
\]
satisfying $y(1) = 1$.

\begin{flushleft}
\textbf{Q.80} Then the degree of $y$ is
\\[1ex] \textit{[GATE EE 2025]}


(A) $n$ \hspace{2em} (B) $n - 1$ \hspace{2em} (C) less than $n - 1$ \hspace{2em} (D) greater than $n + 1$
\end{flushleft}

\begin{flushleft}
\textbf{Q.81} If $I = \int_{-1}^{1} y(x) x^n \, dx$ and $J = \int_{-1}^{1} y(x) x^2 \, dx$, then 
\\[1ex] \textit{[GATE EE 2025]}

(A) $I \neq 0$, $J \neq 0$ \hspace{2em}
(B) $I \neq 0$, $J = 0$ \\
(C) $I = 0$, $J \neq 0$ \hspace{2em}
(D) $I = 0$, $J = 0$
\end{flushleft}

\noindent\textbf{Statement for Linked Answer Questions 82 \& 83:} \\
Consider the boundary value problem:
\[
u_{xx} + u_{yy} = 0,\quad x \in (0,\pi),\ y \in (0,\pi), \\
u(x,0) = u(x,\pi) = u(0,y) = 0.
\]

\begin{flushleft}
\textbf{Q.82} Any solution of this boundary value problem is of the form 
\\[1ex] \textit{[GATE EE 2025]}


(A) $\sum a_n \sinh nx \sin ny$ \hspace{2em}
(B) $\sum a_n \cosh nx \sin ny$ \\
(C) $\sum a_n \sinh nx \cos ny$ \hspace{2em}
(D) $\sum a_n \cosh nx \cos ny$
\end{flushleft}

\begin{flushleft}
\textbf{Q.83} If an additional boundary condition $u_x(\pi, y) = \sin y$ is satisfied, then $u(x, \pi/2)$ is equal to 
\\[1ex] \textit{[GATE EE 2025]}

(A) $\dfrac{\pi}{2} \dfrac{(e^x - e^{-x})(e^x + e^{-x})}{(e^x + e^{-x})}$ \hspace{2em}
(B) $\dfrac{\pi (e^x + e^{-x})}{(e^x - e^{-x})}$ \\
(C) $\dfrac{\pi (e^x - e^{-x})}{(e^x + e^{-x})}$ \hspace{2em}
(D) $\dfrac{\pi}{2} \dfrac{(e^x + e^{-x})(e^x + e^{-x})}{(e^x + e^{-x})}$
\end{flushleft}

\vspace{1em}
\noindent\textbf{Statement for Linked Answer Questions 84 \& 85:} \\
Let a random variable $X$ follow the exponential distribution with mean 2. Define \\
\[
Y = \left\lfloor X - 2 \,\middle|\, X > 2 \right\rfloor
\]

\begin{flushleft}
\textbf{Q.84} The value of $P(Y \geq 1)$ is 
\\[1ex] \textit{[GATE EE 2025]}


(A) $e^{-1/2}$ \hspace{2em} (B) $e^{-2}$ \hspace{2em} (C) $\dfrac{1}{2} e^{-1/2}$ \hspace{2em} (D) $\dfrac{1}{2} e^{-1}$
\end{flushleft}

\begin{flushleft}
\textbf{Q.85} The value of $E(Y)$ is equal to 
\\[1ex] \textit{[GATE EE 2025]}

(A) $\dfrac{1}{4}$ \hspace{2em} (B) $\dfrac{1}{2}$ \hspace{2em} (C) $1$ \hspace{2em} (D) $2$
\end{flushleft}

\vspace{2em}
\begin{center}
\textbf{END OF THE QUESTION PAPER}
\end{center}

\end{document}



