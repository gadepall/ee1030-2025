\let\negmedspace\undefined
\let\negthickspace\undefined
\documentclass[journal]{IEEEtran}
\usepackage[a5paper, margin=10mm, onecolumn]{geometry}
%\usepackage{lmodern} % Ensure lmodern is loaded for pdflatex
\usepackage{tfrupee} % Include tfrupee package
\setlength{\headheight}{1cm} % Set the height of the header box
\setlength{\headsep}{0mm}     % Set the distance between the header box and the top of the text
\usepackage{gvv-book}
\usepackage{gvv}
\usepackage{cite}
\usepackage{amsmath,amssymb,amsfonts,amsthm}
\usepackage{algorithmic}
\usepackage{graphicx}
\usepackage{textcomp}
\usepackage{xcolor}
\usepackage{txfonts}
\usepackage{listings}
\usepackage{enumitem}
\usepackage{mathtools}
\usepackage{gensymb}
\usepackage{comment}
\usepackage[breaklinks=true]{hyperref}
\usepackage{tkz-euclide}
\usepackage{multicol}
\usepackage{listings}                                        
\def\inputGnumericTable{}                                 
\usepackage[latin1]{inputenc}                                
\usepackage{color}                                            
\usepackage{array}                                            
\usepackage{longtable}                                       
\usepackage{calc}                                             
\usepackage{multirow}                                         
\usepackage{hhline}
\usepackage{ifthen}                                           
\usepackage{lscape}
\usepackage{circuitikz}
\renewcommand{\thefigure}{\theenumi}
\renewcommand{\thetable}{\theenumi}
\setlength{\intextsep}{10pt} % Space between text and floats
\numberwithin{equation}{enumi}
\numberwithin{figure}{enumi}
\renewcommand{\thetable}{\theenumi}

\begin{document}
\bibliographystyle{IEEEtran}

\begin{center}
    \LARGE \textbf{GATE 2012 MA}\\[0.5em]
    \large \textbf{EE25BTECH11001 - AARUSH DILAWRI}
\end{center}
\begin{center}
 \textbf{Q.1-Q.25 carry one mark each.}
\end{center}
\vspace{0.25em}

\begin{enumerate}

\item
The straight lines $L_1: x = 0$, $L_2: y = 0$, and $L_3: x + y = 1$ are mapped by the transformation $w = z^2$ into the curves $C_1$, $C_2$, and $C_3$ respectively. The angle of intersection between the curves at $w = 0$ is
\hfill{\text{GATE MA 2012}}
\begin{multicols}{2}
\begin{enumerate}
  \item $0$
  \item $\dfrac{\pi}{4}$
  \item $\dfrac{\pi}{2}$
  \item $\pi$
\end{enumerate}
\end{multicols}

\item
In a topological space, which of the following statements is NOT always true:

\hfill{\text{GATE MA 2012}}

\begin{enumerate}
  \item Union of any finite family of compact sets is compact.
  \item Union of any family of closed sets is closed.
  \item Union of any family of connected sets having a non empty intersection is connected.
  \item Union of any family of dense subsets is dense.
\end{enumerate}


\item
Consider the following statements:
P: The family of subsets $A_n = \cbrak{-n, -n+1, \ldots, n}$ for $n=1,2,\ldots$ satisfies the finite intersection property.

Q: On an infinite set $X$ define the metric $d: X \times X \to \mathbb{R}$ as
\begin{align}
d(x, y) =
\begin{cases}
0, & \text{if } x = y \\
1, & \text{if } x \ne y
\end{cases}
\end{align}
The metric space $(X, d)$ is compact.

R: In a Frechet ($T_1$) topological space, every finite set is closed.

S: If $f: \mathbb{R} \to X$ is continuous, where $\mathbb{R}$ has the usual topology and $(X, \tau)$ is a Hausdorff ($T_2$) space, then $f$ is a one-one function.

Which of the above statements are correct?
\hfill{\text{GATE MA 2012}}
\begin{multicols}{2}
\begin{enumerate}
  \item P and R
  \item P and S
  \item R and S
  \item Q and S
\end{enumerate}
\end{multicols}

\item
Let $H$ be a Hilbert space and $S^\perp$ denote the orthogonal complement of a set $S \subseteq H$. Which of the following is INCORRECT?
\hfill{\text{GATE MA 2012}}
\begin{multicols}{2}
\begin{enumerate}
  \item For $S_1 \subseteq S_2 \subseteq H$, we have $S_2^\perp \subseteq S_1^\perp$.
  \item $(S^\perp)^\perp \subseteq S$
  \item $\{0\}^\perp = H$
  \item $S^\perp$ is always closed.
\end{enumerate}
\end{multicols}

\item
Let $H$ be a complex Hilbert space, $T: H \to H$ a bounded linear operator and $T^*$ its adjoint. Which of the following statements are always TRUE?
P: $\langle Tx, y \rangle = \langle x, T^*y \rangle$ for all $x, y \in H$

Q: $\langle x, Ty \rangle = T \langle x, y \rangle$ for all $x, y \in H$

R: $\langle x, Ty \rangle = \langle x, T^*y \rangle$ for all $x, y \in H$

S: $\langle Tx, Ty \rangle = T^* \langle x, Ty \rangle$ for all $x, y \in H$

\hfill{\text{GATE MA 2012}}
\begin{multicols}{2}
\begin{enumerate}
  \item P and Q
  \item P and R
  \item Q and S
  \item P and S
\end{enumerate}
\end{multicols}

\item
Let $X = \cbrak{a, b, c}$ and $\mathcal{T} = \cbrak{\phi, \cbrak{a}, \cbrak{b}, \cbrak{a, b}, X}$ be a topology defined on $X$. Which statements are TRUE?
P: $(X, \mathcal{T})$ is a Hausdorff space.

Q: $(X, \mathcal{T})$ is a regular space.

R: $(X, \mathcal{T})$ is a normal space.

S: $(X, \mathcal{T})$ is a connected space.

\hfill{\text{GATE MA 2012}}
\begin{multicols}{2}
\begin{enumerate}
  \item P and Q
  \item Q and R
  \item R and S
  \item P and S
\end{enumerate}
\end{multicols}

\item
Consider the statements:

P: If $X$ is a normed linear space and $M \subseteq X$ is a subspace, then the closure $\bar{M}$ is also a subspace of $X$.

Q: If $X$ is a Banach space and $\sum x_n$ is an absolutely convergent series in $X$, then $\sum x_n$ is convergent.

R: Let $M_1$ and $M_2$ be subspaces of an inner product space such that $M_1 \cap M_2 = \phi$. Then for all $m_1 \in M_1, m_2 \in M_2$: $\| m_1 + m_2 \|^2 = \| m_1 \|^2 + \| m_2 \|^2$.

S: Let $f: X \to Y$ be a linear transformation from the Banach Space $X$ into the Banach space $Y$. If $f$ is continuous, then the graph of $f$ is always compact.

The correct statements amongst the above are:
\hfill{\text{GATE MA 2012}}
\begin{multicols}{2}
\begin{enumerate}
  \item P and R only
  \item Q and R only
  \item P and Q only
  \item R and S only
\end{enumerate}
\end{multicols}

\item
A continuous random variable $X$ has the probability density function
\begin{align}
f(x) = 
\begin{cases}
\frac{3}{5} x^3 e^{-x^5/5}, & x > 0 \\
0, & x \leq 0
\end{cases}
\end{align}
The probability density function of $Y = 2X + 3$ is
\hfill{\text{GATE MA 2012}}
\begin{multicols}{2}
\begin{enumerate}
  \item $\frac{1}{5} (y-2)^3 e^{-(y-2)^5/5}, \quad y>2$
  \item $\frac{2}{5} (y-2)^3 e^{-2(y-2)^5/5}, \quad y>2$
  \item $\frac{3}{5} (y-2)^3 e^{-3(y-2)^5/5}, \quad y>2$
  \item $\frac{4}{5} (y-2)^3 e^{-4(y-2)^5/5}, \quad y>2$
\end{enumerate}
\end{multicols}

\item
A simple random sample of size $10$ from $N(\mu, \sigma^2)$ gives $98\%$ confidence interval $(20.49, 23.51)$. Then the null hypothesis $H_0: \mu = 20.5$ against $H_A: \mu \neq 20.5$

\hfill{\text{GATE MA 2012}}

\begin{enumerate}
  \item can be rejected at $2\%$ level of significance
  \item cannot be rejected at $5\%$ level of significance
  \item can be rejected at $10\%$ level of significance
  \item cannot be rejected at any level of significance
\end{enumerate}


\item
For the linear programming problem
\begin{align}
\text{Maximize} \qquad & z = x_1 + 2x_2 + 3x_3 - 4x_4 \\
\text{Subject to} \qquad & x_1 + 2x_2 - 3x_3 - x_4 = 15 \\
& x_1 + 6x_2 + 3x_3 - x_4 = 21 \\
& x_1 + 8x_2 + 2x_3 - 4x_4 = 30 \\
& x_1, x_2, x_3, x_4 \geq 0
\end{align}
and $x_1 = 4$, $x_2 = 3$, $x_3 = 0$, $x_4 = 2$ is
\hfill{\text{GATE MA 2012}}

\begin{enumerate}
  \item an optimal solution
  \item a degenerate basic feasible solution
  \item a non-degenerate basic feasible solution
  \item a non-basic feasible solution
\end{enumerate}



\item
Which one of the following statements is TRUE?
\hfill{\text{GATE MA 2012}}

\begin{enumerate}
  \item A convex set cannot have infinitely many extreme points.
  \item A linear programming problem can have infinitely many extreme points.
  \item A linear programming problem can have exactly two different optimal solutions.
  \item A linear programming problem can have a non-basic optimal solution.
\end{enumerate}


\item
Let $\alpha = e^{2 \pi i/5}$ and the matrix
\begin{align}
M = \myvec{
\alpha & \alpha^2 & \alpha^3 & 4 \\
\alpha^2 & \alpha^3 & \alpha^4 & 0 \\
0 & 0 & 0 & 0 \\
0 & 0 & 0 & 0
}
\end{align}
Then the trace of the matrix $I + M + M^2$ is
\hfill{\text{GATE MA 2012}}
\begin{multicols}{4}
\begin{enumerate}
  \item $5$
  \item $0$
  \item $3$
  \item $-5$
\end{enumerate}
\end{multicols}

\item
Let $V = \mathbb{C}^2$ be the vector space over the complex numbers and $B = \{(1, i), (i, 1)\}$ be a given ordered basis of $V$. Then which of the following pairs $\{f_1, f_2\}$ is a dual basis of $B$ over $\mathbb{C}$?
\hfill{\text{GATE MA 2012}}

\begin{enumerate}
  \item $f_1(z_1, z_2) = \frac{z_1 - z_2}{2}, \quad f_2(z_1, z_2) = \frac{z_1 + z_2}{2}$
  \item $f_1(z_1, z_2) = \frac{z_1 + z_2}{2}, \quad f_2(z_1, z_2) = \frac{z_1 + z_2 i}{2}$
  \item $f_1(z_1, z_2) = \frac{z_1 - z_2}{2}, \quad f_2(z_1, z_2) = \frac{z_1 - z_2 i}{2}$
  \item $f_1(z_1, z_2) = \frac{z_1 + z_2}{2}, \quad f_2(z_1, z_2) = \frac{z_1 - z_2}{2}$
\end{enumerate}


\item
Let $R = \mathbb{Z} \times \mathbb{Z} \times \mathbb{Z}$ and $I = \mathbb{Z} \times \mathbb{Z} \times \{0\}$. Which of the following statements is correct?
\hfill{\text{GATE MA 2012}}

\begin{enumerate}
  \item $I$ is a maximal ideal but not a prime ideal of $R$.
  \item $I$ is a prime ideal but not a maximal ideal of $R$.
  \item $I$ is both a maximal ideal as well as a prime ideal of $R$.
  \item $I$ is neither a maximal ideal nor a prime ideal of $R$.
\end{enumerate}


\item
The function $u(r, \theta)$ satisfying the Laplace equation
\begin{align}
\frac{\partial^2 u}{\partial r^2} + \frac{1}{r} \frac{\partial u}{\partial r} + \frac{1}{r^2} \frac{\partial^2 u}{\partial \theta^2} = 0, \quad 0 < r < e, \quad 0 < \theta < 2\pi
\end{align}
subject to the conditions
\begin{align}
u(e, \theta) = 1, \quad u(e, \theta) = 0
\end{align}
is
\hfill{\text{GATE MA 2012}}
\begin{multicols}{2}
\begin{enumerate}
  \item $\ln \frac{r}{e}$
  \item $2 \ln \frac{r}{e}$
  \item $\ln \left(\frac{r}{e}\right)^2$
  \item $\sum_{n=1}^\infty \frac{1}{n} \sin(n \theta) \left(\frac{r}{e}\right)^n$
\end{enumerate}
\end{multicols}

\item
The functional
\begin{align}
J[y] = \int_0^1 \left(y'^2 + 2kxyy' + y^2 + y' + y\right) dx,
\end{align}
with boundary conditions $y(0) = 0$, $y(1) = 1$, $y'(0) = 2$, $y'(1) = 3$ is path independent if $k$ equals
\hfill{\text{GATE MA 2012}}
\begin{multicols}{4}
\begin{enumerate}
  \item $1$
  \item $2$
  \item $3$
  \item $4$
\end{enumerate}
\end{multicols}

\item
If the transformation $y = uv$ transforms the differential equation
\begin{align}
f''(x) - f'(x) + g(x) = 0
\end{align}
into the equation of the form 
\begin{align}
v'' + h(x) v = 0,
\end{align}
then $u$ must be
\hfill{\text{GATE MA 2012}}
\begin{multicols}{4}
\begin{enumerate}
  \item $\frac{1}{\sqrt{f}}$
  \item $x f$
  \item $\frac{1}{2 f}$
  \item $f^2$
\end{enumerate}
\end{multicols}

\item
The expression
\begin{align}
D = \frac{\partial^2}{\partial x^2} - 2\frac{\partial^2}{\partial x \partial y} - \frac{\partial^2}{\partial y^2}
\end{align}
is equal to
\hfill{\text{GATE MA 2012}}
\begin{multicols}{2}
\begin{enumerate}
  \item $\cos(x - y)$
  \item $\sin(x - y) + \cos(x + y)$
  \item $\cos(x - y) + \sin(x + y)$
  \item $3 \sin(x - y)$
\end{enumerate}
\end{multicols}

\item
The function $\phi(x)$ satisfying the integral equation
\begin{align}
\phi(x) = 2 \int_0^x e^{-\xi^2} \phi(\xi) d\xi
\end{align}
is
\hfill{\text{GATE MA 2012}}
\begin{multicols}{2}
\begin{enumerate}
  \item $e^{x^2}$
  \item $e^{x^2} + x$
  \item $e^{x^2} - x$
  \item $x^2 + \frac{1}{2}$
\end{enumerate}
\end{multicols}

\item
Given data:
\begin{center}
\begin{tabular}{ll}
    \textbf{Group I} & \textbf{Group II} \\
    P. Ferrite & 1. Hexagonal Close Packed (HCP) \\
    Q. Austenite & 2. Body Centered Cubic (BCC) \\
    R. Martensite & 3. Body Centered Tetragonal (BCT) \\
    & 4. Face Centered Cubic (FCC)
\end{tabular}
\end{center}

If the derivative of $y(x)$ is approximated as
\begin{align}
y'(x_k) \approx \frac{1}{2h} \left(-y_{k+2} + 4 y_{k+1} - 3 y_k\right),
\end{align}
then the value of $y'(2)$ is
\hfill{\text{GATE MA 2012}}
\begin{multicols}{4}
\begin{enumerate}
  \item $4$
  \item $8$
  \item $12$
  \item $16$
\end{enumerate}
\end{multicols}


\item
If 
\begin{align}
A = \myvec{
1 & 0 & 0 \\
1 & 0 & 1 \\
0 & 1 & 0
},
\end{align}
then $A^{50}$ is
\hfill{\text{GATE MA 2012}}
\begin{multicols}{2}
\begin{enumerate}
  \item $\myvec{
1 & 0 & 0 \\
50 & 1 & 0 \\
50 & 0 & 1
}$
  \item $\myvec{
1 & 0 & 0 \\
48 & 1 & 0 \\
48 & 0 & 1
}$
  \item $\myvec{
1 & 0 & 0 \\
25 & 1 & 0 \\
25 & 0 & 1
}$
  \item $\myvec{
1 & 0 & 0 \\
24 & 1 & 0 \\
24 & 0 & 1
}$
\end{enumerate}
\end{multicols}

\item
If
\begin{align}
y = \sum_{m=0}^\infty c_m x^{m+r}
\end{align}
is assumed to be a solution of the differential equation
\begin{align}
x^2 y'' - 3 x y' - x y = 0,
\end{align}
then the values of $r$ are
\hfill{\text{GATE MA 2012}}
\begin{multicols}{2}
\begin{enumerate}
  \item $1$ and $3$
  \item $-1$ and $3$
  \item $1$ and $-3$
  \item $-1$ and $-3$
\end{enumerate}
\end{multicols}

\item
Let the linear transformation 
\begin{align}
T : F^2 \to F^3
\end{align}
be defined by
\begin{align}
T(x_1, x_2) = (x_1 + x_2, x_1, x_2).
\end{align}
Then the nullity of $T$ is
\hfill{\text{GATE MA 2012}}
\begin{multicols}{4}
\begin{enumerate}
  \item $0$
  \item $1$
  \item $2$
  \item $3$
\end{enumerate}
\end{multicols}

\item
The approximate eigenvalue of the matrix
\begin{align}
A = \myvec{
15 & 4 & 3 \\
10 & 12 & 6 \\
20 & 4 & 2
}
\end{align}
obtained after two iterations of the Power method, with the initial vector $(1,1,1)^T$, is
\hfill{\text{GATE MA 2012}}
\begin{multicols}{4}
\begin{enumerate}
  \item $7.768$
  \item $9.468$
  \item $10.548$
  \item $19.468$
\end{enumerate}
\end{multicols}

\item
The root of the equation $x e^x = 1$ between 0 and 1, obtained by using two iterations of the bisection method, is
\hfill{\text{GATE MA 2012}}
\begin{multicols}{4}
\begin{enumerate}
  \item $0.25$
  \item $0.50$
  \item $0.75$
  \item $0.65$
\end{enumerate}
\end{multicols}

\item
Let
\begin{align}
\int_C \frac{a\,dz}{(z^2-4)(z-2)} = \pi,
\end{align}
where the closed curve $C$ is the triangle with vertices at $-1$, $i$, $-i$ and the integral is taken in anti-clockwise direction. Then one value of $a$ is
\hfill{\text{GATE MA 2012}}
\begin{multicols}{4}
\begin{enumerate}
\item $1+i$
\item $2+i$
\item $3+i$
\item $4+i$
\end{enumerate}
\end{multicols}

\item
The Lebesgue measure of the set
\begin{align}
A = \cbrak{x:\ \sin x < x,\ 0 < x \leq 1}
\end{align}
is
\hfill{\text{GATE MA 2012}}
\begin{multicols}{4}
\begin{enumerate}
\item $0$
\item $1$
\item $\ln 2$
\item $1 - \ln 2$
\end{enumerate}
\end{multicols}

\item
Which of the following statements are TRUE?

P.\ The set $\cbrak{x \in \mathbb{R}: \cos x \leq 1}$ is compact.

Q.\ The set $\cbrak{x \in \mathbb{R}: \tan x \text{ is not differentiable}}$ is complete.

R.\ The set $\cbrak{x \in \mathbb{R}: \sum_{n=0}^{\infty} \frac{x^{2n+1}}{(2n+1)!}\ \text{is convergent}}$ is bounded.

S.\ The set $\cbrak{x \in \mathbb{R}: f(x) = \cos x\ \text{has local maxima}}$ is closed.
\hfill{\text{GATE MA 2012}}
\begin{multicols}{2}
\begin{enumerate}
\item P and Q
\item R and S
\item Q and S
\item P and S
\end{enumerate}
\end{multicols}

\item
If a random variable $X$ assumes only positive integer values, with probability
\begin{align}
P(X=x) = \frac{1}{2}\frac{1}{3^{x-1}},\quad x=1,2,3,\ldots
\end{align}
then $\mathbb{E}[X]$ is
\hfill{\text{GATE MA 2012}}
\begin{multicols}{4}
\begin{enumerate}
\item $\dfrac{2}{9}$
\item $\dfrac{2}{3}$
\item $1$
\item $\dfrac{3}{2}$
\end{enumerate}
\end{multicols}

\item
The probability density function of the random variable $X$ is
\begin{align}
f(x)=
\begin{cases}
\lambda e^{-\lambda x},& x>0\\
0,& x \leq 0
\end{cases}
\end{align}
where $\lambda>0$. For testing the hypothesis $H_0: \lambda=3$ against $H_A: \lambda=5$, a test is given as "Reject $H_0$ if $X \geq 4.5$". The probability of type I error and power of this test are, respectively,
\hfill{\text{GATE MA 2012}}
\begin{multicols}{2}
\begin{enumerate}
\item $0.1353,\ 0.4966$
\item $0.1827,\ 0.379$
\item $0.2021,\ 0.4493$
\item $0.2231,\ 0.4066$
\end{enumerate}
\end{multicols}

\item
The order of the smallest possible non-trivial group containing elements $x$ and $y$ such that $x^7 = y^2 = e$ and $yx = x^4 y$ is
\hfill{\text{GATE MA 2012}}
\begin{multicols}{4}
\begin{enumerate}
\item $1$
\item $2$
\item $7$
\item $14$
\end{enumerate}
\end{multicols}

\item
The number of 5-Sylow subgroups in a group of order $45$ is
\hfill{\text{GATE MA 2012}}
\begin{multicols}{4}
\begin{enumerate}
\item $1$
\item $2$
\item $3$
\item $4$
\end{enumerate}
\end{multicols}

\item
The solution of the initial value problem
\begin{align}
y'' + 10y' + 6y = \delta(t), \quad y(0) = 0, \quad y'(0) = 0
\end{align}
where $\delta(t)$ denotes the Dirac delta function, is
\hfill{\text{GATE MA 2012}}
\begin{multicols}{2}
\begin{enumerate}
\item $2 e^{-3t} \sin t$
\item $6 e^{-3t} \sin t$
\item $2 e^{-t} \sin 3t$
\item $6 e^{-t} \sin 3t$
\end{enumerate}
\end{multicols}

\item
Let $\omega = \cos \frac{2\pi}{3} + i \sin \frac{2\pi}{3}$, $M = \begin{pmatrix} 0 & 0 \\ i & 0 \end{pmatrix}$, $N = \begin{pmatrix} 0 & i \\ \omega & 0 \end{pmatrix}$, and $G = \myvec{M \\ N}$ be the group generated by $M$ and $N$ under multiplication. Then
\hfill{\text{GATE MA 2012}}
\begin{multicols}{4}
\begin{enumerate}
\item $C_6$
\item $S_3$
\item $C_2$
\item $C_4$
\end{enumerate}
\end{multicols}

\item
The flux of the vector field $u = x \hat{i} + y \hat{j} + z \hat{k}$ flowing out through the surface of the ellipsoid
\begin{align}
\frac{x^2}{a^2} + \frac{y^2}{b^2} + \frac{z^2}{c^2} = 1, \quad a,b,c > 0,
\end{align}
is
\hfill{\text{GATE MA 2012}}
\begin{multicols}{4}
\begin{enumerate}
\item $abc \pi$
\item $2 abc \pi$
\item $3 abc \pi$
\item $4 abc \pi$
\end{enumerate}
\end{multicols}

\item
The integral surface satisfying the partial differential equation
\begin{align}
\frac{\partial z}{\partial x} + \frac{\partial z}{\partial y} = z^2
\end{align}
and passing through the line $x=1$, $y=1$, $z=1$ is
\hfill{\text{GATE MA 2012}}
\begin{multicols}{2}
\begin{enumerate}
\item $x - z^2 + y = 1$
\item $x^2 + y^2 - z^2 = 1$
\item $y - z^2 + x = 1$
\item $x - z^2 + y = 1$
\end{enumerate}
\end{multicols}

\item
The diffusion equation
\begin{align}
u_{xx} = u_t, \quad u(x,0) = \cos x + \sin 5x
\end{align}
admits the solution
\hfill{\text{GATE MA 2012}}
\begin{multicols}{2}
\begin{enumerate}
\item $e^{-36t} \sin 6x + e^{-20t} \sin 4x$
\item $e^{-36t} \sin 4x + e^{-20t} \sin 6x$
\item $e^{-20t} \sin 3x + e^{-15t} \sin 5x$
\item $e^{-36t} \sin 5x + e^{-20t} \sin x$
\end{enumerate}
\end{multicols}

\item
Let $f(x)$ and $x f(x)$ be particular solutions of the differential equation
\begin{align}
y'' + R(x)y' + S(x)y = 0.
\end{align}
Then the solution of
\begin{align}
y'' + R(x) y' + S(x) y = f(x)
\end{align}
is
\hfill{\text{GATE MA 2012}}
\begin{multicols}{2}
\begin{enumerate}
\item $y = -\frac{1}{2} x^2 f(x) + \alpha x + \beta$
\item $y = \frac{1}{2} x^2 f(x) + \alpha x + \beta$
\item $y = - x^2 f(x) + \alpha x + \beta$
\item $y = 3 x^2 f(x) + \alpha x + \beta$
\end{enumerate}
\end{multicols}

\item
Let the Legendre equation
\begin{align}
(1 - x^2) y'' - 2x y' + n(n+1) y = 0
\end{align}
have a $n$th degree polynomial solution $y_n(x)$ such that $y_n(1) = 3$. If
\begin{align}
\int_{-1}^1 y_n(x) y_n(x) dx = \frac{144}{15},
\end{align}
then $n$ is
\hfill{\text{GATE MA 2012}}
\begin{multicols}{4}
\begin{enumerate}
\item $1$
\item $2$
\item $3$
\item $4$
\end{enumerate}
\end{multicols}

\item
The maximum value of the function $f(x,y,z) = xyz$ subject to the constraint $xy + yz + zx = a$, $a > 0$, is
\hfill{\text{GATE MA 2012}}
\begin{multicols}{4}
\begin{enumerate}
\item $\frac{a^2}{3}$
\item $\frac{a^2}{3^3}$
\item $\frac{a^2}{3}(3)$
\item $\frac{2 a^2}{3}$
\end{enumerate}
\end{multicols}

\item
The functional
\begin{align}
J[y] = \int_0^1 \Big( y'^4 + 4 y^3 y' + 4 y^2 e^{y'} \Big) \, dx,
\end{align}
with boundary conditions $y(0) = -1$, $y(1) = -1$, possesses:
\hfill{\text{GATE MA 2012}}
\begin{multicols}{2}
\begin{enumerate}
\item strong minima on $x = \frac{1}{3} y e = -1$
\item strong minima on $x = \frac{4}{3} y e = -1$
\item weak maxima on $x = \frac{1}{3} y e = -1$
\item strong maxima on $x = \frac{4}{3} y e = -1$
\end{enumerate}
\end{multicols}

\item
A particle of mass $m$ constrained to move on a circle of radius $a$, rotating about its vertical diameter with constant angular velocity $\omega$. Initial angular velocity zero, $g$ gravitational acceleration. With $\theta$ inclination and $\dot{\theta} = \frac{d\theta}{dt}$, the Lagrangian is
\begin{align}
L = \frac{1}{2} m a^2 \brak{\dot{\theta}^2 + \omega^2 \sin^2 \theta} + m g a \cos \theta,
\end{align}
or one of the options:
\hfill{\text{GATE MA 2012}}
\begin{multicols}{2}
\begin{enumerate}
\item $\frac{1}{2} m a^2 \brak{\dot{\theta}^2 + \omega^2 \sin^2 \theta} + m g a \cos \theta$
\item $\frac{1}{2} m a^2 \brak{\dot{\theta}^2 + 2 \omega \sin \theta} - m g a \sin \theta$
\item $\frac{1}{2} m a^2 \brak{\dot{\theta}^2 + 2 \omega \cos \theta} - m g a \sin \theta$
\item $\frac{1}{2} m a^2 \brak{\dot{\theta}^2 + \sin 2\theta} + m g a \sin \theta$
\end{enumerate}
\end{multicols}

\item
For the matrix
\begin{align}
    M = \myvec{
2+3i & 4 - 3i & 2 + 5i \\
6 + 4i & 6 - 3i & \phantom{-}0
}
\end{align}


which of the following statements are correct?\\
P: $M$ is skew-Hermitian and $iM$ is Hermitian\\
Q: $M$ is Hermitian and $iM$ is skew-Hermitian\\
R: Eigenvalues of $M$ are real\\
S: Eigenvalues of $iM$ are real
\hfill{\text{GATE MA 2012}}
\begin{multicols}{2}
\begin{enumerate}
\item P and R only
\item Q and R only
\item P and S only
\item Q and S only
\end{enumerate}
\end{multicols}

\item
Let $T: P_3 \to P_3$ be the map defined by
\begin{align}
    T(p)(x) = \int_0^x p(t) \, dt.
\end{align}

If the matrix of $T$ relative to the standard basis $\cbrak{1, x, x^2, x^3}$ is $M$ and $M'$ denotes transpose of $M$, then $M + M'$ is
\hfill{\text{GATE MA 2012}}
\begin{enumerate}
\item $\myvec{
0 & 1 & 1 & 1 \\
1 & 2 & 0 & 0 \\
0 & 0 & 2 & 0 \\
0 & 0 & 0 & 2
}$
\item $\myvec{
1 & 0 & 0 & 2 \\
0 & 1 & 1 & 0 \\
0 & 1 & 2 & 0 \\
2 & 0 & 0 & 1
}$
\item $\myvec{
2 & 0 & 0 & 1 \\
0 & 2 & 1 & 0 \\
1 & 1 & 0 & 1 \\
0 & 1 & -1 & 0
}$
\item $\myvec{
0 & 2 & 2 & 2 \\
2 & 1 & 0 & 0 \\
2 & 0 & 1 & 0 \\
2 & 0 & 0 & 1
}$
\end{enumerate}

\item
Using Eulers method with step size $h=0.1$, the approximate value of $y$ at $x=0.2$ for the initial value problem
\begin{align}
    \frac{dy}{dx} = x^2 + y^2, \quad y(0) = 1
\end{align}

is
\hfill{\text{GATE MA 2012}}
\begin{multicols}{4}
\begin{enumerate}
\item 1.322
\item 1.122
\item 1.222
\item 1.110
\end{enumerate}
\end{multicols}

\item
Transportation problem data:


\begin{tabular}{|c|c|c|}
     \hline
     \textbf{Mineral} & \textbf{Modal abundance \brak{\%}} & \textbf{Partition coefficient}\\
     \hline
     Clinopyroxene & $45$ & $0.506$ \\
      \hline
      Orthopyroxene & $40$ & $0.42$ \\
      \hline
      Olivine & $10$ & $0.045$ \\
      \hline
      Plagioclase & $05$ & $0.019$ \\
      \hline
\end{tabular}

Basic feasible solution:
\begin{align}
    x_{12}=60, x_{22}=10, x_{23}=50, x_{24}=20, x_{31}=40, x_{34}=60.
\end{align}

The variables entering and leaving the basis are:
\hfill{\text{GATE MA 2012}}
\begin{multicols}{2}
\begin{enumerate}
\item $x_{11}$ and $x_{24}$
\item $x_{13}$ and $x_{23}$
\item $x_{14}$ and $x_{24}$
\item $x_{33}$ and $x_{24}$
\end{enumerate}
\end{multicols}

\item
System of equations
\begin{align}
    \myvec{
5 & 1 & 1 \\
10 & 2 & 4 \\
0 & 12 & 1
}
\myvec{x \\ y \\ z} =
\myvec{1 \\ 1 \\ 5}.
\end{align}


Using Jacobis method with initial guess $\myvec{2.0 \\ 3.0 \\ 0.0}$, the approximate solution after two iterations is
\hfill{\text{GATE MA 2012}}
\begin{multicols}{2}
\begin{enumerate}
\item $\myvec{2.64 \\ -1.70 \\ -1.12}$
\item $\myvec{2.64 \\ 1.70 \\ -1.12}$
\item $\myvec{2.64 \\ 1.70 \\ 1.12}$
\item $\myvec{2.64 \\ -1.70 \\ 1.12}$
\end{enumerate}
\end{multicols}

\item
For the primal linear programming problem:
\begin{align}
    \text{Maximize}\ z = 6x_1 + 12x_2 + 12x_3 + 6x_4,
\end{align}

subject to
\begin{align}
    x_1 + x_2 + x_3 + x_4 = 3, \quad x_i \geq 0,
\end{align}

with optimal basic variables $x_1, x_2, x_3, x_4$ having RHS constants $3/4, 0, 1, -1/4$ and $1/4, 1, 0, 1/4$ respectively and given $\boldsymbol{z} = 48$, then:

If $y_1$ and $y_2$ are dual variables for constraints 1 and 2, their values in the optimal dual solution are:
\hfill{\text{GATE MA 2012}}
\begin{multicols}{2}
\begin{enumerate}
\item $0$ and $6$
\item $12$ and $0$
\item $6$ and $3$
\item $4$ and $4$
\end{enumerate}
\end{multicols}

\item
If the right hand side of the second constraint changes from $8$ to $20$, the basic variables in the primal optimal solution will be:
\hfill{\text{GATE MA 2012}}
\begin{multicols}{2}
\begin{enumerate}
\item $x_1$ and $x_2$
\item $x_1$ and $x_3$
\item $x_2$ and $x_3$
\item $x_2$ and $x_4$
\end{enumerate}
\end{multicols}

\item
Consider the Fredholm integral equation:
\begin{align}
    u(x) + \lambda \int_0^1 e^{x t} u(t) dt = x.
\end{align}

The resolvent kernel $R(x,t;\lambda)$ is:
\hfill{\text{GATE MA 2012}}
\begin{multicols}{4}
\begin{enumerate}
\item $t e^{-\lambda x}$
\item $t e^{\lambda x + 1}$
\item $t e^{\lambda x + 2}$
\item $t e^{-\lambda x + 2}$
\end{enumerate}
\end{multicols}

\item
The solution $u(x)$ of the above integral equation is:
\hfill{\text{GATE MA 2012}}
\begin{multicols}{2}
\begin{enumerate}
\item $\displaystyle \frac{x + 1}{1 - \lambda}$
\item $\displaystyle \frac{x - 1}{1 - \lambda^2}$
\item $\displaystyle \frac{x + 2}{1 + \lambda}$
\item $\displaystyle \frac{x - 1}{1 - \lambda}$
\end{enumerate}
\end{multicols}

\item
Given the joint pdf of two variables $X$ and $Y$:
\begin{align}
    f_{X,Y}(x,y) = \begin{cases}
2(x+y), & 0 \leq x \leq 1, 0 \leq y \leq 1, x + y \leq 1, \\
0, & \text{otherwise}.
\end{cases}
\end{align}

Then,
\hfill{\text{GATE MA 2012}}
\begin{multicols}{2}
\begin{enumerate}
\item $\mathbb{E}[X] = \frac{2}{3}$ and $\mathbb{E}[Y] = \frac{5}{5}$
\item $\mathbb{E}[X] = 1$ and $\mathbb{E}[Y] = 1$
\item $\mathbb{E}[X] = \frac{3}{6}$ and $\mathbb{E}[Y] = \frac{5}{5}$
\item $\mathbb{E}[X] = \frac{4}{6}$ and $\mathbb{E}[Y] = \frac{5}{5}$
\end{enumerate}
\end{multicols}

\item
The covariance $\operatorname{Cov}(X,Y)$ is:
\hfill{\text{GATE MA 2012}}
\begin{multicols}{4}
\begin{enumerate}
\item $-0.01$
\item $0$
\item $0.01$
\item $0.02$
\end{enumerate}
\end{multicols}

\item
Given functions
\begin{align}
f(z) = \frac{z^2}{z + \alpha}, \quad g(z) = \sinh \left( \frac{z}{\pi \alpha} \right), \quad \alpha \neq 0,
\end{align}

the residue of $f(z)$ at its pole equals 1, then the value of $\alpha$ is:

\hfill{\text{GATE MA 2012}}
\begin{multicols}{4}
\begin{enumerate}
\item $-1$
\item $1$
\item $2$
\item $3$
\end{enumerate}
\end{multicols}

\item
For the value of $\alpha$ obtained above, $g(z)$ is not conformal at the point:

\hfill{\text{GATE MA 2012}}
\begin{multicols}{4}
\begin{enumerate}
\item $\frac{i \pi}{6} + 3$
\item $\frac{i \pi}{6} + 1$
\item $\frac{2\pi}{3}$
\item $2 i \pi$
\end{enumerate}
\end{multicols}

\item
Choose the most appropriate word to complete the sentence:

Given the seriousness of the situation that he had to face, his \_\_\_ was impressive.

\hfill{\text{GATE MA 2012}}
\begin{multicols}{2}
\begin{enumerate}
\item beggary
\item nomenclature
\item jealousy
\item nonchalance
\end{enumerate}
\end{multicols}

\item
Choose the most appropriate alternative to complete the sentence:

If the tired soldier wanted to lie down, he \_\_\_ the mattress out on the balcony.

\hfill{\text{GATE MA 2012}}
\begin{multicols}{2}
\begin{enumerate}
\item should take
\item shall take
\item should have taken
\item will have taken
\end{enumerate}
\end{multicols}

\item
If \((1.001)^{1259} = 3.52\) and \((1.001)^{2062} = 7.85\), then \((1.001)^{3321} =\)
\hfill{\text{GATE MA 2012}}
\begin{multicols}{4}
\begin{enumerate}
\item 2.23
\item 4.33
\item 11.37
\item 27.64
\end{enumerate}
\end{multicols}

\item
One of the parts (A, B, C, D) in the sentence below contains an ERROR. Identify the incorrect part:

I requested that he should be given the driving test today instead of tomorrow.
\hfill{\text{GATE MA 2012}}
\begin{multicols}{2}
\begin{enumerate}
\item requested that
\item should be given
\item the driving test
\item instead of tomorrow
\end{enumerate}
\end{multicols}

\item
Which is the closest in meaning to the word "Latitude"?
\hfill{\text{GATE MA 2012}}
\begin{multicols}{2}
\begin{enumerate}
\item Eligibility
\item Freedom
\item Coercion
\item Meticulousness
\end{enumerate}
\end{multicols}

\item
There are eight bags of rice looking alike, seven have equal weight and one is slightly heavier. Using a weighing balance of unlimited capacity, the minimum number of weighings required to identify the heavier bag is
\hfill{\text{GATE MA 2012}}
\begin{multicols}{4}
\begin{enumerate}
\item 2
\item 3
\item 4
\item 8
\end{enumerate}
\end{multicols}

\item
Raju has 14 currency notes, consisting of only Rs. 20 and Rs. 10 notes. The total money value is Rs. 230. The number of Rs. 10 notes Raju has is
\hfill{\text{GATE MA 2012}}
\begin{multicols}{4}
\begin{enumerate}
\item 5
\item 6
\item 9
\item 10
\end{enumerate}
\end{multicols}

\item
One legacy of Roman legions was discipline. Military law prevailed and discipline was brutal. Discipline kept units obedient, intact, and fighting even in adverse odds and conditions.

The best summary of this passage is:
\hfill{\text{GATE MA 2012}}
\begin{multicols}{2}
\begin{enumerate}
\item Thorough regimentation was the main reason for efficiency in adverse circumstances.
\item The legions were treated inhumanly as if the men were animals.
\item Discipline was the armies inheritance from their seniors.
\item The harsh discipline led to the odds and conditions being against them.
\end{enumerate}
\end{multicols}

\item
A and B decide to meet between 1 PM and 2 PM. The first to arrive will wait no more than 15 minutes. The probability that they will meet is
\hfill{\text{GATE MA 2012}}
\begin{multicols}{4}
\begin{enumerate}
\item \(\frac{1}{4}\)
\item \(\frac{1}{16}\)
\item \(\frac{7}{16}\)
\item \(\frac{9}{16}\)
\end{enumerate}
\end{multicols}

\item
The table shows the monthly budget of an average household:

\begin{table}[htbp]
  \centering
  \caption{Table-3}
  \label{table3}
  \begin{tabular}{cc}
  \textbf{Processing Technique} & \textbf{Producct} \\ \\
    P. Calendering & 1. Pipes \\
    Q. Extrusion & 2. Disposable cups \\
    R. Injection moulding & 3. Sheets \\
    S. Thermoforming & 4. Nylon gears \\
  \end{tabular}
\end{table}

What is the approximate percentage of the monthly budget NOT spent on savings?

\hfill{\text{GATE MA 2012}}
\begin{multicols}{4}
\begin{enumerate}
\item 10\%
\item 14\%
\item 81\%
\item 86\%
\end{enumerate}
\end{multicols}

\end{enumerate}

\end{document}
