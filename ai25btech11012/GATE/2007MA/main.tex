\let\negmedspace\undefined
\let\negthickspace\undefined
\documentclass[journal]{IEEEtran}
\usepackage[a5paper, margin=10mm, onecolumn]{geometry}
%\usepackage{lmodern} % Ensure lmodern is loaded for pdflatex
\usepackage{tfrupee} % Include tfrupee package

\setlength{\headheight}{1cm} % Set the height of the header box
\setlength{\headsep}{0mm}     % Set the distance between the header box and the top of the text

\usepackage{gvv-book}
\usepackage{gvv}
\usepackage{cite}
\usepackage{amsmath,amssymb,amsfonts,amsthm}
\usepackage{algorithmic}
\usepackage{graphicx}
\usepackage{textcomp}
\usepackage{xcolor}
\usepackage{txfonts}
\usepackage{listings}
\usepackage{enumitem}
\usepackage{mathtools}
\usepackage{gensymb}
\usepackage{comment}
\usepackage[breaklinks=true]{hyperref}
\usepackage{tkz-euclide}
\usepackage{multicol}
\usepackage{listings}                                        
\def\inputGnumericTable{}                                 
\usepackage[latin1]{inputenc}                                
\usepackage{color}                                            
\usepackage{array}                                            
\usepackage{longtable}                                       
\usepackage{calc}                                             
\usepackage{multirow}                                         
\usepackage{hhline}
\usepackage{ifthen}                                           
\usepackage{lscape}
\usepackage{circuitikz}


\renewcommand{\thefigure}{\theenumi}
\renewcommand{\thetable}{\theenumi}
\setlength{\intextsep}{10pt} % Space between text and floats


\numberwithin{equation}{enumi}
\numberwithin{figure}{enumi}
\renewcommand{\thetable}{\theenumi}


% Marks the beginning of the document
\begin{document}
\bibliographystyle{IEEEtran}



\begin{center}
    \LARGE \textbf{GATE 2007 MA}\\[0.5em]
    \large \textbf{AI25BTECH11012 - UNNATHI GARIGE}
\end{center}




\begin{center}
 \textbf{Q.1-Q.20 carry one mark each.}
\end{center}
\vspace{0.25em}

\begin{enumerate}
   

\item Consider $\mathbb{R}^2$ with the usual topology. Let 
\begin{align*}
  S = \{(x, y) \in \mathbb{R}^2 : x \text{ is an integer} \}  
\end{align*}
 Then $S$ is  

\hfill{\text{GATE MA 2007}}
\begin{enumerate} 
    \item open but NOT closed
    \item both open and closed
    \item neither open nor closed
    \item closed but NOT open
\end{enumerate}


\item Suppose $X = \{ \alpha, \beta, \delta \}$. Let
\begin{align*}
\mathcal{T}_1 = \{\emptyset, X, \{\alpha\}, \{\alpha, \beta\} \} \quad \text{and} \quad \mathcal{T}_2 = \{\emptyset, X, \{\alpha\}, \{\beta, \delta\} \}.
\end{align*}
Then
\hfill{\text{GATE MA 2007}}
\begin{enumerate}
     \item both $\mathcal{T}_1 \cap \mathcal{T}_2$ and $\mathcal{T}_1 \cup \mathcal{T}_2$ are topologies
    \item neither $\mathcal{T}_1 \cap \mathcal{T}_2$ nor $\mathcal{T}_1 \cup \mathcal{T}_2$ is a topology
    \item $\mathcal{T}_1 \cup \mathcal{T}_2$ is a topology but $\mathcal{T}_1 \cap \mathcal{T}_2$ is NOT a topology
    \item $\mathcal{T}_1 \cap \mathcal{T}_2$ is a topology but $\mathcal{T}_1 \cup \mathcal{T}_2$ is NOT a topology
\end{enumerate}

\item For a positive integer $n$, let $f_n : \mathbb{R} \to \mathbb{R}$ be defined by

\begin{align*}
f_n(x) = 
\begin{cases}
\frac{1}{4n + 5} & \text{if } 0 \leq x \leq n, \\
0 & \text{otherwise}.
\end{cases}
\end{align*}

Then $\{f_n(x)\}$ converges to zero
\hfill{\text{GATE MA 2007}}
\begin{enumerate}
    \item uniformly but NOT in $L^1$ norm
    \item uniformly and also in $L^1$ norm
    \item pointwise but NOT uniformly
    \item in $L^1$ norm but NOT pointwise
\end{enumerate}


\item Let $P_1$ and $P_2$ be two projection operators on a vector space. \\ 
Then 
\hfill{\text{GATE MA 2007}}
\begin{enumerate}
    \item $P_1 + P_2$ is a projection if $P_1 P_2 = P_2 P_1 = 0$
    \item $P_1 - P_2$ is a projection if $P_1 P_2 = P_2 P_1 = 0$
    \item $P_1 + P_2$ is a projection
    \item $P_1 - P_2$ is a projection
\end{enumerate}



\item Consider the system of linear equations         \hfill{\text{GATE MA 2007}}
\begin{align*}
x + y + z &= 3 ,\\ 
x - y - z &= 4 , \\
-5y + kz &= 6
\end{align*}

Then the value of $k$ for which this system has an infinite number of solutions is

\begin{enumerate}
    \item $k = -5$
    \item $k = 0$
    \item $k = 1$
    \item $k = 3$
\end{enumerate}



\item Let             \hfill{\text{GATE MA 2007}}

A = $\myvec{
1 & 1 & 1 \\
2 & 2 & 3 \\
x & y & z}$

and let $V = \{(x, y, z) \in \mathbb{R}^3 : \det(A) = 0\}$. Then the dimension of $V$ equals:

\begin{multicols}{4}
\begin{enumerate}
    \item $0$
    \item $1$
    \item $2$
    \item $3$
\end{enumerate}
\end{multicols}


 
\item Let  \hfill{\text{GATE MA 2007}}
\begin{align*}
 S = \{0\} \cup \left\{ \frac{1}{4n + 7} : n = 1, 2, \ldots \right\}   
\end{align*} 
Then the number of analytic functions which vanish only on $S$ is:   
\begin{multicols}{4}
\begin{enumerate}
    \item infinite
    \item $0$
    \item $1$
    \item $2$
\end{enumerate}
\end{multicols}
 

\item It is given that $\sum_{n=0}^\infty a_n z^n$ converges at $z = 3 + i4$. Then the radius of convergence of the power series $\sum_{n=0}^\infty a_n z^n$ is:   \hfill{\text{GATE MA 2007}}
\begin{multicols}{4}
\begin{enumerate}
    \item $\leq 5 $
    \item $\geq 5$
    \item $< 5$
    \item $> 5$
\end{enumerate}
\end{multicols}


\item The value of $\alpha$ for which $G = \langle \alpha, 1, 3, 9, 19, 27 \rangle$ is a cyclic group under multiplication modulo $56$ is:      \hfill{\text{GATE MA 2007}}
\begin{multicols}{4}
\begin{enumerate}
    \item $5 $
    \item $15$
    \item $25$
    \item $35$
\end{enumerate}
\end{multicols}



\item Consider $\mathbb{Z}_{24}$ as the additive group modulo $24$. Then the number of elements of order $8$ in the group $\mathbb{Z}_{24}$ is:
\hfill{\text{GATE MA 2007}}
\begin{multicols}{4}
\begin{enumerate}
    \item $1$
    \item $2$
    \item $3$
    \item $4$
\end{enumerate}
\end{multicols}

 
\item Define $f : \mathbb{R}^2 \to \mathbb{R}$ by
\begin{align*}
f(x, y) =
\begin{cases}
1, & \text{if } xy = 0, \\
2, & \text{otherwise}.
\end{cases}
\end{align*}

If $S = \{(x, y) : f \text{ is continuous at the point } (x, y)\}$, then:
\hfill{\text{GATE MA 2007}}

\begin{enumerate}
    \item $S$ is open
    \item $S$ is connected
    \item $S = \emptyset$
    \item $S$ is closed
\end{enumerate}
    



\item Consider the linear programming problem
\begin{align*}
& \text{Maximize } z = c_1 x_1 + c_2 x_2, \quad c_1, c_2 > 0, \\
& \text{subject to} \\
& x_1 + x_2 \le 3 \\
& 2x_1 + 3x_2 \le 4 \\
& x_i \ge 0
\end{align*}
Then:
\hfill{\text{GATE MA 2007}}
\begin{enumerate}
\item  the primal has an optimal solution but the dual does NOT have an optimal solution
\item  both the primal and the dual have optimal solutions
\item the dual has an optimal solution but the primal does NOT have an optimal solution
\item neither the primal nor the dual have optimal solutions
\end{enumerate}


\item Let 
\begin{align*}
  f(x) = x^{10} + x - 1, x \in \mathbb{R}  
\end{align*}
and let $x_k = k$, $k=0,1,2,\dots,10$. Then the value of the divided difference
\begin{align*}
f[x_0, x_1, x_2, \dots, x_{10}]
\end{align*}
is:
\hfill{GATE MA 2007}
\begin{multicols}{4}
\begin{enumerate}
    \item $-1$
    \item $0$
    \item $1$
    \item $10$
\end{enumerate}
\end{multicols}
 

\item  Let $X, Y$ be jointly distributed random variables having the joint probability density function
\begin{align*}
f(x,y) = \begin{cases}
1, & \text{if } 0 < x + y < 1, \\
0, & \text{otherwise.}
\end{cases}
\end{align*}

Then $P(Y \ge \max(X, 1 - X))$ is
\hfill{\text{GATE MA 2007}}

\begin{multicols}{4}
\begin{enumerate}
    \item $\tfrac{1}{2}$
    \item $1$
    \item $\tfrac{1}{4}$
    \item $\tfrac{1}{6}$
\end{enumerate}
\end{multicols}


\item Let $X_1, X_2, \dots$ be a sequence of independent and identically distributed chi-square random variables, each having 4 degrees of freedom. Define
\begin{align*}
S_n = \sum_{i=1}^n X_i
\end{align*}
If $\frac{S_n}{n} \to \mu$ as $n \to \infty$, then $\mu$ =
\hfill{\text{GATE MA 2007}}
\begin{multicols}{4}
\begin{enumerate}
    \item $8$
    \item $16$
    \item $24$
    \item $32$
\end{enumerate}
\end{multicols}
 


\item  Let $\{E_n : n = 1, 2, \dots\}$ be a decreasing sequence of Lebesgue measurable sets on $\mathbb{R}$ and let $F$ be a Lebesgue measurable set on $\mathbb{R}$ such that $E_n \cap F = \emptyset$. Suppose that $F$ has Lebesgue measure 2 and the Lebesgue measure of $E_n$ equals \begin{align*}
    \dfrac{2n + 2}{3n + 1}, n = 1, 2, \dots
\end{align*}

Then the Lebesgue measure of the set $\left(\bigcap_{n=1}^{\infty} E_n\right) \cup F$ equals
\hfill{\text{GATE MA 2007}}

\begin{multicols}{4}
\begin{enumerate}
    \item $\tfrac{5}{3}$
    \item $2$
    \item $\tfrac{7}{3}$
    \item $\tfrac{8}{3}$
\end{enumerate}
\end{multicols}


\item The extremum for the variational problem
\begin{align*}
\int_0^{\frac{\pi}{8}} ((y')^2 + 2yy' - 16y^2) \, dx, \quad y(0) = 0, \; y\left(\frac{\pi}{8}\right) = 1,
\end{align*}
occurs for the curve
\hfill{\text{GATE MA 2007}}

\begin{enumerate}
 \item  $y = \sin(4x)$
 \item  $y = \sqrt{2} \sin(2x)$
 \item $y = 1 - \cos(4x)$ 
 \item $y = \dfrac{1 - \cos(8x)}{2}$
\end{enumerate}


\item Suppose $y_p(x) = x \cos(2x)$ is a particular solution of 
\begin{align*}
y'' + \alpha y = \sin(2x).
\end{align*}
Then the constant $\alpha$ equals \hfill{\text{GATE MA 2007}}
\begin{multicols}{4}
\begin{enumerate}
    \item $-4$
    \item $-2$
    \item $2$
    \item $4$
\end{enumerate}
\end{multicols}


\item If $F(s) = \tan^{-1}(s) + k$ is the Laplace transform of some function  $f(t)$, $t \ge 0$, \\then $k =$   \hfill{\text{GATE MA 2007}}
\begin{multicols}{4}
\begin{enumerate}
    \item $-\pi$
    \item $\dfrac{\pi}{2}$
    \item $0$
    \item $\dfrac{\pi}{2}$ 
\end{enumerate}
\end{multicols}
 

\item Let
\begin{align*}
S = \{(0,1,1), (1,0,1), (-1,2,1)\} \subseteq \mathbb{R}^3.
\end{align*}
Suppose $\mathbb{R}^3$ is endowed with the standard inner product . Define 
\begin{align*}
M = \{x \in \mathbb{R}^3 : \langle x, y \rangle = 0 \text{ for all } y \in S\}
\end{align*}
Then the dimension of $M$ equals \hfill{\text{GATE MA 2007}}
\begin{multicols}{4}
\begin{enumerate}
    \item $0$
    \item $1$
    \item $2$
    \item $3$ 
\end{enumerate}
\end{multicols}


\newpage
\begin{center}
    \textbf{Q.21-Q.75 carry one mark each.}
\end{center}
\vspace{1em}

\item Let $X$ be an uncountable set and let
\begin{align*}
\tau = \{U \subseteq X : X \setminus U \text{ is countable or } X \setminus U \text{ is finite} \}.
\end{align*}
Then the topological space $(X, \tau)$        \hfill{\text{GATE MA 2007}}
\begin{enumerate}
    \item is separable
    \item is Hausdorff
    \item has a countable basis
    \item has a countable basis at each point
\end{enumerate}



\item Suppose $(X, \tau)$ is a topological space. Let $\{S_\alpha\}_{\alpha \in A}$ be a sequence of subsets of $X$.\newline Then \hfill{\text{GATE MA 2007}}

\begin{enumerate}
    \item $\left(S_1\bigcup S_2\right)^" = S_1^" \bigcup S_2^" $
    \item $\left(\bigcap S_"\right)^" = \bigcap S_"$
    \item $\overline{\bigcup S_"} = \bigcup_\alpha \overline{S_"}$
    \item $\overline{S_1\bigcup S_2} = \overline{S_1} \cup \overline{S_2}$
\end{enumerate}


\item Let $(X, d)$ be a metric space. Consider the metric $\rho$ on $X$ defined by
\begin{align*}
\rho(x,y) = \min(d(x,y), 1), \quad x,y \in X.
\end{align*}
Suppose $\tau$ and $\tau_1$ are topologies on $X$ defined by $d$ and $\rho$ respectively. Then
\hfill{\text{GATE MA 2007}}

\begin{enumerate}
    \item $\tau_1$ is a proper subset of $\tau_2$
    \item $\tau_2$ is a proper subset of $\tau_1$
    \item neither $\tau_2$ nor $\tau_1$ is a subset of the other
    \item $\tau_1 = \tau_2$
\end{enumerate}



\item A basis of the vector space 
\begin{align*}
W = \{(x,y,z,w) \in \mathbb{R}^4 : x + y + z = 0, y + z + w = 0, 2x + y - z + w = 0\}
\end{align*}
is  \hfill{\text{GATE MA 2007}}
\begin{enumerate}
    \item $\{(1,1,1,1), (2,1,1,1)\}$
    \item $\{(1,-1,0,1), (0,1,-1,0)\}$
    \item $\{(1,0,-1,0), (2,1,1,1)\}$
    \item $\{(1,0,-1,0), (0,1,-1,0)\}$
\end{enumerate}


\item Consider $\mathbb{R}^3$ with the standard inner product. Let
\begin{align*}
 S = \{(1,1,1), (2,-1,2), (-1,2,1)\}   
\end{align*}

For a subset $W$ of $\mathbb{R}^3$, let $L(W)$ denote the linear span of $W$ in $\mathbb{R}^3$. Then an orthonormal set $T$ with $L(S) = L(T)$ is  \hfill{\text{GATE MA 2007}}
    \begin{multicols}{2}
    \begin{enumerate}
        \item $\left\{ \frac{1}{\sqrt{3}}(1,1,1), \frac{1}{\sqrt{6}}(1,0,-2), \frac{1}{\sqrt{2}}(1,-1,0) \right\}$ 
        \item $\{(0,0,0), (0,1,0), (0,0,1)\}$
        \item $\left\{ \frac{1}{\sqrt{3}}(1,1,1), \frac{1}{\sqrt{2}}(1,0,-1) \right\}$        \item $\left\{ \frac{1}{\sqrt{3}}(1,1,1), \frac{1}{\sqrt{2}}(1,-1,0) \right\}$
    \end{enumerate}
    \end{multicols}


\item Let $A$ be a $3 \times 3$ matrix. Suppose that the eigenvalues of $A$ are $-1, 0, 1$ with respective eigenvectors $(1, -1, 0)^T$, $(1, 1, -2)^T$ and $(1, 1, 1)^T$. \\Then $6A$ equals
\hfill{\text{GATE MA 2007}}

\begin{multicols}{2}
    \begin{enumerate}
        \item $\myvec{ -1 & 5 & 2 \\ 5 & -1 & 2 \\ 2 & 2 & -1 }$
        \item $\myvec{ 1 & 0 & 0 \\ 0 & -1 & 0 \\ 0 & 0 & 0 }$
        \item $\myvec{ 1 & 5 & 3 \\ 5 & 1 & 3 \\ 3 & 3 & 3 }$
        \item $\myvec{ -3 & 9 & 0 \\ 9 & -3 & 0 \\ 0 & 0 & 6}$ 
    \end{enumerate}
    \end{multicols}


\item Let $T:\mathbb{R}^3 \rightarrow \mathbb{R}^3$ be a linear transformation defined by
\begin{align*}
T((x, y, z)) = (x + y - z, x + y + z, y - z).
\end{align*}

Then the matrix of the linear transformation $T$ with respect to the ordered basis
\begin{align*}
    B = \{(0,1,0), (0,0,1), (1,0,0)\} of \mathbb{R}^3
\end{align*}
is \hfill{\text{GATE MA 2007}}

\begin{multicols}{2}
    \begin{enumerate}
        \item $\myvec{ 1 & 1 & 0 \\ 0 & -1 & 1 \\ 1 & 1 & 1 }$
        \item $\myvec{ 1 & 0 & 1 \\ 1 & 1 & 1 \\ -1 & 0 & 1 }$
        \item $\myvec{ 1 & -1 & 0 \\ 1 & -1 & 1 \\ 1 & 1 & 1  }$
        \item $\myvec{ 1 & 1 & 1 \\ -1 & 0 & 1 \\ 1 & -1 & 0 }$ 
    \end{enumerate}
    \end{multicols}


\item Let $Y(x) = (y_1(x), y_2(x))^T$ and let
\begin{align*}
A = \myvec{ -3 & 1 \\ k & -1 }.
\end{align*}
Further, let $S$ be the set of values of $k$ for which all the solutions of the system of equations $Y'(x) = A Y(x)$ tend to zero as $x \rightarrow \infty$. 
\\Then $S$ is given by      \hfill{\text{GATE MA 2007}}
\begin{multicols}{2}
    \begin{enumerate}
        \item $\{k : k \leq -1\}$
        \item $\{k : k \leq 3\}$
        \item $\{k : k < -1\}$
        \item  $\{k : k < 3\}$
    \end{enumerate}
    \end{multicols}


\item Let
\begin{align*}
u(x,y) = f(xe^y) + g(y^2 \cos y),
\end{align*}
where $f$ and $g$ are infinitely differentiable functions. Then the partial differential equation of minimum order satisfied by $u$ is 
\hfill{\text{GATE MA 2007}}
\begin{multicols}{2}
    \begin{enumerate}
        \item  $u_x + x u_{xx} = u_y$
        \item  $u_y + x u_{xx} = x u_y$
        \item  $u_y - x u_{xx} = u_x$ 
        \item  $u_y - x u_{xx} = x u_y$
    \end{enumerate}
    \end{multicols}


\item Let $C$ be the boundary of the triangle formed by the points $(1,0,0)$, $(0,1,0)$, $(0,0,1)$.\\
Then the value of the line integral    \hfill{\text{GATE MA 2007}}
\begin{align*}
\oint_C -2y\,dx + (3x - 4y^2)\,dy + (z^2 + 3y)\,dz
\end{align*}
is
\begin{multicols}{4}
\begin{enumerate}
    \item $0$
    \item $1$
    \item $2$
    \item $4$
\end{enumerate}
\end{multicols}


\item Let $X$ be a complete metric space and let $E \subset X$.
\\Consider the following statements:   \hfill{\text{GATE MA 2007}}

\begin{enumerate}
  \item $E$ is compact,
  \item $E$ is closed and bounded,
  \item $E$ is closed and totally bounded,
  \item Every sequence in $E$ has a subsequence converging in $E$.
\end{enumerate}


Which one of the above statements does \textbf{NOT} imply all the other statements?
\begin{multicols}{4}
\begin{enumerate}
    \item a
    \item b
    \item c
    \item d
\end{enumerate}
\end{multicols}


\item Consider the series
\begin{align*}
\sum_{n=1}^{\infty} \frac{1}{n^2} \sin(nx).
\end{align*}
Then the series   \hfill{\text{GATE MA 2007}}

\begin{enumerate}
  \item converges uniformly on $\mathbb{R}$
  \item converges pointwise but NOT uniformly on $\mathbb{R}$
  \item converges in $L^1$ norm to an integrable function on $[0, 2\pi]$ but does NOT converge uniformly on $\mathbb{R}$
  \item does NOT converge pointwise
\end{enumerate}



\item Let $f(z)$ be an analytic function. Then the value of  \hfill{\text{GATE MA 2007}}
\begin{align*}
\int_0^{2\pi} f(e^{it}) \cos(t)\,dt
\end{align*}
equals  
\begin{multicols}{4}
\begin{enumerate}
    \item $0$
    \item $2\pi f(0)$
    \item $2\pi f'(0)$
    \item $\pi f'(0)$
\end{enumerate}
\end{multicols}


\item Let $G_1$ and $G_2$ be the images of the disc $\{ z \in \mathbb{C} : |z + 1| < 1 \}$ under the transformations
\begin{align*}
w = \frac{(1 - i)z + 2}{(1 + i)z + 2} \quad \text{and} \quad w = \frac{(1 + i)z + 2}{(1 - i)z + 2}
\end{align*}
respectively. Then     

\begin{enumerate}
   
  \item $G_1 = \{w \in \mathbb{C} : \operatorname{Im}(w) < 0\}$ and $G_2 = \{w \in \mathbb{C} : \operatorname{Im}(w) > 0\}$
  \item $G_1 = \{w \in \mathbb{C} : \operatorname{Im}(w) > 0\}$ and $G_2 = \{w \in \mathbb{C} : \operatorname{Im}(w) < 0\}$
  \item $G_1 = \{w \in \mathbb{C} : |w| > 2\}$ and $G_2 = \{w \in \mathbb{C} : |w| < 2\}$
  \item $G_1 = \{w \in \mathbb{C} : |w| < 2\}$ and $G_2 = \{w \in \mathbb{C} : |w| > 2\}$

\end{enumerate}



\item Let $f(z) = 2^z - 2^{-z}$. Then the maximum value of $|f(z)|$ on the unit disc 
\begin{align*}
D = \{ z \in \mathbb{C} : |z| \leq 1 \}
\end{align*}
equals      \hfill{\text{GATE MA 2007}}
\begin{multicols}{4}
\begin{enumerate}
    \item $1$
    \item $2$
    \item $3$
    \item $4$
\end{enumerate}
\end{multicols}



\item Let        \hfill{\text{GATE MA 2007}}
\begin{align*}
f(z) = \frac{1}{z^2 - 3z + 2}
\end{align*}
Then the coefficient of $\frac{1}{z}$ in the Laurent series expansion of $f(z)$ for $|z| > 2$ is  
\begin{multicols}{4}
\begin{enumerate}
    \item $0$
    \item $1$
    \item $3$
    \item $5$
\end{enumerate}
\end{multicols}


\item Let $f: \mathbb{C} \rightarrow \mathbb{C}$ be an arbitrary analytic function satisfying $f(0) = 0$ and $f(1) = 2$. Then  \hfill{\text{GATE MA 2007}}

\begin{enumerate}
\item  there exists a sequence $\{z_n\}$ such that $|z_n| > n$ and $|f(z_n)| > n$
\item there exists a sequence $\{z_n\}$ such that $|z_n| > n$ and $|f(z_n)| < n$
\item there exists a bounded sequence $\{z_n\}$ such that $|f(z_n)| > n$
\item there exists a sequence $\{z_n\}$ such that $z_n \rightarrow 0$ and $f(z_n) \rightarrow 2$
\end{enumerate}


\item Define $f : \mathbb{C} \rightarrow \mathbb{C}$ by
\begin{align*}
f(z) =
\begin{cases}
0, & \text{if } \text{Re}(z) = 0 \text{ or } \text{Im}(z) = 0,\\
\frac{1}{z}, & \text{otherwise}.
\end{cases}
\end{align*}
Then the set of points where $f$ is analytic is    
\hfill{\text{GATE MA 2007}}

\begin{multicols}{2}
    \begin{enumerate}
        \item $\{z : \text{Re}(z) \neq 0 \text{ and } \text{Im}(z) \neq 0 \}$
        \item $\{z : \text{Re}(z) \neq 0 \}$
        \item $\{z : \text{Re}(z) \neq 0 \text{ or } \text{Im}(z) \neq 0 \}$
        \item $\{z : \text{Im}(z) \neq 0 \}$ 
    \end{enumerate}
    \end{multicols}


\item Let $U(n)$ be the set of all positive integers less than $n$ and relatively prime to $n$. Then $U(n)$ is a group under multiplication modulo $n$. For $n = 248$, the number of elements in $U(n)$ is
\hfill{\text{GATE MA 2007}}
\begin{multicols}{4}
\begin{enumerate}
    \item $60$
    \item $120$
    \item $180$
    \item $240$
\end{enumerate}
\end{multicols}



\item Let $\mathbb{R}[x]$ be the polynomial ring in $x$ with real coefficients and let $I = \langle x^2 + 1 \rangle$ be the ideal generated by the polynomial $x^2 + 1$ in $\mathbb{R}[x]$. Then        \hfill{\text{GATE MA 2007}}
\begin{enumerate}
   
\item $I$ is a maximal ideal
\item  $I$ is a prime ideal but NOT a maximal ideal
\item $I$ is NOT a prime ideal
\item  $\mathbb{R}[x]/I$ has zero divisors
\end{enumerate}



    

\item Consider $\mathbb{Z}5$ and $\mathbb{Z}{20}$ as rings modulo 5 and 20, respectively. Then the number of homomorphisms $\varphi: \mathbb{Z}5 \to \mathbb{Z}{20}$ is
\hfill{\text{GATE MA 2007}}
\begin{multicols}{4}
\begin{enumerate}
    \item $1$
    \item $2$
    \item $4$
    \item $5$
\end{enumerate}
\end{multicols}


\item Let $\mathbb{Q}$ be the field of rational numbers and consider $\mathbb{Z}_2$ as a field modulo 2. Let
    \begin{align*}
    f(x) = x^3 - 9x^2 + 9x + 3.
    \end{align*}
    Then $f(x)$ is
  \hfill{\text{GATE MA 2007}} 
\begin{enumerate}   
  \item irreducible over $\mathbb{Q}$ but reducible over $\mathbb{Z}_2$
  \item irreducible over both $\mathbb{Q}$ and $\mathbb{Z}_2$
  \item reducible over $\mathbb{Q}$ but irreducible over $\mathbb{Z}_2$
  \item educible over both $\mathbb{Q}$ and $\mathbb{Z}_2$
 \end{enumerate}




\item Let $\mathbb{Q}$ be the field of rational numbers and consider $\mathbb{Z}_2$ as a field modulo 2. Let
    \begin{align*}
    f(x) = x^3 - 9x^2 + 9x + 3.
    \end{align*}
    Then $f(x)$ is
   \hfill{\text{GATE MA 2007}}  
    \begin{enumerate}
        
       \item irreducible over $\mathbb{Q}$ but reducible over $\mathbb{Z}_2$
        \item irreducible over both $\mathbb{Q}$ and $\mathbb{Z}_2$
        \item reducible over $\mathbb{Q}$ but irreducible over $\mathbb{Z}_2$
        \item reducible over both $\mathbb{Q}$ and $\mathbb{Z}_2$
    \end{enumerate}
 

\item Consider $\mathbb{Z}_5$ as a field modulo 5 and let   \hfill{\text{GATE MA 2007}}  
   \begin{align*}
    f(x) = x^4 + 4x^3 + 4x^2 + 4x + 1.
    \end{align*}

    Then the zeros of $f(x)$ over $\mathbb{Z}_5$ are 1 and 3 with respective multiplicity
\begin{multicols}{4}
\begin{enumerate}
    \item 1 and 4
    \item 2 and 3 
    \item 2 and 2
    \item  1 and 2
\end{enumerate}
\end{multicols}
    
 
\item Consider the Hilbert space   \hfill{\text{GATE MA 2007}}  
    \begin{align*}
    \ell^2 = \left\{ x = \{x_n\};\ x_n \in \mathbb{R},\ \sum x_n^2 < \infty \right\}.
    \end{align*}

    Let
    \begin{align*}
    E = \left\{ x = \{x_n\} \mid |x_n| < \frac{1}{n} \text{ for all } n \right\}
    \end{align*}
    be a subset of $\ell^2$. Then
    
    \begin{enumerate}
        
        \item $E^\circ = \left\{ x \mid |x_n| < \frac{1}{n} \text{ for all } n \right\}$
        \item $E^\circ = E$
        \item $E^\circ = \left\{ x \mid |x_n| < \frac{1}{n} \text{ for all but finitely many } n \right\}$
        \item $E^\circ = \emptyset$
    \end{enumerate}
    

   
\item Let $X$ be a normed linear space and let $E_1, E_2 \subseteq X$. Define
    \begin{align*}
    E_1 + E_2 = \{x + y : x \in E_1, y \in E_2\}.
    \end{align*}
    Then $E_1 + E_2$ is:      \hfill{\text{GATE MA 2007}}
   
    \begin{enumerate}
       
        \item open if $E_1$ or $E_2$ is open
        \item NOT open unless both $E_1$ and $E_2$ are open
        \item closed if $E_1$ or $E_2$ is closed
        \item closed if both $E_1$ and $E_2$ are closed
    \end{enumerate}

\item For each $a \in \mathbb{R}$, consider the linear programming problem:
 
   \begin{flushleft}
    \hspace{3cm} Max. $z = x_1 + 2x_2 + 3x_3 + 4x_4$
   \end{flushleft}
    \hspace{3cm} subject to \hfill{\text{GATE MA 2007}}
    \begin{align*}
        ax_1 + 2x_2 &\leq 1 \\
        x_1 + 2x_2 + 3x_3 &\leq 2 \\
        x_1, x_2, x_3, x_4 &\geq 0
    \end{align*}
    Let $S = \{a \in \mathbb{R} : \text{the given LP problem has a basic feasible solution}\}$. Then:
    \begin{multicols}{2}
    \begin{enumerate}
        \item $S = \emptyset$
        \item $S = \mathbb{R}$
        \item $S = (0, \infty)$
        \item $S = (-\infty, 0)$
    \end{enumerate}
    \end{multicols}
    


\item Consider the linear programming problem:\\
   \hspace{3cm} Max. $z = x_1 + 5x_2 + 3x_3$
   \hspace{3cm} subject to
    \begin{align*}
        2x_1 - 3x_2 + 5x_3 &\leq 3 \\
        x_1 - x_2 &\leq 5 \\
        x_1, x_2, x_3 &\geq 0
    \end{align*}
    Then the dual of this LP problem:
     \hfill{\text{GATE MA 2007}}
    
    \begin{enumerate}
        \item has a feasible solution but does NOT have a basic feasible solution
        \item has a basic feasible solution
        \item has infinite number of feasible solutions
        \item has no feasible solution
    \end{enumerate}
    
\item  Consider a transportation problem with two warehouses and two markets. The warehouse capacities are $a_1 = 2$ and $a_2 = 4$, and the market demands are $b_1 = 3$ and $b_2 = 3$. Let $x_{ij}$ be the quantity shipped from warehouse $i$ to market $j$, and $c_{ij}$ be the corresponding unit cost. Suppose that $c_{11} = 1$, $c_{21} = 1$, and $c_{22} = 2$. Then $(x_{11}, x_{12}, x_{21}, x_{22}) = (2, 0, 1, 3)$ is optimal for every:
\hfill{\text{GATE MA 2007}}
\begin{multicols}{2}
    \begin{enumerate}
        \item $c_{12} \in [1, 2]$
        \item $c_{12} \in [0, 3]$
        \item  $c_{12} \in [1, 3]$
        \item $c_{12} \in [2, 4]$
    \end{enumerate}
    \end{multicols}


\item The smallest degree of the polynomial that interpolates the data

\begin{table}[ht]
\centering
\begin{tabular}{|c|c|c|c|c|c|c|}
\hline      
$x$     & $-2$ & $-1$ & $0$ & $1$ & $2$ & $3$ \\
\hline
$f(x)$  & $-58$ & $-21$ & $-12$ & $-13$ & $-6$ & $27$ \\
\hline

\end{tabular}
\caption{}
\label{tab:Q51}
\end{table}


    is:   \hfill{\text{GATE MA 2007}}
 \begin{multicols}{4}
\begin{enumerate}
    \item 3
    \item 4
    \item 5
    \item 6
\end{enumerate}
\end{multicols}


\item Suppose that $x_n$ is sufficiently close to 3. Which of the following iterations $x_{n+1} = g(x_n)$ will converge to the fixed point $x = 3$?
\hfill{\text{GATE MA 2007}}
\begin{multicols}{2}
    \begin{enumerate}
        \item $x_{n+1} = -16 + 6x_n + \dfrac{3}{x_n}$
        \item $x_{n+1} = \sqrt{3 + 2x_n}$
        \item $x_{n+1} = \dfrac{3}{x_n} - \dfrac{x_n}{2}$
        \item $x_{n+1} = \dfrac{x_n^2 - 3}{2}$
    \end{enumerate}
    \end{multicols}


\item Consider the quadrature formula:
    \begin{align*}
    \int_{x_1}^{x_2} f(x)\, dx \approx \dfrac{1}{2} \left[f(x_1) + f(x_2)\right],
    \end{align*}
    where $x_1$ and $x_2$ are quadrature points. Then the highest degree of the polynomial for which the above formula is exact equals: 
    \hfill{\text{GATE MA 2007}}
   \begin{multicols}{4}
\begin{enumerate}
    \item 1
    \item 2
    \item 3
    \item 4
\end{enumerate}
\end{multicols}
     

\item Let $A$, $B$ and $C$ be three events such that:
    \begin{align*}
    P(A) = 0.4,\quad P(B) = 0.5,\quad P(A \cup B) = 0.6,\quad P(C) = 0.6,\quad \text{and } P(A \cap B \cap C^c) = 0.1.
    \end{align*}
    Then $P(A \cap B \cap C) =$
    \hfill{\text{GATE MA 2007}}
\begin{multicols}{4}
\begin{enumerate}
    \item $\dfrac{1}{2}$
    \item  $\dfrac{1}{3}$
    \item $\dfrac{1}{4}$
    \item $\dfrac{1}{5}$  
\end{enumerate}
\end{multicols}
        

\item Consider two identical boxes $B_1$ and $B_2$, where the box $B_i$ ($i=1,2$) contains $i+1$ red and $5 - i + 1$ white balls. A fair die is cast. Let the number of dots shown on the top face of the die be $N$. If $N$ is even or 5, then two balls are drawn with replacement from the box $B_1$; otherwise, two balls are drawn with replacement from the box $B_2$. The probability that the two drawn balls are of different colours is: \hfill{\text{GATE MA 2007}}
\begin{multicols}{4}
\begin{enumerate}
    \item $\dfrac{7}{25}$
    \item  $\dfrac{9}{25}$
    \item $\dfrac{12}{25}$ 
    \item  $\dfrac{16}{25}$    
\end{enumerate}
\end{multicols}
        


\item Let $X_1, X_2, \ldots$ be a sequence of independent and identically distributed random variables with    
\begin{align*}
P(X_i = 1) = P(X_i = -1) = \frac{1}{2}.
\end{align*}
Suppose for the standard normal random variable $Z$, $P(-0.1 < Z \leq 0.1) = 0.08$. If $S_n = \sum_{i=1}^{n} X_i$, then

\begin{align*}
\lim P\left(\frac{S_n}{\sqrt{n}} > \frac{n}{10}\right) =
\end{align*}
\begin{multicols}{4}
\begin{enumerate}
    \item 0.42
    \item 0.46
    \item 0.5
    \item 0.54  
\end{enumerate}
\end{multicols}



\item  Let $X_1, X_2, \ldots, X_5$ be a random sample of size 5 from a population having standard normal distribution. Let 
\begin{align*}
\bar{X} = \frac{1}{5} \sum_{i=1}^{5} X_i \quad \text{and} \quad T = \sum_{i=1}^{5} (X_i - \bar{X})^2.
\end{align*}
Then $E(T^2 \bar{X}^2) =$ \hfill{\text{GATE MA 2007}}
\begin{multicols}{4}
\begin{enumerate}
    \item 3
    \item 3.6
    \item 4.8
    \item 5.2
\end{enumerate}
\end{multicols}



\item Let $x_1 = 3.5$, $x_2 = 7.5$ and $x_3 = 5.2$ be observed values of a random sample of size three from a population having uniform distribution over the interval $(\theta, \theta + 5)$, where $\theta \in (0, \infty)$ is unknown and is to be estimated. Then which of the following is NOT a maximum likelihood estimate of $\theta$?
\hfill{\text{GATE MA 2007}}
\begin{multicols}{4}
\begin{enumerate}
    \item 2.4
    \item 2.7
    \item 3
    \item 3.3
\end{enumerate}
\end{multicols}


\item The value of 
\begin{align*}
\int_0^1 \int_y^1 x^2 e^{x^2} \, dx \, dy
\end{align*}
equals
\hfill{\text{GATE MA 2007}}
\begin{multicols}{4}
\begin{enumerate}
    \item $\frac{1}{4}$
    \item $\frac{1}{3}$
    \item $\frac{1}{2}$
    \item 1
\end{enumerate}
\end{multicols}



\item 
\begin{align*}
\lim_{n \to \infty} \left[ (n+1) \int_0^1 x^n \ln(1+x) \, dx \right] =
\end{align*}
\hfill{\text{GATE MA 2007}}
\begin{multicols}{4}
\begin{enumerate}
    \item 0
    \item $\ln 2$
    \item $\ln 3$
    \item $\infty$
\end{enumerate}
\end{multicols}



\item Consider the function $f \colon \mathbb{R} \to \mathbb{R}$ defined by
\begin{align*}
f(x) = 
\begin{cases}
x^4, & \text{if } x \text{ is rational}, \\
2x^4 - 1, & \text{if } x \text{ is irrational}.
\end{cases}
\end{align*}
Let $S$ be the set of points where $f$ is continuous. Then
\hfill{\text{GATE MA 2007}}
\begin{multicols}{4}
\begin{enumerate}
    \item $S = \{1\}$
    \item $S = \{-1\}$
    \item $S = \{-1, 1\}$
    \item $S = \emptyset$
\end{enumerate}
\end{multicols}


\item For a positive real number $p$, let $\{f_n: n=1,2,\dots\}$ be a sequence of functions defined on $[0,1]$ by
\begin{align*}
f_n(x) =
\begin{cases}
n^{p+1} x, & 0 \leq x \leq \frac{1}{n} \\
\frac{1}{n^p}, & \frac{1}{n} < x \leq 1.
\end{cases}
\end{align*}
Let $f(x) = \lim\limits_{n \to \infty} f_n(x),\ x \in [0,1]$. Then, on $[0,1]$,
\hfill{\text{GATE MA 2007}}
\begin{enumerate}
    
\item $f$ is Riemann integrable \hspace{2cm}
\item the improper integral $\int_0^1 f(x) dx$ converges for $p \geq 1$ 
\item the improper integral $\int_0^1 f(x) dx$ converges for $p < 1$ \hspace{2cm}
\item $f_n$ converges uniformly
\end{enumerate}


\item Which of the following inequality is NOT true for $x \in \left[ \frac{1}{4}, \frac{3}{4} \right]$
\hfill{\text{GATE MA 2007}}
\begin{multicols}{2}
\begin{enumerate}
\item  $e^{-x} > \sum_{j=0}^{\infty} \frac{(-x)^j}{j!}$ \hspace{2cm}
\item $e^{-x} < \sum_{j=0}^{\infty} \frac{(-x)^j}{j!}$ \\
\item $e^{-x} = \sum_{j=0}^{\infty} \frac{(-x)^j}{j!}$ \hspace{2cm}
\item $e^{-x} > \sum_{j=0}^{10} \frac{(-x)^j}{j!}$
\end{enumerate}
\end{multicols}

\item Let $u(x,y)$ be the solution to the Cauchy problem
\begin{align*}
x u_x + u_y = 1, \quad u(x,0) = 2 \ln(x),\ x > 1.
\end{align*}
Then $u(e,1) =$
\hfill{\text{GATE MA 2007}}
\begin{multicols}{4}
\begin{enumerate}
    \item -1
    \item 0
    \item 1
    \item $e$
\end{enumerate}
\end{multicols}



\item Suppose
\begin{align*}
y(x) = \lambda \int_0^{2\pi} y(t) \sin(x + t)\, dt,\ x \in [0,2\pi]
\end{align*}
has eigenvalues $\lambda = \frac{1}{\pi}$ and $\lambda = -\frac{1}{\pi}$ with corresponding eigenfunctions \\
$y_1(x) = \sin(x) + \cos(x)$ and $y_2(x) = \sin(x) - \cos(x)$, respectively. Then the integral equation
\begin{align*}
y(x) = f(x) + \frac{1}{\pi} \int_0^{2\pi} y(t) \sin(x + t)\, dt,\ x \in [0,2\pi]
\end{align*}
has a solution when $f(x) =$
\hfill{\text{GATE MA 2007}}
\begin{multicols}{4}
\begin{enumerate}
    \item 1
    \item $\cos(x)$
    \item $\sin(x)$
    \item $1 + \sin(x) + \cos(x)$
\end{enumerate}
\end{multicols}



\item Consider the Neumann problem
\begin{align*}
u_{xx} + u_{yy} = 0,\quad 0 < x < \pi,\ -1 < y < 1,
\end{align*}
\begin{align*}
u_y(0, y) = u_y(\pi, y) = 0,
\end{align*}
\begin{align*}
u_y(x, -1) = 0,\quad u_y(x, 1) = \alpha + \beta \sin(x).
\end{align*}
The problem admits solution for
\hfill{\text{GATE MA 2007}}
\begin{multicols}{2}
    \begin{enumerate}
        \item $\alpha = 0,\ \beta = 1$
        \item $\alpha = -1,\ \beta = \dfrac{\pi}{2}$ 
        \item $\alpha = 1,\ \beta = \dfrac{\pi}{2}$
        \item $\alpha = 1,\ \beta = -\pi$
    \end{enumerate}
    \end{multicols}


\item The functional
\begin{align*}
\int_0^1 (1+x)(y')^2 \, dx,\quad y(0) = 0,\ y(1) = 1,
\end{align*}
possesses
\hfill{\text{GATE MA 2007}}

    \begin{enumerate}
        \item strong maxima
        \item strong minima
        \item weak maxima but NOT a strong maxima
        \item weak minima but NOT a strong minima   
    \end{enumerate}
  

\item The value of $\alpha$ for which the integral equation
\begin{align*}
u(x) = \alpha \int_0^1 e^{xt} u(t)\,dt,
\end{align*}
has a non-trivial solution is
\hfill{\text{GATE MA 2007}}
\begin{multicols}{4}
\begin{enumerate}
    \item -2
    \item -1
    \item 1
    \item 2
\end{enumerate}
\end{multicols}



\item Let $P_n(x)$ be the Legendre polynomial of degree $n$ and let
\begin{align*}
P_{n+1}(0) = -\frac{m}{m+1} P_{n-1}(0), \quad m = 1, 2, \ldots
\end{align*}
If $P_2(0) = -\dfrac{5}{16}$ then $\int_{-1}^1 \left[P_2^2(x)\right] dx =$
\hfill{\text{GATE MA 2007}}
\begin{multicols}{4}
\begin{enumerate}
    \item $\dfrac{2}{13}$
    \item $\dfrac{2}{9}$
    \item $\dfrac{5}{16}$ 
    \item $\dfrac{2}{5}$
\end{enumerate}
\end{multicols}



\item For which of the following pair of functions $y_1(x)$ and $y_2(x)$, continuous functions $p(x)$ and $q(x)$ can be determined on $[-1, 1]$ such that $y_1(x)$ and $y_2(x)$ give two linearly independent solutions of \hfill{\text{GATE MA 2007}}
\begin{align*}
y'' + p(x)y' + q(x)y = 0,\quad x \in [-1, 1].
\end{align*}
\begin{multicols}{2}
    \begin{enumerate}
        \item $y_1(x) = x \sin(x),\ y_2(x) = \cos(x)$
        \item $y_1(x) = x e^x,\ y_2(x) = \sin(x)$ 
        \item $y_1(x) = e^{-x},\ y_2(x) = e^{-1}$
        \item $y_1(x) = x^2,\ y_2(x) = \cos(x)$
    \end{enumerate}
    \end{multicols}
 



\item Let $J_0(s)$ and $J_1(s)$ be the Bessel functions of the first kind of orders zero and one, respectively. If
\begin{align*}
\mathcal{L}(J_0)(s) = \frac{1}{\sqrt{s^2 + 1}},
\end{align*}
then $\mathcal{L}(J_1)(s) =$
\hfill{\text{GATE MA 2007}}
\begin{multicols}{2}
    \begin{enumerate}
        \item $\dfrac{s}{\sqrt{s^2 + 1}}$
        \item $\dfrac{1}{\sqrt{s^2 + 1}}$
        \item $1 - \dfrac{1}{\sqrt{s^2 + 1}}$
        \item $\dfrac{1}{\sqrt{s^2 + 1}} - 1$
    \end{enumerate}
    \end{multicols}
 
\bigskip

\begin{center}
    \textbf{Common Data Questions}
\end{center}

\textbf{Common Data for Questions 71, 72, 73:}

Let $P[0,1] = \{p : p \text{ is a polynomial function on } [0,1]\}$. For $p \in P[0,1]$, define
\begin{align*}
\|p\| = \sup \{|p(x)| : 0 \leq x \leq 1\}.
\end{align*}
Consider the map $T : P[0,1] \rightarrow P[0,1]$ defined by
\begin{align*}
(Tp)(x) = \frac{d}{dx} \left(p(x)\right).
\end{align*}
Then $P[0,1]$ is a normed linear space and $T$ is a linear map. The map $T$ is said to be closed if the set $G = \{(p, Tp) : p \in P[0,1]\}$ is a closed subset of $P[0,1] \times P[0,1]$.

\vspace{1em}

\item The linear map $T$ is
\hfill{\text{GATE MA 2007}}
\begin{multicols}{2}
    \begin{enumerate}
        \item one to one and onto
        \item one to one but NOT onto
        \item onto but NOT one to one
        \item neither one to one nor onto
    \end{enumerate}
    \end{multicols}




\item The normed linear space $P[0,1]$ is
\hfill{\text{GATE MA 2007}}

\begin{enumerate}
   
    \item a finite dimensional normed linear space which is NOT a Banach space
    \item a finite dimensional Banach space
    \item an infinite dimensional normed linear space which is NOT a Banach space
    \item an infinite dimensional Banach space
\end{enumerate}



\item The map $T$ is
\hfill{\text{GATE MA 2007}}
\begin{multicols}{2}
    \begin{enumerate}
        \item closed and continuous
        \item neither continuous nor closed
        \item continuous but NOT closed
        \item closed but NOT continuous
    \end{enumerate}
    \end{multicols}

\bigskip

\textbf{Common Data for Questions 74, 75:}
\bigskip

Let $X$ and $Y$ be jointly distributed random variables such that the conditional distribution of $Y$, given $X = x$, is uniform on the interval $(x - 1, x + 1)$. Suppose $\mathbb{E}(X) = 1$ and $\text{Var}(X) = \dfrac{5}{3}$.



\item The mean of the random variable $Y$ is
\hfill{\text{GATE MA 2007}}
\begin{multicols}{4}
    \begin{enumerate}
        \item  $\dfrac{1}{2}$
        \item $1$
        \item $\dfrac{3}{2}$
        \item 2
    \end{enumerate}
    \end{multicols}
\newpage

\item The variance of the random variable $Y$ is
\hfill{\text{GATE MA 2007}}
\begin{multicols}{4}
    \begin{enumerate}
        \item $\dfrac{1}{2}$
        \item $\dfrac{2}{3}$
        \item 1
        \item 2
    \end{enumerate}
    \end{multicols}


\bigskip
\begin{center}
 \textbf{Linked Answer Questions: Q.76 to Q.85 carry two marks each.}   
\end{center}
\vspace{1em}

\textbf{Statement for Linked Answer Questions 76 \& 77:}  

Suppose the equation  
\begin{align*}
x^2 y'' - x y' + (1 + x^2) y = 0
\end{align*}
has a solution of the form  
\begin{align*}
y = x^r \sum_{n=0}^{\infty} c_n x^n,\quad c_0 \neq 0.
\end{align*}

\item The indicial equation for $r$ is
\hfill{\text{GATE MA 2007}}
\begin{multicols}{2}
    \begin{enumerate}
        \item  $r^2 - 1 = 0$
        \item $(r - 1)^2 = 0$
        \item $(r + 1)^2 = 0$ 
        \item $r^2 + 1 = 0$
    \end{enumerate}
    \end{multicols}




\item For $n \geq 2$, the coefficients $c_n$ will satisfy the relation
\hfill{\text{GATE MA 2007}}
\begin{multicols}{2}
    \begin{enumerate}
        \item  $n^2 c_n - c_{n-2} = 0$
        \item $c_n - n^2 c_{n-2} = 0$
        \item $c_n - n^2 c_{n-2} = 0$
        \item $c_n + n^2 c_{n-2} = 0$
    \end{enumerate}
    \end{multicols}

 

\textbf{Statement for Linked Answer Questions 78 \& 79:} 
\newline \vspace{1em}
A particle of mass $m$ slides down without friction along a curve $z = 1 + \dfrac{x^2}{2}$ in the $xz$-plane under the action of constant gravity. Suppose the $z$-axis points vertically upwards. Let $\dot{x}$ and $\ddot{x}$ denote $\dfrac{dx}{dt}$ and $\dfrac{d^2x}{dt^2}$ respectively.

\item The Lagrangian of the motion is      \hfill{\text{GATE MA 2007}}     
\begin{multicols}{2}
    \begin{enumerate}
        \item $\dfrac{1}{2} m \dot{x}^2 (1 + x^2) - mg \left(1 + \dfrac{x^2}{2} \right)$
        \item $\dfrac{1}{2} m \dot{x}^2 (1 + x^2) + mg \left(1 + \dfrac{x^2}{2} \right)$
        \item $\dfrac{1}{2} m x^2 \dot{x}^2 - mg \left(1 + \dfrac{x^2}{2} \right)$
        \item $\dfrac{1}{2} m \dot{x}^2 (1 - x^2) - mg \left(1 + \dfrac{x^2}{2} \right)$
    \end{enumerate}
    \end{multicols}

    
\item The Lagrangian equation of motion is    \hfill{\text{GATE MA 2007}}
\begin{enumerate}
    

    \item $\ddot{x}(1 + x^2) = -x(g + \dot{x}^2)$
    \item $\ddot{x}(1 + x^2) = x(g - \dot{x}^2)$
   
    \item $\ddot{x} = -g x$ 
    \item $\ddot{x}(1 - x^2) = -x(g - \dot{x}^2)$
    
\end{enumerate}

\newpage

\textbf{Statement for Linked Answer Questions 80 \& 81:}

Let $u(x,t)$ be the solution of the one dimensional wave equation
\begin{align*}
u_{tt} = 4u_{xx}, \quad -\infty < x < \infty,\ t > 0,
\end{align*}
\begin{align*}
u(x,0) =
\begin{cases}
16 - x^2, & |x| \leq 4, \\
0, & \text{otherwise},
\end{cases}
\quad \text{and} \quad
u_t(x,0) =
\begin{cases}
1, & |x| \leq 2, \\
0, & \text{otherwise}.
\end{cases}
\end{align*}

\item For $1 < t < 3$, $u(2,t) =$   \hfill{\text{GATE MA 2007}}
\begin{enumerate}
    \item $\left[16 - (2 - 2t)^2\right]^+ + \dfrac{1}{2}\left[1 - \min\{1, t - 1\}\right]$
    \item $\left[32 - (2 - 2t)^2 - (2 + 2t)^2\right]^+ + t$
    \item $\left[32 - (2 - 2t)^2 - (2 + 2t)^2\right]^+ + 1$
    \item $\left[16 - (2 - 2t)^2\right]^+ + \dfrac{1}{2}\left[1 - \max\{1, t - 1\}\right]$
\end{enumerate}



\item The value of $u(2,2)$         \hfill{\text{GATE MA 2007}}
\begin{enumerate}
    \item equals $15$
    \item equals $16$
    \item equals $0$
    \item does NOT exist
\end{enumerate}

\bigskip

\textbf{Statement for Linked Answer Questions 82 \& 83:}
\bigskip
Suppose $E = \{(x, y): xy \ne 0\}$. Let $f: \mathbb{R}^2 \rightarrow \mathbb{R}$ be defined by  
\begin{align*}
f(x, y) =
\begin{cases}
0, & \text{if } xy = 0, \\
y \sin\left(\frac{1}{x}\right) + x \sin\left(\frac{1}{y}\right), & \text{otherwise}.
\end{cases}
\end{align*}

Let $S_1$ be the set of points in $\mathbb{R}^2$ where $f_x$ exists and $S_2$ be the set of points in $\mathbb{R}^2$ where $f_y$ exists. Also, let $E_1$ be the set of points where $f_x$ is continuous and $E_2$ be the set of points where $f_y$ is continuous.

\item $S_1$ and $S_2$ are given by      \hfill{\text{GATE MA 2007}}
\begin{enumerate}
    \item $S_1 = E \cup \{(x, y): y = 0\},\quad S_2 = E \cup \{(x, y): x = 0\}$
    \item $S_1 = E \cup \{(x, y): x = 0\},\quad S_2 = E \cup \{(x, y): y = 0\}$
    \item $S_1 = S_2 = \mathbb{R}^2$
    \item $S_1 = S_2 = E \cup \{(0, 0)\}$
\end{enumerate}




\item $E_1$ and $E_2$ are given by      \hfill{\text{GATE MA 2007}}

\begin{enumerate}
    \item $E_1 = S_1,\quad E_2 = S_1 \cap S_2$
    \item $E_1 = S_1 \cap S_2 \setminus \{(0,0)\},\quad E_2 = S_1$
    \item $E_1 = S_2,\quad E_2 = S_1$
    \item $E_1 = S_2,\quad E_2 = S_2$
\end{enumerate}

\newpage

\textbf{Statement for Linked Answer Questions 84 \& 85:}

Let
\begin{align*}
A = \myvec{
3 & 0 & 0 \\
0 & 6 & 2 \\
0 & 2 & 6
}
\end{align*}
and let $\lambda_1 \geq \lambda_2 \geq \lambda_3$ be the eigenvalues of $A$.

\item The triple $(\lambda_1, \lambda_2, \lambda_3)$ equals    \hfill{\text{GATE MA 2007}}
\begin{multicols}{2}
    \begin{enumerate}
        \item $(9, 4, 2)$
        \item $(8, 4, 3)$
        \item $(9, 3, 3)$
        \item $(7, 5, 3)$
    \end{enumerate}
    \end{multicols}




\item The matrix $P$ such that   \hfill{\text{GATE MA 2007}}
\[
	P^{-1} A P = \myvec{
\lambda_1 & 0 & 0 \\
0 & \lambda_2 & 0 \\
0 & 0 & \lambda_3
}
 \]
is 
\begin{multicols}{2}
    \begin{enumerate}
        \item  $\myvec{
    \dfrac{1}{\sqrt{3}} & 0 & \dfrac{-2}{\sqrt{6}} \\
    \dfrac{1}{\sqrt{3}} & \dfrac{1}{\sqrt{2}} & \dfrac{1}{\sqrt{6}} \\
    \dfrac{1}{\sqrt{3}} & \dfrac{-1}{\sqrt{2}} & \dfrac{1}{\sqrt{6}}
    }$
        \item $\myvec{
    \dfrac{1}{\sqrt{3}} & \dfrac{-2}{\sqrt{6}} & 0 \\
    \dfrac{1}{\sqrt{3}} & \dfrac{1}{\sqrt{6}} & \dfrac{1}{\sqrt{2}} \\
    \dfrac{1}{\sqrt{3}} & \dfrac{1}{\sqrt{6}} & \dfrac{-1}{\sqrt{2}}
    }$
        \item $\myvec{
    0 & 0 & 1 \\
    \dfrac{1}{\sqrt{2}} & \dfrac{1}{\sqrt{2}} & 0 \\
    \dfrac{1}{\sqrt{2}} & \dfrac{-1}{\sqrt{2}} & 0
   }$
        \item $\myvec{
    0 & 1 & 0 \\
    \dfrac{1}{\sqrt{2}} & 0 & \dfrac{1}{\sqrt{2}} \\
    \dfrac{1}{\sqrt{2}} & 0 & \dfrac{-1}{\sqrt{2}}
    }$
    \end{enumerate}
    \end{multicols}




\vspace{5em}
\begin{center}
    \textbf{END OF THE QUESTION PAPER}
\end{center}

\end{enumerate}

\end{document}
