\documentclass[12pt]{article}
\usepackage{graphicx} % Required for inserting images
\usepackage[margin=2cm]{geometry}
\usepackage{amsmath}
\usepackage{float}
\usepackage{enumitem}
\usepackage{multicol}
\usepackage{gvv-book}
\usepackage{gvv}

\title{Chemistry 2022}
\author{Aditya Appana }
\date{August 2025}

\begin{document}

\maketitle
\section*{Q.1 - Q.5 Carry ONE mark each.}

\begin{enumerate}
    \item Inhaling the smoke from a burning \rule{2cm}{0.15mm} could \rule{2cm}{0.15mm} you quickly.
    \begin{enumerate}
        \item tier/tier
        \item tire/tyre
        \item tyre/tire
        \item tyre/tier
    \end{enumerate}

    \item A sphere of radius $r$ cm is packed in a box of cubical shape. \\  
    
         What should be the minimum volume (in $cm^3$ ) of the box that can enclose the
        sphere?

        \begin{enumerate}
            \item ${\frac{r^3}{8}}$
            \item $r^3$
            \item $2r^3$
            \item $8r^3$
         \end{enumerate}

    \item Pipes P and Q can fill a storage tank in full with water in 10 and 6 minutes,
respectively. Pipe R draws the water out from the storage tank at a rate of 34
litres per minute. P, Q and R operate at a constant rate. \\

If it takes one hour to completely empty a full storage tank with all the pipes
operating simultaneously, what is the capacity of the storage tank (in litres)?

\begin{enumerate}
\item 26.8
\item 60.0
\item 120.0
\item 127.5
\end{enumerate}

    \item Six persons P, Q, R, S, T and U are sitting around a circular table facing the
        center not necessarily in the same order. Consider the following statements: 
\begin{itemize}
\item P sits next to S and T.
\item Q sits diametrically opposite to P.
\item The shortest distance between S and R is equal to the shortest distance
    between T and U.
\end{itemize}

Based on the above statements, Q is a neighbor of

\begin{enumerate}
\item U and S
\item R and T
\item R and U
\item P and S
\end{enumerate}

\item A building has several rooms and doors as shown in the top view of the building
given below. The doors are closed initially. \\

What is the minimum number of doors that need to be opened in order to go
from the point P to the point Q?

\begin{figure}[H]
\centering
\includegraphics[scale=0.6]{Figs/Screenshot 2025-08-16 at 11.04.36 AM.png}
\caption{}
\end{figure}

\begin{enumerate}
\item 4
\item 3
\item 2
\item 1
\end{enumerate}
\end{enumerate}
\section*{Q. 6 – Q. 10 Carry TWO marks each.}

\begin{enumerate}
\setcounter{enumi}{5}

\item Rice, a versatile and inexpensive source of carbohydrate, is a critical component
of diet worldwide. Climate change, causing extreme weather, poses a threat to
sustained availability of rice. Scientists are working on developing Green Super
Rice (GSR), which is resilient under extreme weather conditions yet gives higher
yields sustainably \\

Which one of the following is the CORRECT logical inference based on the
information given in the above passage?



\begin{enumerate}
\item GSR is an alternative to regular rice, but it grows only in an extreme weather
\item GSR may be used in future in response to adverse effects of climate change
\item GSR grows in an extreme weather, but the quantity of produce is lesser than
regular rice
\item Regular rice will continue to provide good yields even in extreme weather
\end{enumerate}


\item A game consists of spinning an arrow around a stationary disk as shown below.
When the arrow comes to rest, there are eight equally likely outcomes. It could
come to rest in any one of the sectors numbered 1, 2, 3, 4, 5, 6, 7 or 8 as shown. \\

Two such disks are used in a game where their arrows are independently spun.\\ 

What is the probability that the sum of the numbers on the resulting sectors upon
spinning the two disks is equal to 8 after the arrows come to rest?

\begin{figure}[H]
\centering
\includegraphics[scale=0.6]{Figs/Screenshot 2025-08-16 at 11.04.41 AM.png}
\caption{}
\end{figure}




\begin{enumerate}
\item${\frac{1}{16}}$
\item${\frac{5}{64}}$
\item ${\frac{3}{32}}$
\item ${\frac{7}{64}}$
\end{enumerate}

\item Consider the following inequalities.
    \begin{enumerate}[label=(\roman*)]
        \item $3p-q<4$ 
        \item $3q-p<12$
    \end{enumerate}

    Which one of the following expressions below satisfies the above two
inequalities?
\begin{enumerate}
    

\item $p+q<8$
\item$p+q=8$
\item $8$ {$\le$} $p+q<16$
\item $p+q$ $\ge$ $16$
\end{enumerate}



\item Given below are three statements and four conclusions drawn based on the
statements. \\

Statement 1: Some engineers are writers.\\
Statement 2: No writer is an actor.\\
Statement 3: All actors are engineers.\\

Conclusion I: Some writers are engineers\\
Conclusion II: All engineers are actors.\\
Conclusion III: No actor is a writer.\\
Conclusion IV: Some actors are writers.\\

Which one of the following options can be logically inferred?

\begin{enumerate}
    

\item Only conclusion I is correc
\item Only conclusion II and conclusion III are correct 
\item Only conclusion I and conclusion III are correct
\item Either conclusion III or conclusion IV is correct
\end{enumerate}


\item Which one of the following sets of pieces can be assembled to form a square
with a single round hole near the center? Pieces cannot overlap.

\begin{enumerate}
    

\item \begin{figure}[H]
\centering
\includegraphics[scale=0.6]{Figs/Screenshot 2025-08-16 at 11.04.47 AM.png}
\caption{}
\end{figure}
\item \begin{figure}[H]
\centering
\includegraphics[scale=0.6]{Figs/Screenshot 2025-08-16 at 11.04.51 AM.png}
\caption{}
\end{figure}
\item \begin{figure}[H]
\centering
\includegraphics[scale=0.6]{Figs/Screenshot 2025-08-16 at 11.04.54 AM.png}
\caption{}
\end{figure}
\item \begin{figure}[H]
\centering
\includegraphics[scale=0.6]{Figs/Screenshot 2025-08-16 at 11.04.58 AM.png}
\caption{}
\end{figure}
\end{enumerate}




\end{enumerate}


\section*{Q.11 – Q.35 Carry ONE mark Each}

\begin{enumerate}
\setcounter{enumi}{10}



\item The major product M formed in the following reaction is
\begin{figure}[H]
\centering
\includegraphics[scale=0.6]{Figs/Screenshot 2025-08-16 at 11.05.03 AM.png}
\caption{}
\end{figure}

\begin{enumerate}
    \item \begin{figure}[H]
\centering
\includegraphics[scale=0.6]{Figs/Screenshot 2025-08-16 at 11.05.06 AM.png}
\caption{}
\end{figure}

\item  \begin{figure}[H]
\centering
\includegraphics[scale=0.6]{Figs/Screenshot 2025-08-16 at 11.05.10 AM.png}
\caption{}
\end{figure}


\item  \begin{figure}[H]
\centering
\includegraphics[scale=0.6]{Figs/Screenshot 2025-08-16 at 11.05.23 AM.png}
\caption{}
\end{figure}

\item  
\begin{figure}[H]
\centering
\includegraphics[scale=0.6]{Figs/Screenshot 2025-08-16 at 11.05.26 AM.png}
\caption{}
\end{figure}
\end{enumerate}

%Question 11 stops here

\item The starting material Y in the following reaction is
\begin{figure}[H]
\centering
\includegraphics[scale=0.6]{Figs/Screenshot 2025-08-16 at 11.05.30 AM.png}
\caption{}
\end{figure}

\begin{enumerate} 
    \item \begin{figure}[H]
\centering
\includegraphics[scale=0.6]{Figs/Screenshot 2025-08-16 at 11.05.34 AM.png}
\caption{}
\end{figure}

\item  \begin{figure}[H]
\centering
\includegraphics[scale=0.6]{Figs/Screenshot 2025-08-16 at 11.05.37 AM.png}
\caption{}
\end{figure}


\item  \begin{figure}[H]
\centering
\includegraphics[scale=0.6]{Figs/Screenshot 2025-08-16 at 11.05.44 AM.png}
\caption{}
\end{figure}

\item  
\begin{figure}[H]
\centering
\includegraphics[scale=0.6]{Figs/Screenshot 2025-08-16 at 11.05.46 AM.png}
\caption{}
\end{figure}
\end{enumerate}

%Question 12 ends here

\item The major product in the given reaction is Q. The mass spectrum of Q shows
\{[M] = molecular ion peak\}
\begin{figure}[H]
\centering
\includegraphics[scale=0.6]{Figs/Screenshot 2025-08-16 at 11.05.52 AM.png}
\caption{}
\end{figure}

\begin{enumerate}

    \item {[M], [M+2] and [M+4] with relative intensity of 1:2:1 }
    \item {[M] and [M+2] with relative intensity of 1:1}
    \item {[M], [M+2] and [M+4] with relative intensity of 1:3:1}
    \item {[M] and [M+2] with relative intensity of 2:1}
    
\end{enumerate}

\item A tripeptide on treatment with PhNCS (pH = 8.0) followed by heating with dilute
HCl afforded a cyclic compound M and a dipeptide. The dipeptide on treatment
with PhNCS (pH = 8.0) followed by heating with dilute HCl afforded a cyclic
compound N and an acyclic compound O. The CORRECT sequence (from N- to
C-terminus) of the tripeptide is

\begin{figure}[H]
\centering
\includegraphics[scale=0.6]{Figs/Screenshot 2025-08-16 at 11.05.55 AM.png}
\caption{}
\end{figure}


\begin{enumerate}
    \item glycine-phenylalanine-valine
    \item valine-phenylalanine-glycine
    \item glycine-tyrosine-valine
    \item glycine-phenylalanine-alanine
\end{enumerate}



\item The major product M in the following reaction is
\begin{figure}[H]
\centering
\includegraphics[scale=0.6]{Figs/Screenshot 2025-08-16 at 11.06.01 AM.png}
\caption{}
\end{figure}

\begin{enumerate} 
    \item \begin{figure}[H]
\centering
\includegraphics[scale=0.6]{Figs/Screenshot 2025-08-16 at 11.06.04 AM.png}
\caption{}
\end{figure}

\item  \begin{figure}[H]
\centering
\includegraphics[scale=0.6]{Figs/Screenshot 2025-08-16 at 11.06.07 AM.png}
\caption{}
\end{figure}


\item  \begin{figure}[H]
\centering
\includegraphics[scale=0.6]{Figs/Screenshot 2025-08-16 at 11.06.15 AM.png}
\caption{}
\end{figure}

\item  
\begin{figure}[H]
\centering
\includegraphics[scale=0.6]{Figs/Screenshot 2025-08-16 at 11.06.18 AM.png}
\caption{}
\end{figure}
\end{enumerate}

%Question 14 ends here




\item The major product T formed in the following reaction is
\begin{figure}[H]
\centering
\includegraphics[scale=0.6]{Figs/Screenshot 2025-08-16 at 11.06.22 AM.png}
\caption{}
\end{figure}

\begin{enumerate} 
    \item \begin{figure}[H]
\centering
\includegraphics[scale=0.6]{Figs/Screenshot 2025-08-16 at 11.06.25 AM.png}
\caption{}
\end{figure}

\item  \begin{figure}[H]
\centering
\includegraphics[scale=0.6]{Figs/Screenshot 2025-08-16 at 11.06.28 AM.png}
\caption{}
\end{figure}


\item  \begin{figure}[H]
\centering
\includegraphics[scale=0.6]{Figs/Screenshot 2025-08-16 at 11.06.32 AM.png}
\caption{}
\end{figure}

\item  
\begin{figure}[H]
\centering
\includegraphics[scale=0.6]{Figs/Screenshot 2025-08-16 at 11.06.36 AM.png}
\caption{}
\end{figure}
\end{enumerate}

\item In differential thermal analysis (DTA)

\begin{enumerate}
    \item the temperature differences between the sample and reference are measured as a
function of temperature
    \item the differences in heat flow into the reference and sample are measured as a function
of temperature
    \item the change in the mass of the sample is measured as a function of temperature
    \item the glass transition is observed as a sharp peak
\end{enumerate}

\item The $v_{o-o}$ resonance Raman stretching frequency $(cm^{-1}$) of the coordinated
dioxygen in \textit{oxy-hemoglobin} and \textit{oxy-hemocyanin} appears, respectively, nearly at

\begin{enumerate}
    \item 1136 and 744
    \item 1550 and 744
    \item 744 and 1136
    \item 744 and 1550
\end{enumerate}

\item The number of metal-metal bond(s), with $\sigma , \pi, \delta$ character, present in
$[Mo_2(CH_3CO_2)_4]$ complex is(are), respectively,

\begin{enumerate}
    \item 1, 2, 1
    \item 1, 2, 0
    \item 1, 1, 0
    \item 1, 1, 1
\end{enumerate}

\item 1$s_A$ and 1$s_B$ are the normalized eigenfunctions of two hydrogen atoms $H_A $ and $H_B$,
respectively. If S = $\langle  $s_{A}$ $|$ $s_{B}$ \rangle$, the option that is ALWAYS CORRECT is

\begin{enumerate}
    \item S = 1
    \item S = 0
    \item S = imaginary constant
    \item $0 \leq S \leq 1$
\end{enumerate}


\item The pure vibrational spectrum of a hypothetical diatomic molecule shows three
peaks with the following intensity at three different temperatures.


\begin{table}[H]
    \centering
    \begin{tabular}{|c|c|c|c|}
    \hline
         Peak&  \multicolumn{3}{c|}{Intensity(Arbitrary Unit)}\\ \hline
         &  300K&  600K& 900K\\ \hline
         I&  1.0&  1.0& 1.0\\ \hline
         II&  0.1&  0.1& 0.1\\ \hline
         III&  0.02&  0.04& 0.06\\ \hline
    \end{tabular}
    \caption{Caption}
    \label{tab:placeholder}
\end{table}


The CORRECT statement is

\begin{enumerate}
    \item Peak I appears at the lowest energy
    \item Peak II appears at the lowest energy
    \item Peak III appears at the lowest energy
    \item Peak I appears at the highest energy
\end{enumerate}

\item The point group of $SF_6$ is

\begin{enumerate} \begin{multicols}{4}
    \item $D_{6h}$
    \item $O_{h}$
    \item $D_{6d}$
    \item $C_{6v}$
    \end{multicols}
\end{enumerate}



\item A point originally at (1, 3, 5) was subjected to a symmetry operation $\hat{O_1}$ that
shifted the point to ($-1, -3, 5$). Subsequently, the point at ($-1, -3, 5$) was
subjected to another symmetry operation $\hat{O_2}$ that shifted this point to
($-1, -3, -5$).The symmetry operators $\hat{O_1}$ and $\hat{O_2}$ are, respectively

\begin{enumerate} \begin{multicols}{2}
\item $\hat{C_2}$(x) and $\hat{\sigma}$(xy)
\item $\hat{C_2}$(z) and $\hat{\sigma}$(xy)
\item $\hat{\sigma}$(xy) and $\hat{C_2}$(z)
\item $\hat{S_1}$ and $\hat{S_2}$
\end{multicols}
\end{enumerate}    

\item Adsorption of a gas with pressure P on a solid obeys the Langmuir adsorption
isotherm. For a fixed fractional coverage, the correct relation between K and P at a
fixed temperature is \\

[$K = k_a/k_b, k_a $ and $k_b$ are the rate constants for adsorption and desorption,
respectively. Assume non-dissociative adsorption.]

\begin{enumerate}
\begin{multicols}{4}

\item $K$$\propto$$P^{-1/2}$
\item $K$$\propto$$P$

\item $K$$\propto$$P^{-1}$
\item $K$$\propto$$P^{-1/2}$
    
\end{multicols}
\end{enumerate}


\item The temperature dependence of the rate constant for a second-order chemical
reaction obeys the Arrhenius equation. The SI unit of the ‘pre-exponential factor’
is

\begin{enumerate}
    \item $s^{-1}$
    \item $m^3 mol^{-1} s^{-1}$
    \item $ mol$$ m^{3} s^{-1}$
    \item ${(m^3mol^{-1})}^2s^{-1}$   
\end{enumerate}
    

\item The CORRECT reagent(s) for the given reaction is(are)

\begin{figure}[H]
\centering
\includegraphics[scale=0.6]{Figs/Screenshot 2025-08-16 at 11.06.44 AM.png}
\caption{}
\end{figure}

\begin{enumerate}
    \item $H_2O_2$ , NaOH
    \item   \begin{figure}[H]
            \centering
            \includegraphics[scale=0.6]{Figs/Screenshot 2025-08-16 at 11.06.47 AM.png}
            \caption{}
            \end{figure}

\item DIBAL-H, then mCPBA
\item $SO_3$ - pyridine, $Me_2SO$


\end{enumerate}

\item The CORRECT statement(s) about the ${^1}$H NMR spectra of compounds \textbf{P and Q}
is(are) \begin{figure}[H]
            \centering
            \includegraphics[scale=0.6]{Figs/Screenshot 2025-08-16 at 11.06.51 AM.png}
            \caption{}
            \end{figure}

\begin{enumerate}
    \item \textbf{P} shows a sharp singlet at $\delta$ = 3.70 ppm (for $H_a$ and $H_b$)
    \item \textbf{Q} shows a sharp singlet at $\delta$ = 3.70 ppm (for $H_a$ and $H_b$)
    \item \textbf{P} shows a AB-quartet centered at $\delta$ = 3.63 ppm (for $H_a$ and $H_b$)
    \item \textbf{Q} shows a AB-quartet centered at $\delta$ = 3.63 ppm (for $H_a$ and $H_b$)

    
\end{enumerate}


\item The CORRECT statement(s) about thallium halides is(are)

\begin{enumerate}
    \item TlF is highly soluble in water whereas other Tl-halides are sparingly soluble
    \item TlF adopts a distorted NaCl structure
    \item $TlI_3$ is isomorphic with $CsI_3$ and the oxidation state of Tl is +3
    \item Both TlBr and TlCl have CsCl structure
    
\end{enumerate}

\item The CORRECT statement(s) about the spectral line broadening in atomic spectra
is(are)

\begin{enumerate}
    \item The collision between atoms causes broadening of the spectral line
    \item Shorter the lifetime of the excited state, the broader is the line width
    \item Doppler broadening is more pronounced as the flame temperature increases
    \item In flame and plasma, the natural line broadening exceeds the collisional line
broadening
\end{enumerate}


\item Match the CORRECT option(s) from column A with column B according to the
metal centre present in the active site of metalloenzyme.


\begin{table}[H]
    \centering
    \begin{tabular}{|c|c|c|c|} \hline
         \multicolumn{2}{|c|}{\textbf{A}}&  \multicolumn{2}{|c|}{\textbf{B}}\\ \hline
         P&  Cu&  I& $B_{12}$-coenzyme\\ \hline
         Q&  Mo&  II& Carboxypeptidase\\ \hline
         R&  Co&  III& Nitrate reductase\\ \hline
         S&  Zn&  IV& Cytochrome P-450\\ \hline
         &  &  V& Tyrosinase\\ \hline
    \end{tabular}
    \caption{Caption}
    \label{tab:placeholder}
\end{table}

\item The CORRECT statement(s) about the following phase diagram for a hypothetical
pure substance X is(are)


\begin{figure}[H]
            \centering
            \includegraphics[scale=0.6]{Figs/Screenshot 2025-08-16 at 11.06.56 AM.png}
            \caption{}
            \end{figure}

\begin{enumerate}
    \item The molar volume of solid X is less than the molar volume of liquid X
    \item X does not have a normal boiling point
    \item The melting point of X decreases with increase in pressure
    \item On increasing the pressure of the gas isothermally, it is impossible to reach solid
phase before reaching liquid phase
\end{enumerate}

\item The parameter(s) fixed for each system in a canonical ensemble is(are)

\begin{enumerate}
    \item temperature
    \item pressure
    \item volume
    \item composition
\end{enumerate}

\item The number of peaks exhibited by T in its broadband proton decoupled $^{13}$C NMR
spectrum recorded at 25 ℃ in $CDCl_3$ is

\begin{figure}[H]
            \centering
            \includegraphics[scale=0.6]{Figs/Screenshot 2025-08-16 at 11.07.02 AM.png}
            \caption{}
            \end{figure}


\item The diffraction angle (in degree, rounded off to one decimal place) of (321) sets of
plane of a metal with atomic radius 0.125 nm, and adopting BCC structure is \\ 

(Given: the order of reflection is 1 and the wavelength of X-ray is 0.0771 nm)

\item For the angular momentum operator $\hat{L}$ and the spherical harmonics $Y_{lm}(\theta,\phi)$,
\begin{figure}[H]
\centering
\includegraphics[scale=0.6]{Figs/Screenshot 2025-08-16 at 12.37.39 PM.png}
\end{figure}
The value of $n$ is

\end{enumerate}

\textbf{Q.26 – Q.55 Carry TWO marks Each}

\begin{enumerate} \setcounter{enumi}{35}
    \item The major product P obtained in the following reaction sequence is
\begin{figure}[H]
\centering
\includegraphics[scale=0.6]{Figs/Screenshot 2025-08-16 at 11.07.06 AM.png}
\caption{}
\end{figure}
  

\begin{enumerate} 
    \item \begin{figure}[H]
\centering
\includegraphics[scale=0.6]{Figs/Screenshot 2025-08-16 at 11.07.09 AM.png}
\caption{}
\end{figure}

\item  \begin{figure}[H]
\centering
\includegraphics[scale=0.6]{Figs/Screenshot 2025-08-16 at 11.07.12 AM.png}
\caption{}
\end{figure}


\item  \begin{figure}[H]
\centering
\includegraphics[scale=0.6]{Figs/Screenshot 2025-08-16 at 11.07.16 AM.png}
\caption{}
\end{figure}

\item  
\begin{figure}[H]
\centering
\includegraphics[scale=0.6]{Figs/Screenshot 2025-08-16 at 11.07.20 AM.png}
\caption{}
\end{figure}

\end{enumerate}

\item The major product Q obtained in the following reaction sequence is
\begin{figure}[H]
\centering
\includegraphics[scale=0.6]{Figs/Screenshot 2025-08-16 at 11.07.25 AM.png}
\caption{}
\end{figure}
  

\begin{enumerate} 
    \item \begin{figure}[H]
\centering
\includegraphics[scale=0.6]{Figs/Screenshot 2025-08-16 at 11.07.28 AM.png}
\caption{}
\end{figure}

\item  \begin{figure}[H]
\centering
\includegraphics[scale=0.6]{Figs/Screenshot 2025-08-16 at 11.07.31 AM.png}
\caption{}
\end{figure}


\item  \begin{figure}[H]
\centering
\includegraphics[scale=0.6]{Figs/Screenshot 2025-08-16 at 11.07.35 AM.png}
\caption{}
\end{figure}

\item  
\begin{figure}[H]
\centering
\includegraphics[scale=0.6]{Figs/Screenshot 2025-08-16 at 11.07.38 AM.png}
\caption{}
\end{figure}

\end{enumerate}


\item The major product P obtained in the following reaction sequence is
\begin{figure}[H]
\centering
\includegraphics[scale=0.6]{Figs/Screenshot 2025-08-16 at 11.07.42 AM.png}
\caption{}
\end{figure}
  

\begin{enumerate} 
    \item \begin{figure}[H]
\centering
\includegraphics[scale=0.6]{Figs/Screenshot 2025-08-16 at 11.07.46 AM.png}
\caption{}
\end{figure}

\item  \begin{figure}[H]
\centering
\includegraphics[scale=0.6]{Figs/Screenshot 2025-08-16 at 11.07.48 AM.png}
\caption{}
\end{figure}


\item  \begin{figure}[H]
\centering
\includegraphics[scale=0.6]{Figs/Screenshot 2025-08-16 at 11.07.51 AM.png}
\caption{}
\end{figure}

\item  
\begin{figure}[H]
\centering
\includegraphics[scale=0.6]{Figs/Screenshot 2025-08-16 at 11.07.56 AM.png}
\caption{}
\end{figure}

\end{enumerate}


\item The major product P obtained in the following reaction sequence is
\begin{figure}[H]
\centering
\includegraphics[scale=0.6]{Figs/Screenshot 2025-08-16 at 11.08.01 AM.png}
\caption{}
\end{figure}
  

\begin{enumerate} 
    \item \begin{figure}[H]
\centering
\includegraphics[scale=0.6]{Figs/Screenshot 2025-08-16 at 11.08.04 AM.png}
\caption{}
\end{figure}

\item  \begin{figure}[H]
\centering
\includegraphics[scale=0.6]{Figs/Screenshot 2025-08-16 at 11.08.07 AM.png}
\caption{}
\end{figure}


\item  \begin{figure}[H]
\centering
\includegraphics[scale=0.6]{Figs/Screenshot 2025-08-16 at 11.08.10 AM.png}
\caption{}
\end{figure}

\item  
\begin{figure}[H]
\centering
\includegraphics[scale=0.6]{Figs/Screenshot 2025-08-16 at 11.08.14 AM.png}
\caption{}
\end{figure}

\end{enumerate}


\item The major product P obtained in the following reaction sequence is
\begin{figure}[H]
\centering
\includegraphics[scale=0.6]{Figs/Screenshot 2025-08-16 at 11.08.18 AM.png}
\caption{}
\end{figure}
  

\begin{enumerate} 
    \item \begin{figure}[H]
\centering
\includegraphics[scale=0.6]{Figs/Screenshot 2025-08-16 at 11.08.22 AM.png}
\caption{}
\end{figure}

\item  \begin{figure}[H]
\centering
\includegraphics[scale=0.6]{Figs/Screenshot 2025-08-16 at 11.08.25 AM.png}
\caption{}
\end{figure}


\item  \begin{figure}[H]
\centering
\includegraphics[scale=0.6]{Figs/Screenshot 2025-08-16 at 11.08.30 AM.png}
\caption{}
\end{figure}

\item  
\begin{figure}[H]
\centering
\includegraphics[scale=0.6]{Figs/Screenshot 2025-08-16 at 11.08.34 AM.png}
\caption{}
\end{figure}

\end{enumerate}






\item The major product P obtained in the following reaction sequence is
\begin{figure}[H]
\centering
\includegraphics[scale=0.6]{Figs/Screenshot 2025-08-16 at 11.08.39 AM.png}
\caption{}
\end{figure}
  

\begin{enumerate} 
    \item \begin{figure}[H]
\centering
\includegraphics[scale=0.6]{Figs/Screenshot 2025-08-16 at 11.08.44 AM.png}
\caption{}
\end{figure}

\item  \begin{figure}[H]
\centering
\includegraphics[scale=0.6]{Figs/Screenshot 2025-08-16 at 11.08.47 AM.png}
\caption{}
\end{figure}


\item  \begin{figure}[H]
\centering
\includegraphics[scale=0.6]{Figs/Screenshot 2025-08-16 at 11.08.51 AM.png}
\caption{}
\end{figure}

\item  
\begin{figure}[H]
\centering
\includegraphics[scale=0.6]{Figs/Screenshot 2025-08-16 at 11.08.54 AM.png}
\caption{}
\end{figure}

\end{enumerate}

\item Three different crystallographic planes of a unit cell of a metal are given below
(solid circles represent atom). The crystal system of the unit cell is

\begin{figure}[H]
\centering
\includegraphics[scale=0.6]{Figs/Screenshot 2025-08-16 at 11.08.57 AM.png}
\caption{}
\end{figure}

\begin{enumerate}
    \item triclinic
    \item monoclinic
    \item tetragonal
    \item orthorombic
\end{enumerate}

\item The number of equivalents of $H_2S$ gas released from the active site of \textit{rubredoxin,
2-iron ferredoxin, and 4-iron ferredoxin} when treated with mineral acid,
respectively, are

\begin{enumerate}
    \item 4,6,8
    \item 0,2,4
    \item 1,2,4
    \item 0,2,3
\end{enumerate}


\item The number of $v_{S=O}$ stretching vibration band(s) observed in the IR spectrum of
the high-spin ${[Mn(dmso)_6]}^{3+}$ complex (dmso: dimethylsulfoxide) is

\begin{enumerate}
    \item only one
    \item two with intensity ratio 1:2
    \item two with intensity ratio 1:1
    \item six with intensity ratio 1:1:1:1:1:1

\end{enumerate}



\item 

*indicates a radioactive isotope \\

\begin{figure}[H]
\centering
\includegraphics[scale=0.6]{Figs/Screenshot 2025-08-16 at 11.09.04 AM.png}
\caption{}
\end{figure} 
The rate constants in the given self-exchange electron transfer reactions at a
certain temperature follow

\begin{enumerate}
    \item $k_{11} > k_{22} > k_{33}$
    \item $k_{22} > k_{11} > k_{33}$
    \item $k_{22} > k_{22} > k_{11}$
    \item $k_{22} > k_{33} > k_{11}$
    
\end{enumerate}




\item The CORRECT distribution of the products in the following reaction is
\begin{figure}[H]
\centering
\includegraphics[scale=0.6]{Figs/Screenshot 2025-08-16 at 11.09.08 AM.png}
\caption{}
\end{figure}
  

\begin{enumerate} 
    \item \begin{figure}[H]
\centering
\includegraphics[scale=0.6]{Figs/Screenshot 2025-08-16 at 11.09.11 AM.png}
\caption{}
\end{figure}

\item  \begin{figure}[H]
\centering
\includegraphics[scale=0.6]{Figs/Screenshot 2025-08-16 at 11.09.15 AM.png}
\caption{}
\end{figure}


\item  \begin{figure}[H]
\centering
\includegraphics[scale=0.6]{Figs/Screenshot 2025-08-16 at 11.09.18 AM.png}
\caption{}
\end{figure}

\item  
\begin{figure}[H]
\centering
\includegraphics[scale=0.6]{Figs/Screenshot 2025-08-16 at 11.09.22 AM.png}
\caption{}
\end{figure}

\end{enumerate}


\item The addition of $K_4[Fe(CN)_6]$ to a neutral aqueous solution of the cationic species of
a metal produces a brown precipitate that is insoluble in dilute acid. The cationic
species is

\begin{enumerate} \begin{multicols}{4}
    

    \item $Fe^{3+}$
    \item ${UO_2}^{2+}$
    \item $Th^{4+}$
    \item $Cu^{2+}$
    \end{multicols}
\end{enumerate}



\item The electronic spectrum of a Ni(II) octahedral complex shows four $d-d$ bands,
labelled as \textbf{ P, Q, R, and S}. Match the bands corresponding to the transitions.


\begin{table}[H]
    \centering
    \begin{tabular}{|c|c|c|c|}\hline
         \multicolumn{2}{|c|}{${\lambda}_{max}$, nm ($\epsilon$ , $M^{-1}cm^{-1}$ }&
         \multicolumn{2}{|c|}{Transitions}\\ \hline
         P&   1000 (50)&  I& $^3{A_2}_g(F)$ → $^3{T_1}_g(P)$\\ \hline
         Q&  770 (8)&  II& $^3{A_2}_g(F)$ → $^3{T_1}_g(F)$\\ \hline
         R&  630 (55)&  III& $^3{A_2}_g(F)$ → $^3{T_2}_g(F)$\\ \hline
         S&   375 (110)&  IV& $^3{A_2}_g(F)$ → $^3{E}_g(D)$\\ \hline
    \end{tabular}
    \caption{Caption}
    \label{tab:placeholder}
\end{table}
    \begin{enumerate}
        \item P-IV, Q-III, R-II, S-I
        \item P-III, Q-IV, R-II, S-I
        \item P-II, Q-IV, R-I, S-III
        \item P-I, Q-IV, R-II, S-III
    \end{enumerate}


\item In the following table, the left column represents the rigid-rotor type and the right
column shows a set of molecules.


\begin{table}[H]
    \centering
    \begin{tabular}{|c|c|}
    \hline
          P. Symmetric Motor (Oblate)& 1.$SiH_4$\\ \hline
         Q. Symmetric Motor (Prolate)& 2.$CH_3Cl$\\ \hline
         R. Spherical Rotor& $C_6H_6$ \\ \hline
         S. Asymmetrical Rotor&$CH_3OH$ \\ \hline
         & $CO_2$\\
         \hline
    \end{tabular}
    \caption{Caption}
    \label{tab:placeholder}
\end{table}


\begin{enumerate}
    \item P-1, Q-2, R-3, S-4
    \item P-3, Q-2, R-1, S-4
    \item P-3, Q-5, R-1, S-2
    \item P-5, Q-4, R-3, S-2
\end{enumerate}

\item The CORRECT statement regarding the following three normal modes of
vibration of $SO_3$ 


\begin{figure}[H]
\centering
\includegraphics[scale=0.6]{Figs/Screenshot 2025-08-16 at 11.09.29 AM.png}
\caption{}
\end{figure}

\begin{enumerate}
    \item (I) and (II) are infrared active while (III) is infrared inactive
    \item (I) is infrared inactive while (II) and (III) are infrared active
    \item (I) and (III) are infrared inactive while (II) is infrared active
    \item None of the modes are infrared active since SO3 has zero dipole moment
\end{enumerate}


\item The reaction(s) that yield(s) 3-phenylcyclopentanone as the major product is(are)


\begin{enumerate} 
    \item \begin{figure}[H]
\centering
\includegraphics[scale=0.6]{Figs/Screenshot 2025-08-16 at 11.09.33 AM.png}
\caption{}
\end{figure}

\item  \begin{figure}[H]
\centering
\includegraphics[scale=0.6]{Figs/Screenshot 2025-08-16 at 11.09.37 AM.png}
\caption{}
\end{figure}


\item  \begin{figure}[H]
\centering
\includegraphics[scale=0.6]{Figs/Screenshot 2025-08-16 at 11.09.40 AM.png}
\caption{}
\end{figure}

\item  
\begin{figure}[H]
\centering
\includegraphics[scale=0.6]{Figs/Screenshot 2025-08-16 at 11.09.44 AM.png}
\caption{}
\end{figure}

\end{enumerate}



\item The reaction(s) that yield(s) M as the major product is(are)
\begin{figure}[H]
\centering
\includegraphics[scale=0.6]{Figs/Screenshot 2025-08-16 at 11.09.47 AM.png}
\caption{}
\end{figure}
  

\begin{enumerate} 
    \item \begin{figure}[H]
\centering
\includegraphics[scale=0.6]{Figs/Screenshot 2025-08-16 at 11.09.51 AM.png}
\caption{}
\end{figure}

\item  \begin{figure}[H]
\centering
\includegraphics[scale=0.6]{Figs/Screenshot 2025-08-16 at 11.09.54 AM.png}
\caption{}
\end{figure}


\item  \begin{figure}[H]
\centering
\includegraphics[scale=0.6]{Figs/Screenshot 2025-08-16 at 11.09.58 AM.png}
\caption{}
\end{figure}

\item  
\begin{figure}[H]
\centering
\includegraphics[scale=0.6]{Figs/Screenshot 2025-08-16 at 11.10.03 AM.png}
\caption{}
\end{figure}  
\end{enumerate}



\item The CORRECT statement(s) regarding $B_{10}H_{14}$ is(are)

\begin{enumerate}
    \item Brønsted acidity of $B_{10}H_{14}$ is higher than that of $B_{5}H_{9}$
    \item Structurally $B_{10}H_{14}$ is a closo-borane
    \item The metal-promoted fusion of ${B_{5}H_{8}}^-$ produces $B_{10}H_{14}$
    \item Both $B_{10}H_{14}$ and $B_{10}H_{12}(SEt_2)_2$have the same number of valence electrons
\end{enumerate}

\item The CORRECT statement(s) about the Group-I metals is(are)

\begin{enumerate}
    \item Reactivity of Group-I metals with water decreases down the group
    \item Among the Group-I metals, Li spontaneously reacts with $N_2$ to give a red-brown
layer-structured material
\item Thermal stability of Group-I metal peroxides increases down the group
\item All the Group-I metal halide are high-melting colorless crystalline solids
\end{enumerate}


\item The compound(s) that satisfies/satisfy the 18-electron rule is(are) \\

(Atomic number of Os = 76, Rh = 45, Mo = 42, and Fe = 26)

\begin{enumerate} 
    \item \begin{figure}[H]
\centering
\includegraphics[scale=0.6]{Figs/Screenshot 2025-08-16 at 11.10.09 AM.png}
\caption{}
\end{figure}

\item  \begin{figure}[H]
\centering
\includegraphics[scale=0.6]{Figs/Screenshot 2025-08-16 at 11.10.13 AM.png}
\caption{}
\end{figure}


\item  \begin{figure}[H]
\centering
\includegraphics[scale=0.6]{Figs/Screenshot 2025-08-16 at 11.10.17 AM.png}
\caption{}
\end{figure}

\item  
\begin{figure}[H]
\centering
\includegraphics[scale=0.6]{Figs/Screenshot 2025-08-16 at 11.10.20 AM.png}
\caption{}
\end{figure}

\end{enumerate}



    
\item For three operators $\hat{A}$, $\hat{B}$, and $\hat{C}$,  {$\hat{A}$,  {$\hat{B}$, $\hat{C}$}} +  {$\hat{B}$,  {$\hat{C}$, $\hat{A}$}}

\begin{enumerate}
    \item {$\hat{C}$,  [{$\hat{A}$, $\hat{B}$}]}
    \item {$\hat{C}$,  [{$\hat{B}$, $\hat{A}$}]}
    \item {[{$\hat{B}$, $\hat{A}$}], $\hat{C}$}
    \item  {$\hat{A}$,  [{$\hat{B}$, $\hat{C}$}]}
\end{enumerate}

\item The difference between the number of Gauche-butane interactions present in P
and Q is 

\begin{figure}[H]
\centering
\includegraphics[scale=0.6]{Figs/Screenshot 2025-08-16 at 11.10.25 AM.png}
\caption{}
\end{figure}

\item The calculated magnetic moment (in BM, rounded off to two decimal places) of a
$Ce^{3+}$ complex is

\item The state of the electron in a He+ ion is described by the following normalized
wavefunction 
\begin{figure}[H]
\centering
\includegraphics[scale=0.6]{Figs/Screenshot 2025-08-16 at 11.10.30 AM.png}
\caption{}
\end{figure}

Here, $R_{nl}$ and $Y_{lm}$ represent the radial and angular components of the
eigenfunctions of He+ ion, respectively, and $x$ is an unknown constant. If the energy
of the ion is measured in the above state, the probability (rounded off to two decimal
places) of obtaining the energy of $-\frac{2}{9}$ atomic unit is

\item A certain wavefunction for the hydrogen-like atom is given by
\begin{figure}[H]
\centering
\includegraphics[scale=0.6]{Figs/Screenshot 2025-08-16 at 11.10.33 AM.png}
\caption{}
\end{figure}

The number of node(s) in this wavefunction is

\item EMF of the following cell \\

$Cu$ $ | $ $CuSO_4$ (aq, 1.0 mol/kg) $ ||$  $Hg_2SO_4$(s)$ | $$Hg(l)$ $ | $$Pt$ \\ 

at 25 ℃ and 1 bar is 0.36 V. The value of the mean activity coefficient (rounded
off to three decimal places) of $CuSO_4$ at 25 ℃ and 1 bar is\\ 

[Given: Standard electrode potential values at 25 ℃ for] \\

$Cu^{2+}$ + $2e^{-}$ → Cu and \\

$Hg_2SO_4$ + 2$e^{-}$ → 2Hg + ${SO_4}^{2-}$ \\

are 0.34 V and 0.62 V, respectively. \\

Consider: RT/F at 25℃ = 0.0256 V



\item The radius of gyration (in nm, rounded off to one decimal place), for three
dimensional random coil linear polyethylene of molecular weight 8,40,000 is\\

[Given: C-C bond length = 0.154 nm]

\item The activation energy of the elementary gas-phase reaction $O_3$ + NO → $NO_2$ + $O_2$
is 10.5 kJ $mol^{-1}$. The value of the standard enthalpy of activation (rounded off to
two decimal places in kJ mol-1) at 25 ℃ is\\

[Given: R is 8.314 $J mol^{-1} K^{-1}$]

\item In a collection of molecules, each molecule has two non-degenerate energy levels
that are separated by 5000 $cm^{-1}$. On measuring the population at a particular
temperature, it was found that the ground state population is 10 times that of the
upper state. The temperature (in K, rounded off to the nearest integer) of
measurement is\\ 

[Given: Value of the Boltzmann constant = 0.695 $cm^{-1}K^{-1}$]

\item The change in entropy of the surroundings (in $J K^{-1}$, rounded off to two decimal
places) to convert 1 mol of supercooled water at 263 K to ice at 263 K at 1 bar is


[Consider: $\Delta_{fus}H°$ at 273 K = 6.0 kJ $mol^{-1}$, and the molar heat capacity of water is
higher than that of ice by 37.0 $J K^{-1} mol^{-1}$ in the temperature range of 263 K to
273 K]


\end{enumerate}
\end{document}
