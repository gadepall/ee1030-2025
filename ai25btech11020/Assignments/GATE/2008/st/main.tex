\documentclass[journal, 11pt, onecolumn]{IEEEtran}
\usepackage{graphicx}
\usepackage{fancyhdr}
\usepackage{lastpage}
\usepackage[a4paper,margin=1in]{geometry}
\usepackage{newtxtext,newtxmath}
\usepackage{enumitem}
\usepackage{multicol}
\usepackage{array}

\usepackage{cite}
\usepackage{amsmath,amssymb,amsfonts,amsthm}
\usepackage{algorithmic}
\usepackage{graphicx}
\usepackage{textcomp}
\usepackage{xcolor}

\usepackage{listings}
\usepackage{enumitem}
\usepackage{mathtools}
\usepackage{gensymb}
\usepackage{comment}
\usepackage[breaklinks=true]{hyperref}
\usepackage{tkz-euclide} 
\usepackage{gvv}                                        
\def\inputGnumericTable{}                                 
\usepackage[latin1]{inputenc}     
\usepackage{xparse}
\usepackage{color}                                            
\usepackage{array}                                            
\usepackage{longtable}                                       
\usepackage{calc}                                             
\usepackage{multirow}
\usepackage{multicol}
\usepackage{hhline}                                           
\usepackage{ifthen}                                           
\usepackage{lscape}
\usepackage{tabularx}
\usepackage{array}
\usepackage{float}
\newtheorem{theorem}{Theorem}[section]
\newtheorem{problem}{Problem}
\newtheorem{proposition}{Proposition}[section]
\newtheorem{lemma}{Lemma}[section]
\newtheorem{corollary}[theorem]{Corollary}
\newtheorem{example}{Example}[section]
\newtheorem{definition}[problem]{Definition}
\newcommand{\BEQA}{\begin{eqnarray}}
\newcommand{\EEQA}{\end{eqnarray}}
\newcommand{\define}{\stackrel{\triangle}{=}}
\theoremstyle{remark}
\newtheorem{rem}{Remark}


\pagestyle{fancy}

% Header and footer text
\fancyhead[L]{2008}
\fancyhead[C]{ }
\fancyhead[R]{MAIN PAPER-MT}
\fancyfoot[L]{MT}
\fancyfoot[C]{ }
\fancyfoot[R]{\thepage/\pageref{LastPage}}

% Adjust distances
\setlength{\headheight}{14pt}
\setlength{\headsep}{5pt}
\setlength{\footskip}{20pt}


% Line thickness
\renewcommand{\headrulewidth}{0.4pt}
\renewcommand{\footrulewidth}{0.4pt}

\begin{document}

\begin{center}
    \Large{AI25BTECH11020}
\end{center} 



\begin{enumerate}
\item The yield point of the phenomenon observed in annealed low carbon steels is due to the presence of

\begin{multicols}{4}
\begin{enumerate}
\item silicon
\item chromium
\item phosphorous 
\item carbon
\end{enumerate}
\end{multicols}

\hfill(GATE MT 2008)

\item In a tensile test of a ductile material, necking starts at

\begin{multicols}{2}
\begin{enumerate}
\item lower yield stress
\item upper yield stress
\item ultimate tensile strength
\item just before fracture
\end{enumerate}
\end{multicols}


\hfill(GATE MT 2008)

\item Fatigue resistance of a steel is reduced by

\begin{multicols}{2}
\begin{enumerate}
\item decarburization
\item polishing the surface
\item reducing the grain size
\item shot peening
\end{enumerate}
\end{multicols}


\hfill(GATE MT 2008)

\item The stress concentration factor K, for a circular hole located at the center of a plate is

\begin{multicols}{4}
\begin{enumerate}
\item $0$
\item $1$
\item $3$ 
\item tends to $\infty$
\end{enumerate}
\end{multicols}


\hfill(GATE MT 2008)

\item Cassiterite is an important source for

\begin{multicols}{4}
\begin{enumerate}
\item tin
\item titanium 
\item molybdenum
\item thorium
\end{enumerate}
\end{multicols}

\hfill(GATE MT 2008)

\item High top pressure in the blast furnace

\begin{multicols}{1}
\begin{enumerate}
\item decreases the time of contact between gas and solid
\item increases the time of contact between gas and solid 
\item increases fuel consumption
\item increases the rate of solution loss reaction
\end{enumerate}
\end{multicols}

\hfill (GATE MT 2008)

\item For a closed system of fixed internal energy and volume, at equilibrium

\begin{multicols}{2}
\begin{enumerate}
\item Gibb's free energy is minimum
\item entropy is maximum
\item Helmholtz's free energy is minimum
\item enthalpy is maximum
\end{enumerate}
\end{multicols}

\hfill(GATE MT 2008)

\item Intergranular corrosion of 18-8 stainless steel can NOT be prevented by

\begin{multicols}{1}
\begin{enumerate}
\item reducing the carbon content to less than 0.05\%
\item quenching it from high temperature to prevent chromium carbide precipitation 
\item adding strong carbide forming elements
\item increasing the carbon content
\end{enumerate}
\end{multicols}

\hfill(GATE MT 2008)

\item Riser is NOT required for the castings of \hfill (GATE EE 2025)

\begin{multicols}{4}
\begin{enumerate}
\item grey cast iron
\item white cast iron 
\item Al-4\% Cu
\item Al-12\% Si
\end{enumerate}
\end{multicols}

\hfill(GATE MT 2008)

\item The NDT technique used to detect deep lying defects in a large sized casting is

\begin{multicols}{2}
\begin{enumerate}
\item liquid penetrant inspection
\item magnetic particle inspection
\item ultrasonic inspection
\item eddy current inspection
\end{enumerate}
\end{multicols}

\hfill(GATE MT 2008)

\item The maximum number of phases in a quaternary system at atmospheric pressure are

\begin{multicols}{4}
\begin{enumerate}
\item 2
\item 3 
\item 4
\item 5
\end{enumerate}
\end{multicols}

\hfill(GATE MT 2008)

\item In Cu-Al phase diagram, the solubility of Al in Cu at room temperature is about 10\% and that of Cu in Al is less than 1\%. The Hume-Rothery rule that justifies this difference is

\begin{multicols}{2}
\begin{enumerate}
\item size factor
\item electro-negativity
\item structure
\item valency
\end{enumerate}
\end{multicols}

\hfill(GATE MT 2008)

\item Mannesmann process

\begin{multicols}{1}
\begin{enumerate}
\item is a cold working process
\item is used for making thin walled seamless tubes 
\item uses parallel rolls
\item is used for making thick walled seamless tubes
\end{enumerate}
\end{multicols}

\hfill(GATE MT 2008)

\item The intensive thermodynamic variables among the following are

\begin{multicols}{4}
\begin{enumerate}
\item pressure
\item volume 
\item temperature
\item enthalpy
\end{enumerate}
\end{multicols}

\hfill(GATE MT 2008)


\begin{multicols}{4}
\begin{enumerate}
\item P, Q
\item P, R 
\item R, S
\item Q, R
\end{enumerate}
\end{multicols}

\hfill(GATE MT 2008)

\item In a binary phase diagram, the activity of the solute in a two phase field at a given temperature

\begin{multicols}{1}
\begin{enumerate}
\item increases linearly with the solute content
\item decreases linearly with the solute content
\item remains constant
\item is proportional to the square root of the solute content
\end{enumerate}
\end{multicols}

\hfill(GATE MT 2008)

\item In Jominy curves of steel A (Fe-0.4\% C) and steel B (Fe-0.4\% C -1.0\% Ni),

\begin{multicols}{1}
\begin{enumerate}
\item depth of hardening in steel A is more than in steel B 
\item depth of hardening in steel B is more than in steel A
\item hardness at the quenched end in steel A is more than in steel B
\item hardness at the quenched end in steel B is more than in steel A
\end{enumerate}
\end{multicols}

\hfill(GATE MT 2008)

\item Determinant of \myvec{3 & 1 & 2\\1 & 2 & 1\\ 4 & 2 & 3}
%$\begin{pmatrix}3 & 1 & 2\\1 & 2 & 1\\ 4 & 2 & 3\end{pmatrix}$

\begin{multicols}{4}
\begin{enumerate}
\item $-2$
\item $-1$ 
\item $1$
\item $2$
\end{enumerate}
\end{multicols}

\hfill(GATE MT 2008)

\item  $\displaystyle \int \frac{dx}{ax + b}$ is

\begin{multicols}{4}
\begin{enumerate}
\item {\large $\frac{1}{b}$ln\brak{ax+b} + c}
\item {\large ln\brak{ax+b) + c}
\item {\large b ln\brak{ax+b} + c}
\item {\large $\frac{1}{a}$ln\brak{ax+b} + c}
\end{enumerate}
\end{multicols}

\hfill(GATE MT 2008)

\item The value of $dy/dx$ for the following data set at $x = 3.5$, computed by central difference method, is
\begin{center}
\begin{tabular}{ | m{1cm} | m{1cm}| m{1cm} | m{1cm} | m{1cm} | m{1cm} | } 
\hline
x & 1 & 2 & 3 & 4 & 5 \\ 
\hline
y & 0 & 3 & 8 & 15 & 24 \\ 
\hline 
\end{tabular}
\end{center}

\begin{multicols}{4}
\begin{enumerate}
\item $3.5$
\item $7$
\item $10.5$
\item $14$
\end{enumerate}
\end{multicols}

\hfill(GATE MT 2008)


\item The velocity at which particles from a fluidized bed are carried away by the fluid passing through it, is known as

\begin{multicols}{2}
\begin{enumerate}
\item elutriation velocity
\item terminal velocity
\item minimum fluidization velocity
\item superficial velocity
\end{enumerate}
\end{multicols}

\hfill(GATE MT 2008)

\item A metal with an average grain size of 36 $\mu$m has yield strength of 250 MPa and that with 4 $\mu$m has 500 MPa. The friction stress of the metal in MPa is

\begin{multicols}{4}
\begin{enumerate}
\item $31.2$
\item $62.5$
\item $125$
\item $250$
\end{enumerate}
\end{multicols}

\hfill(GATE MT 2008)

\item The stacking sequence of close packed planes with a stacking fault is

\begin{multicols}{2}
\begin{enumerate}
\item \textit{a b c a b c a b c}
\item \textit{a b a b a b a b a b}
\item \textit{a b c a c a b c a b}
\item \textit{a b c a b a c b a}
\end{enumerate}
\end{multicols}

\hfill(GATE MT 2008)

\item The slip directions on a ($\bar{1}$1$\bar{1}$) plane of a fcc crystal are

\begin{multicols}{2}
\begin{enumerate}
\item $[101], [011], [110]$
\item $[101], [110], [101]$
\item $[101], [110], [011]$
\item $[101], [110], [011]$
\end{enumerate}
\end{multicols}

\hfill(GATE MT 2008)

\item The correct statements among the following are
\begin{enumerate}
    \item screw dislocations cannot climb
    \item screw dislocations cannot cross-slip
    \item edge dislocations cannot climb
    \item edge dislocations cannot cross-slip
\end{enumerate}

\begin{multicols}{4}
\begin{enumerate}
\item P, R
\item P, S
\item Q, R
\item Q, S
\end{enumerate}
\end{multicols}

\hfill(GATE MT 2008)

\item A steel bar (elastic modulus = 200 GPa and yield strength = 400 MPa) is loaded to a tensile stress of 1 GPa and undergoes a plastic strain of 2\%. The elastic strain in the bar in percent is

\begin{multicols}{4}
\begin{enumerate}
\item $0$
\item $0.2$
\item $0.5$
\item $2.0$
\end{enumerate}
\end{multicols}

\hfill(GATE MT 2008)



\item The ASTM grain size number of a material which shows 64 grains per square inch at a magnification of 200X is

\begin{multicols}{4}
\begin{enumerate}
\item $5$
\item $6$
\item $7$
\item $8$
\end{enumerate}
\end{multicols}

\hfill(GATE MT 2008)

\item Two samples P and Q of a brittle material have crack lengths in the ratio 4:1. The ratio of fracture strengths of P and Q, measured normal to the cracks, will be 
\begin{multicols}{4}
\begin{enumerate}
\item $1:4$
\item $1:2$
\item $2:1$
\item $4:1$
\end{enumerate}
\end{multicols}

\hfill(GATE MT 2008)

\item The structure-sensitive properties are
\begin{enumerate}
    \item elastic modulus
    \item yield strength
    \item melting point
    \item fracture strength
\end{enumerate}

\begin{multicols}{4}
\begin{enumerate}
\item P, S
\item Q, S
\item Q, R
\item P, R
\end{enumerate}
\end{multicols}

\hfill(GATE MT 2008)

\item The time taken for 50\% recrystallization of cold worked Al is 100 hours at 500 K and 10 minutes at 600 K. Assuming Arrhenius kinetics, the activation energy for recrystallization in kJ $mol^{-1}$ is

\begin{multicols}{4}
\begin{enumerate}
\item $50$
\item $80$
\item $160$
\item $320$
\end{enumerate}
\end{multicols}

\hfill(GATE MT 2008)

\item Match the mechanical behaviour in Group 1 with the terms in Group 2 
\begin{multicols}{2}
\underline{Group 1}
\begin{enumerate}[label=(\Alph*), start=16]
\item Low cycle fatigue
\item Creep
\item Impact toughness
\item Stretcher strain
\end{enumerate}


\underline{Group 2}
\begin{enumerate}[label=(\arabic*), start=1]
\item Charpy test
\item Portevin-LeChatelier effect
\item Coffin-Manson equation
\item Larson-Miller parameter
\item Jominy test
\end{enumerate}
\end{multicols}    

\begin{multicols}{2}
\begin{enumerate}
\item P-$2$, Q-$4$, R-$1$, S-$5$
\item P-$2$, Q-$1$, R-$5$, S-$3$
\item P-$3$, Q-$4$, R-$1$, S-$2$
\item P-$3$, Q-$1$, R-$4$, S-$5$
\end{enumerate}
\end{multicols}

\hfill(GATE MT 2008)

\item Match the processes in Group 1 with the physical principles in Group 2

\begin{multicols}{2}
\underline{Group 1}
\begin{enumerate}
\item Floatation
\item Jigging
\item Tabling
\item Heavy media separation
\end{enumerate}

\underline{Group 2}
\begin{enumerate}[label=(\arabic*), start=1]
\item Differential initial acceleration
\item Differential lateral movement
\item Difference in density
\item Modification of surface tension 
\end{enumerate}
\end{multicols}

\begin{multicols}{2}
\begin{enumerate}
\item $P-4, Q-1, R-2, S-3$
\item $P-4, Q-1, R-3, S-2$
\item $P-2, Q-3, R-4, S-1$
\item $P-1, Q-3, R-4, S-2$
\end{enumerate}
\end{multicols}

\hfill(GATE MT 2008)

\item Which of the following is the solution for {\LARGE $\frac{\partial z}{\partial t}$ = $\frac{\partial^2z}{\partial^2x}$}

\begin{multicols}{2}
\begin{enumerate}
\item {\large$z(x,t)= [Asinx]$$e^{-\lambda^2 t}$}
\item {\large$z(x,t)= [Asin(\lambda x)]$$e^{-\lambda^2 t}$} 
\item {\large$z(x,t)=\frac{A}{t}$$e^{-x^2 t}$}
\item {\large$z(x,t)= [Acos(\lambda x)]$$\sqrt{t}$}
\end{enumerate}
\end{multicols}

\hfill(GATE MT 2008)

\item Match the unit processes in Group 1 with the objectives in Group 2
\begin{multicols}{2}
\underline{Group 1}
\begin{enumerate}[label=(\Alph*), start=16]
\item Leaching
\item Cementation
\item Roasting
\item Converting
\end{enumerate}

\underline{Group 2}
\begin{enumerate}[label=(\arabic*), start=1]
\item Precipitation of metal in aqueous solution
\item Selective dissolution of metal
\item Conversion of matte to metal
\item Conversion of sulphide to oxide 
\item Separation of metal from slag
\end{enumerate}
\end{multicols}

\begin{multicols}{2}
\begin{enumerate}
\item $P-2, Q-1, R-3, S-5$
\item $P-2, Q-1, R-4, S-3$
\item $P-3, Q-4, R-5, S-2$
\item $P-4, Q-3, R-2, S-1$
\end{enumerate}
\end{multicols}

\hfill(GATE MT 2008)

\item Match the following metals in Group 1 with their production methods in Group 2
\begin{multicols}{2}
\underline{Group 1}
\begin{enumerate}[label=(\Alph*), start=16]
\item Titanium
\item Nickel
\item Magnesium
\item Zinc
\end{enumerate}

\underline{Group 2}
\begin{enumerate}[label=(\arabic*), start=1]
\item Mond's process
\item Pidgeon's process
\item Imperial smelting
\item Kroll's process
\item Cyanidation
\end{enumerate}
\end{multicols}

\setlist[enumerate,2]{label=(\MakeUppercase{\alph*}), itemsep=0pt, leftmargin=1.8em}
\begin{multicols}{2}
\begin{enumerate}
\item $P-5, Q-2, R-3, S-4$
\item $P-3, Q-5, R-4, S-2$
\item $P-4, Q-1, R-2, S-3$
\item $P-4, Q-1, R-5, S-3$
\end{enumerate}
\end{multicols}

\hfill(GATE MT 2008)

\item Manganese recovery in steelmaking is aided by
\begin{enumerate}[label=(\MakeUppercase{\alph*}), start=16]
\item oxidizing slag
\item reducing slag
\item high temperature
\item low temperature
\item acidic slag 
\end{enumerate}

\begin{multicols}{4}
\begin{enumerate}
\item P, Q
\item Q, S
\item Q, R
\item P, R
\end{enumerate}
\end{multicols}

\hfill(GATE MT 2008)

\item A flotation plant treats 100 tons of chalcopyrite containing 2\% Cu and produces 6 tons of concentrate. 
The concentrate has 25\% Cu. The percentage Cu in the tailings is
\begin{multicols}{4}
\begin{enumerate}
\item $0.35 $
\item $0.53$ 
\item $0.86$ 
\item $0.93$
\end{enumerate}
\end{multicols}

\hfill(GATE MT 2008)

\item One ton of liquid steel initially containing $0.08$\% S is brought into equilibrium with $0.1$ ton of liquid slag containing no sulphur. The sulphur distribution ratio 
$\displaystyle \frac{\%S_{\text{slag}}}{\%S_{\text{metal}}} = 30$ 
at equilibrium. The final sulphur content of steel in wt.\% is

\begin{multicols}{4}
\begin{enumerate}
\item $0.01$ 
\item $0.02$ 
\item $0.03$ 
\item $0.04$
\end{enumerate}
\end{multicols}

\hfill(GATE MT 2008)  
\item Deoxidation of liquid steel with ferrosilicon produces spherical silica particles. 
The particles of $5$ $\mu$m diameter take $3000$ minutes to float up through a $2$ m height of liquid steel. 
For particles of $50$ $\mu$m diameter to float up through the same height, the time required in minutes is

\begin{multicols}{4}
\begin{enumerate}
\item $30$ 
\item $300$ 
\item $960$ 
\item $3000$
\end{enumerate}
\end{multicols}

\hfill(GATE MT 2008)

\item Match applications in Group 1 with the commonly used corrosion protection methods in Group 2
\begin{multicols}{2}
\underline{Group 1}
\begin{enumerate}[label=(\Alph*), start=16]
\item Seagoing vessel
\item Underground pipeline
\item Electric traction tower
\item Electric poles
\end{enumerate}

\underline{Group 2}
\begin{enumerate}[label=(\arabic*), start=1]
\item Inorganic coating
\item Sacrificial anode
\item Aluminium paint
\item Impressed current
\end{enumerate}
\end{multicols}

\begin{multicols}{2}
\begin{enumerate}
\item $P-2, Q-4, R-5, S-3$
\item $P-2, Q-3, R-5, S-1$
\item $P-1, Q-2, R-5, S-4$
\item $P-4, Q-3, R-1, S-2$
\end{enumerate}
\end{multicols}
\hfill(GATE MT 2008) 

\item For a regular solution A-B, $\Delta H$ is 2660.5 J at $x_B = 0.6$. The critical point of the miscibility gap in the system would be at

\begin{multicols}{2}
\begin{enumerate}
\item $x_B = 0.5, T = 1000 \ \mathrm{K}$
\item $x_B = 0.6, T = 1000 \ \mathrm{K}$
\item $x_B = 0.5, T = 500 \ \mathrm{K}$
\item $x_B = 0.6, T = 2000 \ \mathrm{K}$
\end{enumerate}
\end{multicols}

\hfill(GATE MT 2008) 
\item For Ni + 0.5O$_2$ = NiO, $\Delta G^\circ = -250,000 + 100T$ Joules. At 1000 K, the $p_{O_2}$ in equilibrium with Ni/NiO in atm is 

\begin{multicols}{4}
\begin{enumerate}
\item $2.13 \times 10^{-16}$
\item $8.54 \times 10^{-16}$
\item $1.46 \times 10^{-8}$
\item $2.92 \times 10^{-8}$
\end{enumerate}
\end{multicols}

\hfill(GATE MT 2008) 
\item The planar density for (111) plane in a fcc crystal is 

\begin{multicols}{4}
\begin{enumerate}
\item $0.68$
\item $0.74$ 
\item $0.85$ 
\item $0.91$
\end{enumerate}
\end{multicols}

\hfill(GATE MT 2008)

\item Iridium has fcc structure. Its density and atomic weight are 22,400 kg/m$^3$ and 192.2, respectively. The atomic radius of iridium in nm is

\begin{multicols}{4}
\begin{enumerate}
\item $0.126$ 
\item $0.136$ 
\item $0.146$ 
\item $0.156$
\end{enumerate}
\end{multicols}

\hfill(GATE MT 2008) 

\item Match the names in Group 1 with the invariant reactions in binary phase diagrams in Group 2
\begin{multicols}{2}
\underline{Group 1}
\begin{enumerate}
\item Eutectic
\item Eutectoid 
\item Peritectoid
\item Monotectic
\end{enumerate}

\underline{Group 2}
\begin{enumerate}[label=(\arabic*), start=1]
\item S1 = S2 + S3
\item L = S1 + S2
\item L1 = L2 + S
\item S1 + S2 = S3
\end{enumerate}
\end{multicols}

\begin{multicols}{2}
\begin{enumerate} 
\item $P-2, Q-1, R-3, S-4$
\item $P-2, Q-1, R-4, S-3$
\item $P-3, Q-4, R-2, S-1$
\item $P-4, Q-3, R-1, S-2$
\end{enumerate}
\end{multicols}

\hfill(GATE MT 2008)

\item Match the properties in Group 1 with the units in Group 2
\begin{multicols}{2}
\underline{Group 1}
\begin{enumerate}[label=(\Alph*), start=16]
\item Thermal conductivity
\item Heat transfer coefficient 
\item Specific heat
\item Diffusivity 
\end{enumerate}

\underline{Group 2}
\begin{enumerate}[label=(\arabic*), start=1]
\item J m$^{-2}$ s$^{-1}$ K$^{-1}$
\item J m$^{-1}$ s$^{-1}$ K$^{-1}$ 
\item m$^3$ s$^{-1}$
\item mol$^{-1}$ K$^{-1}$
\end{enumerate}
\end{multicols}

\begin{multicols}{2}
\begin{enumerate} 
\item $P-1, Q-2, R-4, S-3$
\item $P-2, Q-3, R-1, S-4$
\item $P-2, Q-1, R-4, S-3$
\item $P-2, Q-4, R-3, S-1$
\end{enumerate}
\end{multicols}

\hfill(GATE MT 2008) 

\item Match the heat treatment processes of steels in Group 1 with the microstructural features in Group 2
\begin{multicols}{2}
\underline{Group 1}
\begin{enumerate}[label=(\Alph*), start=16]
\item Quenching
\item Maraging 
\item Tempering
\item Austempering
\end{enumerate}

\underline{Group 2}
\begin{enumerate}[label=(\arabic*), start=1]
\item Bainite
\item Martensite
\item Intermetallic precipitates
\item Epsilon carbide
\end{enumerate}
\end{multicols}

\begin{multicols}{2}
\begin{enumerate} 
\item $P-2, Q-3, R-1, S-4$
\item $P-1, Q-3, R-2, S-4$
\item $P-2, Q-3, R-4, S-1$
\item $P-3, Q-2, R-1, S-4$
\end{enumerate}
\end{multicols}

\hfill(GATE MT 2008) 

\item Match the nonferrous alloys in Group 1 with their applications in Group 2
\begin{multicols}{2}
\underline{Group 1}
\begin{enumerate}[label=(\Alph*), start=16]
\item Ti alloy
\item Zr alloy
\item Ni alloy
\item Cu alloy
\end{enumerate}

\underline{Group 2}
\begin{enumerate}[label=(\arabic*), start=1]
\item Nuclear reactors
\item Bells
\item Dental implants 
\item Gas turbines
\end{enumerate}
\end{multicols}

\begin{multicols}{2}
\begin{enumerate} 
\item $P-3, Q-1, R-4, S-2$
\item $P-2, Q-3, R-4, S-1$
\item $P-2, Q-1, R-3, S-4$
\item $P-3, Q-4, R-1, S-2$
\end{enumerate}
\end{multicols}
\hfill(GATE MT 2008) 

\item Match the materials in Group 1 with their functional applications in Group 2
\begin{multicols}{2}
\underline{Group 1}
\begin{enumerate}[label=(\Alph*), start=16]
\item Nb$_3$Sn
\item GaAs 
\item Fe-$4$\%Si alloy
\item SiO$_2$
\end{enumerate}

\underline{Group 2}
\begin{enumerate}[label=(\arabic*), start=1]
\item Dielectric
\item Soft magnet
\item Superconductor  
\item Semiconductor
\end{enumerate}
\end{multicols}

\begin{multicols}{2}
\begin{enumerate} 
\item $P-3, Q-1, R-4, S-2$
\item $P-1, Q-4, R-2, S-3$
\item $P-3, Q-2, R-4, S-1$
\item $P-3, Q-4, R-2, S-1$
\end{enumerate}
\end{multicols}
\hfill(GATE MT 2008) 

\item An annealed hypoeutectoid steel has 10\% of proeutectoid ferrite at room temperature. The eutectoid carbon content of the steel is 0.8\%. The carbon content in the steel in percent is

\begin{multicols}{4}
\begin{enumerate} 
\item $0.58$ 
\item $0.68$ 
\item $0.72$ 
\item $0.78$
\end{enumerate}
\end{multicols}

\hfill(GATE MT 2008) 
\item The melting point and latent heat of fusion of copper are 1356 K and 13 kJ mol$^{-1}$, respectively. 
Assume that the specific heats of solid and liquid are same. 
The free energy change for the liquid to solid transformation at 1250 K in kJ mol$^{-1}$ is

\begin{multicols}{4}
\begin{enumerate} 
\item $-4$
\item $-3$
\item $-2$
\item $-1$
\end{enumerate}
\end{multicols}

\hfill(GATE MT 2008) 

\item According to the Clausius Clapeyron equation, the melting point of aluminium

\begin{multicols}{2}
\begin{enumerate} 
\item increases linearly with pressure
\item decreases linearly with pressure
\item increases exponentially with pressure
\item does not vary with pressure
\end{enumerate}
\end{multicols}

\item Match the cast irons in Group 1 with the distinguishing microstructural features in Group 2
\begin{multicols}{2}
\underline{Group 1}
\begin{enumerate}[label=(\Alph*), start=16]
\item Grey cast iron
\item Ductile cast iron  
\item Malleable cast iron
\item White cast iron
\end{enumerate}

\underline{Group 2}
\begin{enumerate}[label=(\arabic*), start=1]
\item Temper graphite
\item Pearlite
\item Graphite flakes  
\item Massive cementite
\end{enumerate}
\end{multicols}

\begin{multicols}{2}
\begin{enumerate} 
\item $P-3, Q-5, R-4, S-2$
\item $P-1, Q-5, R-4, S-2$
\item $P-2, Q-4, R-5, S-3$
\item $P-3, Q-5, R-1, S-4$
\end{enumerate}
\end{multicols}
\hfill(GATE MT 2008) 

\item Match the casting defects in Group 1 with causes given in Group 2
\begin{multicols}{2}
\underline{Group 1}
\begin{enumerate}[label=(\Alph*), start=16]
\item Hot tear
\item Misrun  
\item Blister
\item Rat tail 
\end{enumerate}

\underline{Group 2}
\begin{enumerate}[label=(\arabic*), start=1]
\item Insufficient melt super heat
\item High residual stresses
\item Improper venting  
\item Expansion of sand
\end{enumerate}
\end{multicols}

\begin{multicols}{2}
\begin{enumerate} 
\item $P-1, Q-2, R-3, S-4$
\item $P-3, Q-4, R-1, S-2$
\item $P-4, Q-3, R-2, S-1$
\item $P-2, Q-1, R-3, S-4$
\end{enumerate}
\end{multicols}
\hfill(GATE MT 2008)

\item The thickness of a plate is to be reduced from 60 to 30 mm by multipass rolling. The roll radius is 350 mm and coefficient of friction is 0.15. Assuming equal draft in each pass, the minimum number of passes required would be
\begin{multicols}{4}
\begin{enumerate} 
\item $2$
\item $4$ 
\item $5$ 
\item $6$
\end{enumerate}
\end{multicols}
\hfill(GATE MT 2008)

\item Match the particle morphologies in Group 1 with the powder production methods in Group 2
\begin{multicols}{2}
\underline{Group 1}
\begin{enumerate}[label=(\Alph*), start=16]
\item Superalloy powder with rounded morphology 
\item Monosized spherical Ta powder
\item Fe powder with \textit{onion peel} structure
\item Irregularly shaped W powder
\end{enumerate}

\underline{Group 2}
\begin{enumerate}[label=(\arabic*), start=1]
\item Carbonyl process
\item Gas atomization
\item Oxide reduction
\item Rotating electrode process
\end{enumerate}
\end{multicols}

\begin{multicols}{2}
\begin{enumerate} 
\item $P-2, Q-1, R-4, S-3$
\item $P-1, Q-4, R-3, S-2$
\item $P-2, Q-4, R-1, S-3$
\item $P-4, Q-1, R-2, S-3$
\end{enumerate}
\end{multicols}
\hfill(GATE MT 2008)

\item One mole of monatomic ideal gas is reversibly and isothermally expanded at 1000 K to twice its original volume. The work done by the gas in Joules is
\begin{multicols}{4}
\begin{enumerate} 
\item $2430$ 
\item $2503$ 
\item $5006$ 
\item $5763$
\end{enumerate}
\end{multicols}
\hfill(GATE MT 2008)

\item In the Ellingham diagram C$\rightarrow$CO line intersects M$\rightarrow$MO line at temperature $T_1$ and N$\rightarrow$NO line at temperature $T_2$. M and N are metals. $T_2$ is greater than $T_1$. The correct statements among the following are:
\begin{enumerate}[label=(\MakeUppercase{\alph*}), start= 16]
\item carbon will reduce both MO and NO at temperatures $T_1 > T_2$
\item carbon will reduce both MO and NO at temperatures between $T_1 and T_2$
\item carbon will reduce both MO and NO at temperatures $T_2 < T_1$
\item carbon will reduce MO but not NO at temperatures between $T_1 and T_2$
\item carbon will reduce NO but not MO at temperatures between $T_1 and T_2$
\end{enumerate}

\begin{multicols}{2}
\begin{enumerate} 
\item P, S
\item Q, T
\item R, S
\item P, T
\end{enumerate}
\end{multicols}
\hfill(GATE MT 2008)

\item Match the forms of corrosion in Group 1 with the typical examples in Group 2
\begin{multicols}{2}
\underline{Group 1}
\begin{enumerate}[label=(\Alph*), start=16]
\item Filiform corrosion  
\item Crevice corrosion
\item Galvanic corrosion
\item Stress corrosion cracking 
\end{enumerate}

\underline{Group 2}
\begin{enumerate}[label=(\arabic*), start=1]
\item Austenitic stainless steel in chloride environment
\item Nut bolt with gasket
\item Painted food cans
\item Steel studs in copper plate
\end{enumerate}
\end{multicols}

\begin{multicols}{2}
\begin{enumerate} 
\item $P-3, Q-2, R-4, S-1$
\item $P-1, Q-3, R-4, S-2$
\item $P-3, Q-4, R-2, S-1$
\item $P-2, Q-3, R-4, S-1$
\end{enumerate}
\end{multicols}
\hfill(GATE MT 2008)

\item Given the following assertion 'a' and the reason \text{'r'}, the correct option is

\textbf{Assertion a:} Phosphorous removal in steelmaking is favoured by basic slag \\
\textbf{Reason r:} Basic slag decreases the activity of P$_2$O$_5$ in the slag

\begin{multicols}{2}
\begin{enumerate} 
\item Both a and r are true and r is the correct reason for a
\item Both a and r are true
\item a is true but r is false
\item Both a and r are true but r is not the correct reason for a
\end{enumerate}
\end{multicols}
\hfill(GATE MT 2008)
\item Given the following assertion \text{'a'} and the reason \text{'r'}, the correct option is

\textbf{Assertion a:} In Bayer's process high pressure is used to dissolve alumina from bauxite \\
\textbf{Reason r:} Pressure increases the boiling point of water

\begin{multicols}{2}
\begin{enumerate} 
\item Both a and r are correct, but r is not the correct reason for a
\item Both a and r are false
\item Both a and r are correct and r is the correct reason for a
\item a is true but r is false
\end{enumerate}
\end{multicols}
\hfill(GATE MT 2008) 

\item Match the alloys in Group 1 with the main precipitates responsible for hardening in Group 2
\begin{multicols}{2}
\underline{Group 1}
\begin{enumerate}[label=(\Alph*), start=16]
\item Al-4.4\%Cu-1.5\%Mg-0.6\%Mn
\item Fe-18.0\%Ni-8.5\%Co-3.5\%Mo-0.2\%Ti-0.1\%Al
\item Al-1.0\%Mg-0.6\%Si-0.3\%Cu-0.2\%Cr
\item Ni-15.0\%Cr-2.7\%Al-1.7\%Ti-1.0\%Fe
\end{enumerate}

\underline{Group 2}
\begin{enumerate}[label=(\arabic*), start=1]
\item Ni$_3$Mo
\item Mg$_2$Si
\item CuAl$_2$ 
\item TiAl$_3$
\end{enumerate}
\end{multicols}

\begin{multicols}{2}
\begin{enumerate} 
\item $P-3, Q-5, R-2, S-4$
\item $P-1, Q-3, R-2, S-4$
\item $P-4, Q-1, R-3, S-5$
\item $P-3, Q-1, R-2, S-5$
\end{enumerate}
\end{multicols}

\hfill(GATE MT 2008)
\item Identify the attributes associated with dispersion hardened alloys
\begin{enumerate}[label=(\MakeUppercase{\alph*}), start= 16]
\item dispersoids do not dissolve in the matrix even at high temperatures
\item dispersoids are coherent with the matrix
\item dispersoids impart creep resistance to the alloy
\item dispersoids improve the corrosion resistance of the alloy
\end{enumerate}
\begin{multicols}{2}
\begin{enumerate} 
\item P, S
\item Q, R
\item Q, S
\item P, R
\end{enumerate}
\end{multicols}
\hfill(GATE MT 2008)
\item In a gaseous mixture, CO, CO$_2$ and O$_2$ are in equilibrium at temperature T. 
For the reaction CO + $0.5O$$_2$ = CO$_2$, 
$\Delta G^\circ = -281,400 + 87.6T$ Joules. 
The correct statements among the following are:

\begin{enumerate}[label=(\MakeUppercase{\alph*}), start= 16]
\item The reaction will shift to left on increasing T
\item The reaction will shift to right on increasing T
\item The reaction will shift to left on increasing pressure
\item The reaction will shift to right on increasing pressure
\end{enumerate}

\begin{multicols}{2}
\begin{enumerate} 
\item P, S
\item P, Q
\item Q, R
\item R, S
\end{enumerate}
\end{multicols}
\hfill(GATE MT 2008)

\item The casting processes that require expendable moulds are
\begin{multicols}{1}
\begin{enumerate}[label=(\MakeUppercase{\alph*}), start= 16]
\item investment casting
\item low-pressure casting
\item shell moulding
\item slush casting
\end{enumerate}
\end{multicols}

\begin{multicols}{2}
\begin{enumerate} 
\item P, Q
\item Q, R
\item R, S
\item P, R
\end{enumerate}
\end{multicols}
\hfill(GATE MT 2008)
\item Transport mechanisms that do \textbf{NOT} contribute to densification during sintering are
\begin{enumerate}[label=(\MakeUppercase{\alph*}), start= 16]
\item surface diffusion
\item grain boundary diffusion
\item bulk diffusion
\item evaporation-condensation
\item viscous flow
\end{enumerate}

\begin{multicols}{2}
\begin{enumerate} 
\item P, Q
\item Q, S
\item Q, T
\item P, S
\end{enumerate}
\end{multicols}
\hfill(GATE MT 2008)
\item The order of decreasing weldability among the following steels is
\begin{enumerate}[label=(\MakeUppercase{\alph*}), start= 16]
\item Fe-$0.6$\%C
\item Fe-$0.4$\%C
\item HSLA
\end{enumerate}

\begin{multicols}{2}
\begin{enumerate} 
\item R $\rightarrow$ Q $\rightarrow$ P
\item P $\rightarrow$ Q $\rightarrow$ R
\item Q $\rightarrow$ P $\rightarrow$ R
\item Q $\rightarrow$ R $\rightarrow$ P
\end{enumerate}
\end{multicols}
\hfill(GATE MT 2008)
\item Match the welding processes in Group 1 with the sources of heat in Group 2

\begin{multicols}{2}
\underline{Group 1}
\begin{enumerate}[label=(\Alph*), start=16]
\item Ultrasonic welding
\item Spot welding
\item SMAW
\item Thermit welding
\end{enumerate}

\underline{Group 2}
\begin{enumerate}[label=(\arabic*), start=1]
\item Thermochemical 
\item Electrical resistance
\item Conversion of matte to metal
\item Friction
\item Electrical arc
\end{enumerate}
\end{multicols}

\begin{multicols}{2}
\begin{enumerate} 
\item P-3, Q-2, R-1, S-4
\item P-4, Q-3, R-2, S-1
\item P-1, Q-3, R-4, S-2
\item P-3, Q-2, R-4, S-1
\end{enumerate}
\end{multicols}
\hfill(GATE MT 2008)

\item A cup is to be made from a 2 mm thick metal sheet by deep-drawing. 
The height of the cup is 75 mm and the inside diameter is 100 mm. 
For a drawing ratio of 1.25, the blank diameter in mm is
\begin{multicols}{4}
\begin{enumerate} 
\item $62.5$
\item $125$
\item $225$
\item $250$
\end{enumerate}
\end{multicols}
\hfill(GATE MT 2008)

\item The defects that are \textbf{NOT} observed in extruded products are
\begin{enumerate}[label=(\MakeUppercase{\alph*}), start=16]
\item chevron cracking
\item fold
\item piping
\item surface cracking
\item alligatoring
\end{enumerate}

\begin{multicols}{2}
\begin{enumerate} 
\item P, Q
\item R, T
\item P, S
\item Q, T
\end{enumerate}
\end{multicols}
\hfill(GATE MT 2008)
\item Oil impregnated bronze bearings are manufactured using
\begin{multicols}{2}
\begin{enumerate} 
\item pressure die casting
\item centrifugal casting
\item solid-state sintering
\item liquid phase sintering
\end{enumerate}
\end{multicols}


\textbf{Common Data Questions}

\textbf{Common Data for Questions 71, 72 and 73:}

The diffusivities of carbon in $\gamma$-iron at 1173 K and 1273 K are $5.90 \times 10^{-12}$ and $1.94 \times 10^{-11}$ m$^2$s$^{-1}$, respectively.
\hfill(GATE MT 2008)
\item The activation energy for diffusion in kJ mol$^{-1}$ is
\begin{multicols}{4}
\begin{enumerate} 
\item $138$
\item $148$
\item $158$
\item $168$
\end{enumerate}
\end{multicols}
\hfill(GATE MT 2008)
\item The diffusivity of carbon in $\gamma$-iron at 1373 K in m$^2$s$^{-1}$ is
\begin{multicols}{4}
\begin{enumerate} 
\item $3.4$ $\times 10^{-11}$
\item $4.4$ $\times 10^{-11}$
\item $5.4$ $\times 10^{-11}$
\item $6.4$ $\times 10^{-11}$
\end{enumerate}
\end{multicols}
\hfill(GATE MT 2008)
\item During the carburization of a steel, a case depth of d has been obtained in 40 hours at 1173 K. 
For achieving a case depth of d/2 at 1273 K, the time required in hours is
\begin{multicols}{4}
\begin{enumerate} 
\item$ 1$
\item $2$
\item $3$
\item$ 4$
\end{enumerate}
\end{multicols}


\textbf{Common Data for Questions 74 and 75:}

A copper alloy powder has an apparent density of 3000 kg m$^{-3}$ and tap density of 4500 kg m$^{-3}$. 
The powder is compacted in a cylindrical die at 300 MPa to a green density of 6000 kg m$^{-3}$. 
Subsequently, the compact is sintered to a density of 7500 kg m$^{-3}$. 
Th(GATE MT 2008)
\item If the powder is compressed to 10 mm height, the initial fill height in mm is
\begin{multicols}{4}
\begin{enumerate} 
\item $12$
\item $15$
\item $20$
\item $25$
\end{enumerate}
\end{multicols}
\hfill(GATE MT 2008)
\item The densification parameter of the sintered compact is
\begin{multicols}{4}
\begin{enumerate} 
\item $0.50$
\item $0.67$
\item $0.75$
\item $0.83$
\end{enumerate}
\end{multicols}


\textbf{Linked Answer Questions: Q.76 to Q.85 carry two marks each.}

\textbf{Statement for Linked Answer Questions 76 and 77:}

A polyester-matrix composite is unidirectionally reinforced with 60 vol.\% of E-glass fibers. 
The elastic moduli of the matrix and the fiber are 6.9 and 72.4 GPa, respectively.
\hfill(GATE MT 2008)
\item The elastic modulus of the composite parallel to the fiber direction in GPa is
\begin{multicols}{4}
\begin{enumerate} 
\item $15.1$
\item $23.1$
\item $43.4$
\item $46.2$
\end{enumerate}
\end{multicols}
\hfill(GATE MT 2008)
\item If a load of 100 kg is applied on the composite in the fiber direction, the load carried by the fibers in kg is
\begin{multicols}{4}
\begin{enumerate} 
\item $6$
\item $47$
\item $94$
\item $100$
\end{enumerate}
\end{multicols}
\hfill(GATE MT 2008)


\textbf{Statement for Linked Answer Questions 78 and 79:}

1000 kg of zinc concentrate of composition 78\% ZnS and 22\% inerts is roasted in a multiple hearth furnace.
Roasting converts ZnS to ZnO, SO$_2$ and SO$_3$. The exit gas contains 6 vol.\% SO$_2$ and 2 vol.\% SO$_3$. \\
Molecular weights: Zn = 65, S = 32, O$_2$ = 32. \\
Composition of air (in vol.\%) = 21\% O$_2$ and 79\% N$_2$. \\
1 kg mol of gas occupies 22.4 m$^3$ at 273 K and 1 atm.

\item Volume of the exit gas (at 1 atm pressure and 273 K) in m$^3$ is
\begin{multicols}{4}
\begin{enumerate} 
\item $2129$
\item $2252$
\item $2628$
\item $2923$
\end{enumerate}
\end{multicols}
\hfill(GATE MT 2008)
\item Stoichiometric amount of air used (at 1 atm pressure and 273 K) in m$^3$ is
\begin{multicols}{4}
\begin{enumerate} 
\item $1010$
\item $1394$
\item $1520$
\item $2020$
\end{enumerate}
\end{multicols}
\hfill(GATE MT 2008)\\


\textbf{Statement for Linked Answer Questions 80 and 81:}

Density of Al = 2700 kg m$^{-3}$, atomic weight of Al = 27, density of Al$_2$O$_3$ = 3700 kg m$^{-3}$.

\item The Pilling-Bedworth ratio for the oxidation of Al is
\begin{multicols}{4}
\begin{enumerate} 
\item $0.57$
\item $0.74$
\item $1.38$
\item $3.12$
\end{enumerate}
\end{multicols}
\hfill(GATE MT 2008)
\item The oxidation law that governs the high temperature oxidation of Al is
\vspace{-0.9em}
\begin{multicols}{4}
\begin{enumerate} 
\item parabolic
\item linear
\item logarithmic
\item paralinear
\end{enumerate}
\end{multicols}
\hfill(GATE MT 2008)\\


\textbf{Statement for Linked Answer Questions 82 and 83:}

In the diffraction pattern of a fcc metal obtained using CuK$_\alpha$ radiation (wavelength of 0.154 nm), 
a diffraction peak appears at 2$\theta$ of 58.4$^\circ$. The lattice parameter of the crystal is 0.316 nm.

\item The interplanar spacing in nm is

\begin{multicols}{4}
\begin{enumerate} 
\item $0.158$
\item $0.164$
\item $0.177$
\item $0.185$
\end{enumerate}
\end{multicols}

\hfill(GATE MT 2008)

\item The Miller indices of the reflecting plane are
\vspace{-0.9em}
\begin{multicols}{4}
\begin{enumerate} 
\item \brak{111}
\item \brak{200}
\item \brak{220}
\item \brak{222}
\end{enumerate}
\end{multicols}
\hfill(GATE MT 2008)\\



\textbf{Statement for Linked Answer Questions 84 and 85:}

Mg casting with a volume to surface area ratio (casting modulus) of 0.1 m is made by gravity die casting.
Heat transfer coefficient at the metal-mould interface is 1.9 kJ m$^{-2}$ K$^{-1}$ s$^{-1}$. 
The density and melting point of Mg are 1700 kg m$^{-3}$ and 923 K, respectively. 
Assume ambient temperature to be 293 K.

\item If the solidification time is 50 s, the latent heat of fusion in kJ mol$^{-1}$ is

\begin{multicols}{4}
\begin{enumerate} 
\item $300$
\item $352$
\item $472$
\item $532$
\end{enumerate}
\end{multicols}

\hfill(GATE MT 2008)

\item In a spiral channel of 10 mm diameter and with an entrance flow velocity of 300 mm s$^{-1}$, the fluidity of the melt in mm is
\vspace{-0.9em}
\begin{multicols}{4}
\begin{enumerate} 
\item $75$
\item $175$
\item $275$
\item $375$
\end{enumerate}
\end{multicols}
\hfill(GATE MT 2008)
\end{enumerate}

\textbf{END OF THE QUESTION PAPER}



\end{document}