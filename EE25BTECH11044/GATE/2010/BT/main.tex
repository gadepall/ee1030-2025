%iffalse
\let\negmedspace\undefined
\let\negthickspace\undefined
\documentclass[journal,12pt,onecolumn]{IEEEtran}
\usepackage{cite}
\usepackage{amsmath,amssymb,amsfonts,amsthm}
\usepackage{algorithmic}
\usepackage{graphicx}
\usepackage{textcomp}
\usepackage{xcolor}
\usepackage{txfonts}
\usepackage{listings}
\usepackage{enumitem}
\usepackage{mathtools}
\usepackage{gensymb}
\usepackage{comment}
\usepackage[breaklinks=true]{hyperref}
\usepackage{tkz-euclide} 
\usepackage{listings}
\usepackage{gvv}                                        
% \usepackage{gvv}  
\usepackage[latin1] {inputenc}
\usepackage{xparse}
\usepackage{color}                                            
\usepackage{array}                                            
\usepackage{longtable}                                       
\usepackage{calc}                                             
\usepackage{multirow}
\usepackage{multicol}
\usepackage{hhline}                                           
\usepackage{ifthen}                                           
\usepackage{lscape}
\usepackage{tabularx}
\usepackage{array}
\usepackage{float}
\newtheorem{theorem}{Theorem}[section]
\newtheorem{problem}{Problem}
\newtheorem{proposition}{Proposition}[section]
\newtheorem{lemma}{Lemma}[section]
\newtheorem{corollary}[theorem]{Corollary}
\newtheorem{example}{Example}[section]
\newtheorem{definition}[problem]{Definition}
\newcommand{\BEQA}{\begin{eqnarray}}
\newcommand{\EEQA}{\end{eqnarray}}
\usepackage{float}
%\newcommand{\define}{\stackrel{\triangle}{=}}
\theoremstyle{remark}
\usepackage{ circuitikz }
%\newtheorem{rem}{Remark}
% Marks the beginning of the document
\begin{document}
\title{GATE BT 2010}
\author{EE25BTECH11044 - Pappula Sai Hasini}
\maketitle
\renewcommand{\thefigure}{\theenumi}
\renewcommand{\thetable}{\theenumi}
%GATE BT 2010
\begin{enumerate}


\item Hybridoma technology is used to produce  
\begin{enumerate}
      \item monoclonal antibodies  
      \item polyclonal antibodies  
      \item both monoclonal and polyclonal antibodies  
      \item B cells  
\end{enumerate}
\hfill (GATE BT 2010)

\item Ames test is used to determine  
\begin{enumerate}
     \item the mutagenicity of a chemical  
     \item carcinogenicity of a chemical  
     \item both mutagenicity and carcinogenicity of a chemical  
     \item toxicity of a chemical  
\end{enumerate}
\hfill (GATE BT 2010)

\item The bacteria known to be naturally competent for transformation of DNA is  
\begin{enumerate}
    \item Escherichia coli  
    \item Bacillus subtilis  
    \item Mycobacterium tuberculosis  
    \item Yersinia pestis  
\end{enumerate}
\hfill (GATE BT 2010)

\item 
Antibiotic resistance marker that CANNOT be used in a cloning vector in Gram negative bacteria is  
\begin{enumerate}
    \item Streptomycin  
    \item Ampicillin  
    \item Vancomycin  
    \item Kanamycin  
\end{enumerate}
\hfill (GATE BT 2010)

\item Program used for essentially local similarity search is  
\begin{enumerate}
    \item BLAST  
    \item RasMol  
    \item ExPASy  
    \item SWISS-PROT  
\end{enumerate}
\hfill (GATE BT 2010)

\item Peptidyl transferase activity resides in  
\begin{enumerate}
    \item $16S$ rRNA  
    \item $23S$ rRNA  
    \item $5S$ rRNA  
    \item $28S$ rRNA  
\end{enumerate}
\hfill (GATE BT 2010)

\item In transgenics, alterations in the sequence of nucleotide in genes are due to  

P. Substitution  
Q. Deletion  
R. Insertion  
S. Rearrangement  

\begin{enumerate}
    \item P and Q  
    \item P, Q and R  
    \item Q and R  
    \item R and S  
\end{enumerate}
\hfill (GATE BT 2010)

\item During transcription  
\begin{enumerate}
    \item DNA Gyrase introduces negative supercoils and DNA Topoisomerase I removes negative supercoils  
    \item DNA Topoisomerase I introduces negative supercoils and DNA Gyrase removes negative supercoils  
    \item both DNA Gyrase and DNA Topoisomerase I introduce negative supercoils  
    \item both DNA Gyrase and DNA Topoisomerase I remove negative supercoils  
\end{enumerate}
\hfill (GATE BT 2010)

\item 
Under stress conditions bacteria accumulate  
\begin{enumerate}
    \item ppGpp (Guanosine tetraphosphate)  
    \item pppGpp (Guanosine pentaphosphate)  
    \item both ppGpp and pppGpp  
    \item either ppGpp or pppGpp  
\end{enumerate}
\hfill (GATE BT 2010)

\item An example for template independent DNA polymerase is  
\begin{enumerate}
    \item DNA Polymerase I  
    \item RNA polymerase  
    \item Terminal deoxynucleotidyl transferase  
    \item DNA polymerase III  
\end{enumerate}
\hfill (GATE BT 2010)

\item Which one of the following DOES NOT belong to the domain of Bacteria?  
\begin{enumerate}
    \item Cyanobacteria  
    \item Proteobacteria  
    \item Bacteroides  
    \item Methanobacterium  
\end{enumerate}
\hfill (GATE BT 2010)

\item Interferon-$\beta$ is produced by  
\begin{enumerate}
    \item bacteria infected cells  
    \item virus infected cells  
    \item both virus and bacteria infected cells  
    \item fungi infected cells  
\end{enumerate}
\hfill (GATE BT 2010)

\item A culture of bacteria is infected with bacteriophage at a multiplicity of $0.3$.  
The probability of a single cell infected with $3$ phages is  
\begin{enumerate}
    \item $0.9$  
    \item $0.27$  
    \item $0.009$  
    \item $0.027$  
\end{enumerate}
\hfill (GATE BT 2010)

\item A neonatally thymectomized mouse, immunized with protein antigen shows  
\begin{enumerate}
    \item both primary and secondary responses to the antigen  
    \item only primary response to the antigen  
    \item delayed type hypersensitive reactions  
    \item no response to the antigen  
\end{enumerate}
\hfill (GATE BT 2010)

\item Lymphocytes interact with foreign antigens in  
\begin{enumerate}
   \item Bone marrow  
   \item Peripheral blood  
   \item Thymus  
   \item Lymph nodes  
\end{enumerate}
\hfill (GATE BT 2010)

\item Somatic cell gene transfer is used for  

P. transgenic animal production  
Q. transgenic diploid cell production  
R. in-vitro fertilization  
S. classical breeding of farm animals  

\begin{enumerate}
    \item P, R and S  
    \item P, Q and R  
    \item P and R  
    \item P only  
\end{enumerate}
\hfill (GATE BT 2010)

\item Accession number is a unique identification assigned to a  
\begin{enumerate}
   \item single database entry for DNA/Protein  
   \item single database entry for DNA only  
   \item single database entry for Protein only  
   \item multiple database entry for DNA/Protein  
\end{enumerate}
\hfill (GATE BT 2010)

\item Expressed Sequence Tag is defined as  
\begin{enumerate}
    \item a partial sequence of a codon randomly selected from cDNA library  
    \item the characteristic gene expressed in the cell  
    \item the protein coding DNA sequence of a gene  
    \item uncharacterized fragment of DNA present in the cell  
\end{enumerate}
\hfill (GATE BT 2010)

\item In a chemostat operating under steady state, a bacterial culture can be grown at dilution rate higher than maximum growth rate by  
\begin{enumerate}
    \item partial cell recycling  
    \item using sub-optimal temperature  
    \item pH cycling  
    \item substrate feed rate cycling  
\end{enumerate}
\hfill (GATE BT 2010)

\item During lactic acid fermentation, net yield of ATP and NADH per mole of glucose is  
\begin{enumerate}
   \item 2 ATP and 2 NADH  
   \item 2 ATP and 0 NADH  
   \item 4 ATP and 2 NADH  
   \item 4 ATP and 0 NADH  
\end{enumerate}
\hfill (GATE BT 2010)

\item Identify the enzyme that catalyzes the following reaction  
\[
\alpha\text{-}Ketoglutarate + NADH + NH_{4}^{+} + H^{+}
\;\longrightarrow\;
Glutamate + NAD^{+} + H_{2}O
\]

\begin{enumerate}
   \item Glutamate synthetase  
   \item Glutamate oxoglutarate aminotransferase  
   \item Glutamate dehydrogenase  
   \item $\alpha$-ketoglutarate deaminase  
\end{enumerate}
\hfill (GATE BT 2010)

\item The degree of inhibition for an enzyme catalyzed reaction at a particular inhibitor concentration is independent of initial substrate concentration. The inhibition follows  
\begin{enumerate}
   \item competitive inhibition  
   \item mixed inhibition  
   \item un-competitive inhibition  
   \item non-competitive inhibition  
\end{enumerate}
\hfill (GATE BT 2010)

\item Oxidation reduction reactions with positive standard redox potential $(\Delta E^{0})$ have  
\begin{enumerate}
   \item positive $\Delta G^{0}$  
   \item negative $\Delta G^{0}$  
   \item positive $\Delta E^{0}$  
   \item negative $\Delta E^{0}$  
\end{enumerate}
\hfill (GATE BT 2010)

\item Nuclease-hypersensitive sites in the chromosomes are sites that appear to be  
\begin{enumerate}
   \item H2 and H4 histone free  
   \item H1 and H2 histone free  
   \item H3 and H4 histone free  
   \item Nucleosome free  
\end{enumerate}
\hfill (GATE BT 2010)

\item The formation of peptide cross-links between adjacent glycan chains in cell wall synthesis is called  
\begin{enumerate}
   \item Transglycosylation  
   \item Autoglycosylation  
   \item Autopeptidation  
   \item Transpeptidation  
\end{enumerate}
\hfill (GATE BT 2010)

\item An immobilized enzyme being used in a continuous plug flow reactor exhibits an effectiveness factor $(\eta)$ of $1.2$. The value of $\eta$ being greater than $1.0$ could be apparently due to  
\begin{enumerate}
   \item substrate inhibited kinetics with internal pore diffusion limitation  
   \item external pore diffusion limitation  
   \item sigmoidal kinetics  
   \item unstability of the enzyme  
\end{enumerate}
\hfill (GATE BT 2010)

\item A roller bottle culture vessel perfectly cylindrical in shape having inner radius $(r) = 10 \, cm$ and length $(l) = 20 \, cm$ was fitted with a spiral film of length $(L) = 30 \, cm$ and width $(W) = 20 \, cm$. If the film can support $10^5$ anchorage dependent cells per $cm^2$, the increase in the surface area after fitting the spiral film and the additional number of cells that can be grown respectively are  
\begin{enumerate}
\item $1200 \, cm^2$ and $1.2 \times 10^7$ cells  
\item $600 \, cm^2$ and $6 \times 10^7$ cells  
\item $600 \, cm^2$ and $8.3 \times 10^7$ cells  
\item $1200 \, cm^2$ and $8.3 \times 10^7$ cells  
\end{enumerate}
\hfill (GATE BT 2010)


\item Determine the correctness or otherwise of the following Assertion (a) and the Reason (r)  

Assertion: MTT assay is used to determine cell viability based on the principle of colour formation by DNA fragmentation.  

Reason: MTT assay is used to determine cell viability based on the colour development by converting tetrazolium soluble salt to insoluble salt.  

\begin{enumerate}
   \item both (a) and (r) are true and (r) is the correct reason for (a)  
   \item both (a) and (r) are true and (r) is not the correct reason for (a)  
   \item (a) is true but (r) is false  
   \item (a) is false but (r) is true  
\end{enumerate}
\hfill (GATE BT 2010)

\item Match the following antibiotics in Group I with their mode of action in Group II  

P.\ Chloramphenicol \hspace{2cm} 1.\ Inhibits peptidyl transferase \\  
Q.\ Norfloxacin \hspace{2.9cm} 2.\ Binds to RNA Polymerase \\  
R.\ Puromycin \hspace{3.1cm} 3.\ Mimics aminoacyl-tRNA \\  
S.\ Rifampicin \hspace{3.3cm} 4.\ Binds to DNA gyrase \\  

\begin{enumerate}
\item P-1, Q-3, R-2, S-4  
\item P-3, Q-1, R-2, S-4  
\item P-3, Q-1, R-4, S-2  
\item P-4, Q-2, R-3, S-1  
\end{enumerate}
\hfill (GATE BT 2010)


\item Match the chemicals in Group I with the possible type/class in Group II  

P.\ Picloram \hspace{2.9cm} 1.\ Vitamin \\  
Q.\ Zeatin \hspace{3.5cm} 2.\ Auxin \\  
R.\ Thiamine \hspace{3cm} 3.\ Amino Acid \\  
S.\ Glutamine \hspace{2.6cm} 4.\ Cytokinin \\  

\begin{enumerate}
   \item P-2, Q-4, R-1, S-3  
   \item P-4, Q-1, R-2, S-3  
   \item P-3, Q-1, R-2, S-4  
   \item P-4, Q-2, R-1, S-3  
\end{enumerate}
\hfill (GATE BT 2010)


\item Match Group I with Group II  

Group I \hspace{3cm} Group II  

P. Fibronectin \hspace{2.7cm} 1. Uptake of amino acids and glucose  
Q. Insulin \hspace{3.7cm} 2. Trypsin inhibitor  
R. $\alpha$-Macroglobulin \hspace{1.6cm} 3. Binds iron  
S. Transferrin \hspace{3.2cm} 4. Cell attachment to substratum  

\begin{enumerate}
   \item P-2, Q-1, R-4, S-3  
   \item P-3, Q-2, R-1, S-4  
   \item P-4, Q-2, R-1, S-3  
   \item P-4, Q-1, R-2, S-3  
\end{enumerate}
\hfill (GATE BT 2010)

\item Match the promoters listed in Group I with the tissues listed in Group II  

Group I \hspace{3cm} Group II  

P. $\alpha$-Amylase \hspace{2.6cm} 1. Endosperm  
Q. Glutelin \hspace{3.6cm} 2. Tuber  
R. Phaseollin \hspace{3.1cm} 3. Aleurone  
S. Patatin \hspace{3.7cm} 4. Cotyledon  

\begin{enumerate}
   \item P-3, Q-1, R-4, S-2  
   \item P-3, Q-4, R-1, S-2  
   \item P-4, Q-2, R-1, S-3  
   \item P-1, Q-3, R-2, S-4  
\end{enumerate}
\hfill (GATE BT 2010)

\item Consider the following statements.  

I. T4 DNA ligase can catalyze blunt end ligation more efficiently than \textit{E. coli} DNA ligase.  
II. The ligation efficiency of T4 DNA ligase can be increased with PEG and ficoll.  

\begin{enumerate}
   \item only I is true  
   \item both I and II are true  
   \item only II is true  
   \item I is true and II is false  
\end{enumerate}
\hfill (GATE BT 2010)

\item The turnover numbers for the enzymes, E1 and E2 are $150 \, s^{-1}$ and $15 \, s^{-1}$ respectively. This means  

\begin{enumerate}
   \item E1 binds to its substrate with higher affinity than E2  
   \item The velocity of reactions catalyzed by E1 and E2 at their respective saturating substrate concentrations could be equal, if concentration of E2 used is 10 times that of E1  
   \item The velocity of E1 catalyzed reaction is always greater than that of E2  
   \item The velocity of E1 catalyzed reaction at a particular enzyme concentration and saturating substrate concentration is lower than that of E2 catalyzed reaction under the same conditions  
\end{enumerate}
\hfill (GATE BT 2010)

\item Match the items in Group I with Group II  

Group I (Vectors) \hspace{2cm} Group II (Maximum DNA packaging)  

P.\ $\lambda$ phage\hspace{2.6cm} 1.\ 35--45 kb\\
Q.\ Bacterial Artificial Chromosomes (BACs)\hspace{0.8cm} 2.\ 100--300 kb\\
R.\ PI derived Artificial Chromosomes (PACs)\hspace{0.7cm} 3.\ $\leq 300$ kb\\
S.\ $\lambda$ cosmid\hspace{2.7cm} 4.\ 5--25 kb

\begin{enumerate}
   \item P-3, Q-4, R-1, S-2  
   \item P-1, Q-3, R-2, S-4  
   \item P-4, Q-3, R-2, S-1  
   \item P-1, Q-2, R-3, S-4  
\end{enumerate}
\hfill (GATE BT 2010)

\item Match Group I with Group II  

Group I \hspace{3.5cm} Group II  

P. Staphylococcus aureus \hspace{2.0cm} 1. Biofilms  
Q. Candida albicans \hspace{2.8cm} 2. Bacteriocins  
R. Mycobacterium tuberculosis \hspace{1.2cm} 3. Methicillin resistance  
S. Lactobacillus lactis \hspace{2.1cm} 4. Isoniazid  

\begin{enumerate}
   \item P-1, Q-4, R-2, S-3  
   \item P-2, Q-3, R-1, S-4  
   \item P-3, Q-1, R-4, S-2  
   \item P-1, Q-2, R-4, S-3  
\end{enumerate}
\hfill (GATE BT 2010)

\item A mutant G$_{\alpha}$ protein with increased GTPase activity would  

\begin{enumerate}
   \item not bind to GTP  
   \item not bind to GDP  
   \item show increased signaling  
   \item show decreased signaling  
\end{enumerate}
\hfill (GATE BT 2010)

\item Dizygotic twins are connected to a single placenta during their embryonic development. These twins  

\begin{enumerate}
   \item have identical MHC haplotypes  
   \item have identical T$_H$ cells  
   \item have identical T cells  
   \item can accept grafts from each other (both (A) and (B))  
\end{enumerate}
\hfill (GATE BT 2010)

\item The dissociation constant $K_d$ for ligand binding to the receptor is $10^{-7}$ M. The concentration of ligand required for occupying 10\% of receptors is  

\begin{enumerate}
  \item $10^{-6}$ M  
  \item $10^{-7}$ M  
  \item $10^{-8}$ M  
  \item $10^{-9}$ M  
\end{enumerate}
\hfill (GATE BT 2010)

\item Receptor R is overexpressed in CHO cells and analysed for expression. $6 \times 10^7$ cells were incubated with its radioactive ligand (specific activity 100 counts per picomole). If the total counts present in cell pellet was 1000 cpm, the average number of receptors R per cell is (assume complete saturation of receptors with ligand and one ligand binds to one receptor)  

\begin{enumerate}
   \item $10^4$  
   \item $10^5$  
   \item $10^6$  
   \item $10^7$  
\end{enumerate}
\hfill (GATE BT 2010)

\item A cell has five molecules of a rare mRNA. Each cell contains $4 \times 10^5$ mRNA molecules. How many clones one will need to screen to have 99\% probability of finding at least one recombinant cDNA of the rare mRNA, after making cDNA library from such cell?  

\begin{enumerate}
   \item $4.50 \times 10^5$  
   \item $3.50 \times 10^5$  
   \item $4.20 \times 10^5$  
   \item $4.05 \times 10^5$  
\end{enumerate}
\hfill (GATE BT 2010)

\item Match the products in Group I with the microbial cultures in Group II used for their industrial production

Group I \hspace{3cm} Group II

P. Gluconic acid \hspace{3.1cm} 1. Leuconostoc mesenteroides
Q. L - Lysine \hspace{4.2cm} 2. Aspergillus niger
R. Dextran \hspace{4.5cm} 3. Brevibacterium flavum
S. Cellulase \hspace{4.3cm} 4. Trichoderma reesei

\begin{enumerate}
   \item P-2, Q-1, R-3, S-4
   \item P-1, Q-3, R-4, S-2
   \item P-2, Q-3, R-1, S-4
   \item P-3, Q-2, R-4, S-1
\end{enumerate}
\hfill (GATE BT 2010)

\item Determine the correctness or otherwise of the following Assertion (a) and the Reason (r)

Assertion: Cytoplasmic male sterility (cms) is invariably due to defect(s) in mitochondrial function.

Reason: cms can be overcome by pollinating a fertility restoring (Rf) plant with pollen from a non cms plant.

\begin{enumerate}
  \item both (a) and (r) are true and (r) is the correct reason for (a)
  \item both (a) and (r) are true and (r) is not the correct reason for (a)
  \item (a) is false but (r) is true
  \item (a) is true but (r) is false
\end{enumerate}
\hfill (GATE BT 2010)

\item Thermal death of microorganisms in the liquid medium follows first order kinetics. If the initial cell concentration in the fermentation medium is $10^{6}$ cells/ml and the final acceptable contamination level is $10^{3}$ cells, for how long should $1\,\mathrm{m}^{3}$ medium be treated at temperature of $120^{\circ}$ (thermal deactivation rate constant $= 0.23\ \mathrm{min}^{-1}$) to achieve acceptable load?

\begin{enumerate}
   \item $48$ min
   \item $11$ min
   \item $110$ min
   \item $20$ min
\end{enumerate}
\hfill (GATE BT 2010)

\item True breeding Drosophila flies with curved wings and dark bodies were mated with true breeding short wings and tan body Drosophila. The F$_1$ progeny was observed to be with curved wings and tan body. The F$_1$ progeny was again allowed to breed and produced flies of the following phenotype, $45$ curved wings tan body, $15$ short wings tan body, $16$ curved wings dark body, and $6$ short wings dark body.

The mode of inheritance is
\begin{enumerate}
   \item Typical Mendelian with curved wings and tan body being dominant
   \item Typical non-Mendelian with curved wings and tan body not following any pattern
   \item Mendelian with suppression of phenotypes
   \item Mendelian with single crossover
\end{enumerate}
\hfill (GATE BT 2010)

\item Match Group I with Group II

Group I \hspace{3cm} Group II

P. Real Time-PCR \hspace{3.2cm} 1. Biochips
Q. 2-D Electrophoresis \hspace{2.4cm} 2. Syber Green
R. Affinity chromatography \hspace{2.2cm} 3. Antibody linked sepharose beads
S. Microarray \hspace{3.6cm} 4. Ampholytes

\begin{enumerate}
   \item P-1, Q-2, R-4, S-3
   \item P-2, Q-3, R-4, S-1
   \item P-2, Q-4, R-3, S-1
   \item P-3, Q-2, R-1, S-4
\end{enumerate}
\hfill (GATE BT 2010)

\item A culture of Rhizobium is grown in a chemostat ($100\ m^{3}$ bioreactor). The feed contains $12\ g/L$ sucrose, $K_{s}$ for the organism is $0.2\ g/L$ and $\mu_{m}=0.3\ h^{-1}$.

The flow rate required to result in steady state concentration of sucrose as $1.5\ g/L$ in the bioreactor will be
\begin{enumerate}
   \item $15\ m^{3}\ h^{-1}$
   \item $26\ m^{3}\ h^{-1}$
   \item $2.6\ m^{3}\ h^{-1}$
   \item $150\ m^{3}\ h^{-1}$
\end{enumerate}
\hfill (GATE BT 2010)

\item If $Y_{x/s}=0.4\ g/g$ for the above culture and steady state cell concentration in the bioreactor is $4\ g/L$, the resulting substrate concentration will be
\begin{enumerate}
   \item $2\ g/L$
   \item $8\ g/L$
   \item $4\ g/L$
   \item $6\ g/L$
\end{enumerate}
\hfill (GATE BT 2010)

\item The width of the lipid bilayer membrane is $30\ \text{\AA}$. It is permeated by a protein which is a right handed $\alpha$-helix.  

The number of $\alpha$-helical turns permeating the membrane is
\begin{enumerate}
   \item $5.6$ turns
   \item $3.5$ turns
   \item $6.5$ turns
   \item $5.0$ turns
\end{enumerate}
\hfill (GATE BT 2010)

\item The number of amino acid residues present in the protein is
\begin{enumerate}
   \item $15$
   \item $18$
   \item $17$
   \item $20$
\end{enumerate}
\hfill (GATE BT 2010)

\item The standard redox potential values for two half-reactions are given below. The value for Faraday constant is $96.48\ kJ\ V^{-1}\ mol^{-1}$ and Gas constant $R=8.31\ J\ K^{-1}\ mol^{-1}$  

\[
NAD^{+} + H^{+} + 2e^{-} \leftrightarrow NADH \quad -0.315\ V
\]  
\[
FAD + 2H^{+} + 2e^{-} \leftrightarrow FADH_{2} \quad -0.219\ V
\]

The $\Delta G^{0}$ for the oxidation of NADH by FAD is
\begin{enumerate}
   \item $-9.25\ kJ\ mol^{-1}$
   \item $-103.04\ kJ\ mol^{-1}$
   \item $+51.52\ kJ\ mol^{-1}$
   \item $-18.5\ kJ\ mol^{-1}$
\end{enumerate}
\hfill (GATE BT 2010)

\item The value of $\Delta G'$, given $K_{eq}=1.7$ at $23^{\circ}C$ will be
\begin{enumerate}
  \item $-17.19\ kJ\ mol^{-1}$
  \item $-19.8\ kJ\ mol^{-1}$
  \item $+52.82\ kJ\ mol^{-1}$
  \item $-17.07\ kJ\ mol^{-1}$
\end{enumerate}
\hfill (GATE BT 2010)
Statement for Linked Answer Questions 54 and 55:

During bioconversion of sucrose to citric acid by Aspergillus niger, final samples of 6 batches of fermentation broth were analyzed for citric acid content. The results (in g/L) were found to be $47.3, 52.2, 49.2, 52.4, 49.1 and 46.3$.


\item
The mean value of acid concentration will be
\begin{enumerate}
   \item $49.4$
   \item $51.0$
   \item $48.2$
   \item $50.8$
\end{enumerate}
\hfill (GATE BT 2010)

\item
The standard deviation for the above results is
\begin{enumerate}
\item $2.49$
\item $3.0$
\item $1.84$
\item $5.91$
\end{enumerate}
\hfill (GATE BT 2010)
\end{enumerate}
\end{document}












