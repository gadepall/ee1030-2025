\documentclass[journal,onecolumn]{IEEEtran}
\usepackage[a5paper,margin=10mm]{geometry}

% Essential math and formatting packages
\usepackage{amsmath,amssymb,amsthm}
\usepackage{mathtools}
\usepackage{gensymb}
\usepackage{booktabs}
\usepackage{tikz}
\usetikzlibrary{arrows.meta,angles,quotes}

% For better referencing and citing
\usepackage{cite}
\usepackage[breaklinks=true]{hyperref}

% For vectors
\newcommand{\myvec}[1]{\begin{bmatrix}#1\end{bmatrix}}

% Set spacing and counters
\setlength{\headheight}{1cm}
\setlength{\headsep}{0mm}
\setlength{\intextsep}{10pt}

% Commented out as they cause errors without enumerate environment
%\numberwithin{equation}{enumi}
%\numberwithin{figure}{enumi}
%\numberwithin{table}{enumi}

% Title and author
\title{Coordinate Calculation for Point Dividing a Line Segment}
\author{AI25BTECH11034 - Sujal Chauhan}

\begin{document}

\maketitle

\textbf{Question:}\\
Find the coordinate of the point which divides the line segment joining points \(A(4,-3)\) and \(B(8,5)\) in the ratio \(3:1\) internally.\\[1cm]

\textbf{Solution:}\\
Let \(O\) be the origin. Then the position vectors 
\[
\overrightarrow{OA} = \myvec{4 \\ -3}, \quad \overrightarrow{OB} = \myvec{8 \\ 5}.
\]

The point \(C\), dividing the segment \(AB\) in the ratio \(3:1\) internally, has the position vector
\[
\overrightarrow{OC} = \frac{3 \overrightarrow{OB} + 1 \overrightarrow{OA}}{3 + 1} = \frac{3 \myvec{8 \\ 5} + \myvec{4 \\ -3}}{4} = \frac{\myvec{24 \\ 15} + \myvec{4 \\ -3}}{4} = \frac{\myvec{28 \\ 12}}{4} = \myvec{7 \\ 3}.
\]

Therefore, the coordinate of point \(C\) is \(\boxed{(7,3)}\).

\begin{figure}[htbp]
\centering
% \documentclass[tikz,border=10pt]{standalone}
\usepackage{tikz}
\usetikzlibrary{arrows.meta,calc,decorations.pathreplacing}

\begin{document}

\begin{tikzpicture}[>=Stealth, scale=1.0]

    % Define points
    \coordinate (O) at (0,0);
    \coordinate (A) at (4,-3);
    \coordinate (B) at (8,5);
    % Calculate C = (3B + A)/4
    \coordinate (C) at (7,3);

    % Draw axes
    \draw[->] (-1,0) -- (10,0) node[right] {\(x\)};
    \draw[->] (0,-5) -- (0,7) node[above] {\(y\)};

    % Draw vectors
    \draw[->, thick, blue] (O) -- (A) node[midway, below left] {\(\overrightarrow{OA}\)};
    \draw[->, thick, red] (O) -- (B) node[midway, above right] {\(\overrightarrow{OB}\)};
    \draw[->, thick, green!70!black] (O) -- (C) node[midway, right] {\(\overrightarrow{OC}\)};

    % Mark points
    \filldraw[black] (O) circle (2pt) node[below left] {\(O(0,0)\)};
    \filldraw[black] (A) circle (2pt) node[below right] {\(A(4,-3)\)};
    \filldraw[black] (B) circle (2pt) node[above right] {\(B(8,5)\)};
    \filldraw[black] (C) circle (2pt) node[right] {\(C(7,3)\)};

    % Optional: Draw segment AB with dashed line
    \draw[dashed, gray] (A) -- (B);

\end{tikzpicture}

\end{document}
 % Removed as file likely missing

% TikZ diagram inserted inline instead:
\begin{tikzpicture}[>=Stealth, scale=1.0]

    % Define points
    \coordinate (O) at (0,0);
    \coordinate (A) at (4,-3);
    \coordinate (B) at (8,5);
    \coordinate (C) at (7,3);

    % Draw axes
    \draw[->] (-1,0) -- (10,0) node[right] {\(x\)};
    \draw[->] (0,-5) -- (0,7) node[above] {\(y\)};

    % Draw vectors
    \draw[->, thick, blue] (O) -- (A) node[midway, below left] {\(\overrightarrow{OA}\)};
    \draw[->, thick, red] (O) -- (B) node[midway, above right] {\(\overrightarrow{OB}\)};
    \draw[->, thick, green!70!black] (O) -- (C) node[midway, right] {\(\overrightarrow{OC}\)};

    % Mark points
    \filldraw[black] (O) circle (2pt) node[below left] {\(O(0,0)\)};
    \filldraw[black] (A) circle (2pt) node[below right] {\(A(4,-3)\)};
    \filldraw[black] (B) circle (2pt) node[above right] {\(B(8,5)\)};
    \filldraw[black] (C) circle (2pt) node[right] {\(C(7,3)\)};

    % Dashed segment AB
    \draw[dashed, gray] (A) -- (B);

\end{tikzpicture}

\caption{}
\label{1.4.4}
\end{figure}

\end{document}
